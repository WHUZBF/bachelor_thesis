% !TeX program = xelatex

\documentclass[type = bachelor]{whu-thesis}

\whusetup{
	info = {
		title      = {纯旋量超弦},
		title*     = {Pure Spinor Superstring Theory},
		department = {物理科学与技术学院},
		department* = {School of Physics and Technology},
		author     = {郑卜凡},
		author*    = {Bufan Zheng},
		student-id = {2021302022016},
		supervisor = {杜一剑},
		supervisor* = {Yi-jian Du},
		supervisor-outer = {},
		academic-title = {副教授},
		academic-title* = {},
		academic-title-outer = {},
		subject = {},
		major   = {物理学},
		major* = {Physics},
		research-area = {理论物理},
		research-area* = {},
		year = 2025,
		keywords = {超弦理论,散射振幅,纯旋量,色-运动学对偶,量子场论},
		keywords* = {Superstring Theory,Scattering Amplitudes,Pure Spinor,Color-Kinematic Duality, Quantum Field Theory},
		month = 5,
		clc = O175.29,
		udc = 517.9,
	},
	style = {
		% footnote-style = libertinus-sans,
		font = times,
		math-font = default,
		cjk-font = windows,
		% library,
		% cjk-fakefont = true,
		bib-backend = bibtex,
		% bib-style = numerical,
		bib-resource = {ref/bachelor-refs},
		bachelor-encover = false, % 显示英文封面
		license = false
	}
}

% 宏包引入
\usepackage{physics}
\usepackage{amsmath}
%% 允许公式换页
\allowdisplaybreaks[4]
\usepackage{extarrows}
\usepackage{annotate-equations}
\usepackage{cancel}
\usepackage{slashed}
\usepackage{tcolorbox}
\usepackage{shuffle}
\usepackage{simpler-wick}
\usepackage{mathtools}

%自定义命令
%% 实现双重尖括号
\makeatletter
\newsavebox{\@brx}
\newcommand{\llangle}[1][]{\savebox{\@brx}{\(\m@th{#1\langle}\)}%
	\mathopen{\copy\@brx\kern-0.5\wd\@brx\usebox{\@brx}}}
\newcommand{\rrangle}[1][]{\savebox{\@brx}{\(\m@th{#1\rangle}\)}%
	\mathclose{\copy\@brx\kern-0.5\wd\@brx\usebox{\@brx}}}
\makeatother
%% 重要定理框注
\newenvironment{boxedtext}[1][定理] % 定义带参数的环境,参数数量为1(定理名称)
{
	\begin{tcolorbox}[
		colback=gray!10,                % 灰色背景
		colframe=black,                 % 黑色边框
		arc=1mm,                        % 圆角弧度
		auto outer arc,                 % 自动外弧
		boxrule=0.5pt,                  % 边框线宽
		title={\bfseries #1},           % 定理名称参数(加粗显示)
		coltitle=white,                 % 标题文字白色
		colbacktitle=black!70,          % 标题背景深灰色(非纯黑)
		]
	}
	{
	\end{tcolorbox}
}

\begin{document}
	%生成目录
	\tableofcontents
	% \listoffigures
	% \listoftables
	% 正文
	\mainmatter
	\chapter{引言}
弦理论是量子引力理论的重要候选者之一,有望统一四大基本相互作用。对弦论的研究最早起源于对强相互作用的研究,1968年,Veneziano给出了强子散射的经验公式,这被认为是弦理论的第一个公式,现代语境下其描述开弦四快子振幅。从某种角度上来说对弦论的研究起源于对弦振幅的研究。而且目前弦论只有微扰意义上的定义,所以弦振幅的研究尤其重要\cite{berkovits2022snowmasswhitepaperstring}。

最早被发展的弦理论是不含费米子谱的玻色弦理论,为了能够描述费米子并消除不稳定真空,需要引入超对称研究超弦理论。由于弦理论本身可以看作是世界面上的共形场论,所以引入超对称最简单的方式是直接引入世界面场论,利用二维$\mathcal{N}=1$ 超对称共形场论来构造超弦,这便是Ramond-Neveu-Schwarz(RNS)超弦。但最终弦振幅的结果和世界面没有关系,真正需要的超对称构造是靶空间超对称,在RNS超弦中这通过GSO投影实现。由于弦振幅本身是具有靶空间超对称而不是世界面超对称的物理量,所以直接使用RNS超弦计算振幅会十分复杂。比如两圈四点RNS超弦振幅计算,D'Hoker和Phong用了六篇论文才完全解决\cite{DHoker:2001kkt,DHoker:2001qqx,DHoker:2001foj,DHoker:2001jaf,DHoker:2005dys,DHoker:2005vch,DHoker:2002hof}。

虽然超弦已经经历过两次革命,但关于保持靶空间超对称的超弦理论的寻找还是个难题。最早Green-Schwarz超弦\cite{Green:1983wt,Green:1983sg}实现了靶空间超对称,但是只能在非协变的光锥坐标下量子化。后续Siegle改进了这一形式\cite{Siegel:1985xj}但最终被发现与RNS形式不等价,所以无法得到正确的弦振幅。本世纪初,Nathan Berkovits在Siegle超弦形式下成功发展出了能够协变量子化的保持靶空间超对称的超弦\cite{Berkovits:2000fe}。由于Berkovis的构造依赖于一种特殊的靶空间旋量——纯旋量,所以这一形式也被称为纯旋量超弦。

利用这一形式,原本在RNS超弦中难以计算的弦振幅被大大简化,比如利用纯旋量超弦计算两圈四点振幅要容易得多\cite{Berkovits:2005df},后来这也被证明与RNS超弦的计算结果等价\cite{Berkovits:2005ng}。利用纯旋量超弦,Berkovits的学生Mafra,和Schlotterer、Stieberger合作给出了任意点开弦无质量态盘面振幅的一般公式\cite{Mafra:2011nv,Mafra:2011nw}:
\begin{equation}
	\mathcal{A}_{n}(P)=(2\alpha^{\prime})^{n-3}\int d\mu_{P}^{n}\left[\prod_{k=2}^{n-2}\sum_{m=1}^{k-1}\frac{s_{mk}}{z_{mk}}A^{\text{SYM}}_{n}(1,2,\ldots,n)+\mathrm{perm}(2,3,\ldots,n-2)\right]
\end{equation}
这是本论文的核心,本论文将聚焦于弦论盘面振幅的计算,尤其是如何从纯旋量超弦导出此公式。下面我将给出本论文的行文结构,方便读者阅读。

第\ref{chap:2}章首先介绍了玻色弦理论,特别是正则量子化、路径积分量子化和BRST量子化三种后面会经常用到的量子化方法;第\ref{chap:3}章介绍RNS超弦和GSO投影,详细构造了RNS超弦无质量顶角算符。这两章可以看作是对弦理论的简介,本文仅仅假设读者对量子场论和场论中的散射振幅有基本的了解。不过篇幅所限这部分略去了很多相关计算细节,读者还需参考弦论教科书。\cite{Polchinski:1998rq,Polchinski:1998rr,Blumenhagen:2013fgp,Becker:2006dvp,Green:2012oqa,Green:2012pqa,Cecotti:2023dnp,Kiritsis:2019npv}

由于弦振幅涉及到黎曼曲面模空间的计算,所以第\ref{chap:4}章首先简短的从数学上介绍黎曼曲面及其模空间。不过本论文主要考虑球面和盘面振幅,所以模空间是计算是平凡的。为了完整性,在第四章依旧介绍了玻色弦任意圈弦振幅的数学形式,虽然本论文主要考虑超弦振幅,但了解玻色弦振幅的数学形式对理解超弦振幅是必不可少的。可惜由于超弦圈级振幅计算极其复杂,所以本论文并未讨论其一般形式。文献\cite{Witten:2012bh,DHoker:2002hof}中有较为细致的考虑。本章还利用RNS超弦进行了一些具体计算,给出了球面和盘面弦振幅之间的Kawai-Lewellen-Tye关系以及盘面振幅满足的单值关系。

第\ref{chap:5}章是本文主要工具纯旋量超弦的介绍,从Brink-Schwarz超对称粒子出发指出GW形式不可协变量子化的问题,然后启发性(自上而下)地构造出了纯旋量超弦。从历史的角度(自下而上)出发构造纯旋量超弦可见文献\cite{Berkovits:2002zk,Mafra:2008gkx}。

第\ref{chap:6}章则是本论文的主要结论,利用纯旋量超弦导出了任意点无质量态开弦盘面振幅的一般形式,并且讨论了其场论极限,十维超对称Yang-Mills理论振幅。由于纯旋量超弦的计算中会自然涌现出自由李代数结构,这一结构非常容易帮助讨论规范理论的色-运动学对偶,所以在本章最后讨论了如何构造满足色-运动学对偶的Bern-Carrasco-Johanson分子\cite{Mafra:2011kj}。

最后,本论文还给出了两个方便于实际计算的附录,附录\ref{appendix:A}给出了本论文主要使用到的算符乘积展开,在正文中将不再重复提及;纯旋量超弦计算涉及到大量有关$\gamma$矩阵的恒等式,这被囊括在附录\ref{appendix:B}中。另外,本论文还有一些标有$*$号的章节,这表示其脱离于本论文主线但是在本论文正文中有所提及,为了完整性而包括在内,略过并不影响对本文主要结论的理解。

本论文同时也是作者对过去半年多时间学习弦理论的总结,若存在疏漏之处,恳请学界同仁不吝指正!
	\chapter{玻色弦及其量子化}
\label{chap:2}
本章简要回顾玻色弦的量子化,虽然玻色弦有诸如真空不稳定以及没有费米子激发等问题,但对玻色弦的研究有助于理解超弦的相关问题。这里只做简要的回顾,更多相关细节读者可以参考\cite{Polchinski:1998rq,Blumenhagen:2013fgp,Becker:2006dvp},另外我们将使用更现代的共形场论的语言,相关细节可以在\cite{DiFrancesco:1997nk,Blumenhagen:2009zz}中找到。

\section{正则量子化}
弦论量子化其实是约束体系量子化问题,利用正则量子化并不能很好地解决,但是正则量子化的好处是能看出弦论的粒子谱。
\subsection{Nambu-Goto 作用量}
自由点粒子的作用量正比于其世界线场,受此启发可以立刻写下弦的作用量:
\begin{equation}
	S_{\text{NG}}=-\frac{1}{2\pi\alpha^\prime}\int_M d\tau d\sigma \left(\det h_{ab}\right)^{1/2},\quad h_{ab}:=\partial_a X^\mu \partial_b X_\mu
\end{equation}
由于作用量中包含根号,更利于量子化的方式是引入辅助场$\gamma_{ab}$
\begin{equation}
	\label{eq:2.2}
	S_\mathrm{P}[X,\gamma]=-\frac{1}{4\pi\alpha^{\prime}}\int_Md\tau d\sigma\left(-\gamma\right)^{1/2}\gamma^{ab}\partial_aX^\mu\partial_bX_\mu
\end{equation}
上述作用量有全局庞加莱对称性:
\begin{equation}
	\begin{aligned}&X^{\prime\mu}(\tau,\sigma)=\Lambda_{\nu}^{\mu}X^{\nu}(\tau,\sigma)+a^{\mu},\\&\gamma_{ab}^{\prime}(\tau,\sigma)=\gamma_{ab}(\tau,\sigma).\end{aligned}
\end{equation}
以及局域规范对称性$\mathrm{diff}\times\mathrm{Weyl}$:
\begin{equation}
	\begin{aligned}
		X^{\prime}{}^{\mu}(\tau^{\prime},\sigma^{\prime})&=X^{\mu}(\tau,\sigma),\\ \frac{\partial\sigma^{\prime c}}{\partial\sigma^a}\frac{\partial\sigma^{\prime a}}{\partial\sigma^b}\gamma_{cd}^{\prime}(\tau^{\prime},\sigma^{\prime})&=\gamma_{ab}(\tau,\sigma),
	\end{aligned}
\end{equation}
	
\begin{equation}
	\begin{aligned}
		X^{\prime\mu}(\tau,\sigma)&=X^\mu(\tau,\sigma),\\\gamma_{ab}^{\prime}(\tau,\sigma)&=\exp(2\omega(\tau,\sigma))\gamma_{ab}(\tau,\sigma),
	\end{aligned}
\end{equation}
由于$\gamma_{ab}$没有动力学,其运动方程将在量子化时作为约束引入:
\begin{equation}
	\label{eq:2.6}
	\frac{\delta S_\mathrm{P}}{\delta \gamma_{ab}}\sim T^{ab}=0
\end{equation}
同时不难验证弦经典运动方程为一维波动方程:
\begin{equation}
		\frac{\delta S_\mathrm{P}}{\delta X^\mu}\sim\left(\frac{\partial^2}{\partial\sigma^2}-\frac{\partial^2}{\partial\tau^2}\right)X^\mu=0
\end{equation}
由于一维弦的非平凡拓扑,若加入周期性边界条件则为闭弦:
\begin{equation}
	\label{eq:2.8}
	X^\mu(\tau,\sigma+\ell) = X^\mu(\tau,\sigma)
\end{equation}
而开弦端点上可以引入两种不同的边界条件:
\begin{equation}
	\label{eq:2.9}
	\begin{aligned}
	&	\left.n^a\partial_aX_\mu\right|_{\partial M}=0\quad \text{(Neumann)}\\
	&	X^\mu(\tau,0)=X^\mu(\tau,\ell)=x_0\quad \text{(Dirichlet)}
	\end{aligned}
\end{equation}
表面上看似乎只有第一种边界条件才不会破坏庞加莱对称性,Dirichlet边界条件其实相当于要求开弦端点依附于D膜上。而超弦中D膜其实可以作为BPS态稳定存在,所以D膜作为靶空间的非平凡缺陷也应当看作弦论自由度的一部分,这意味着Dirichlet边界条件也是可行的。

弦论中可以通过Chan-Paton因子引入$U(N)$规范对称性\footnote{非定向弦对应$SO(N)$和$Sp(N)$对称性},具体体现在开弦端点带上$U(1)\times \bar U(1)$荷,其可以解释为开弦端点依附的D膜指标:
\begin{equation}
	|N;k;a\rangle=\sum_{i,j=1}^n|N;k;ij\rangle\lambda_{ij}^a,\quad \lambda\in \mathfrak{u}(N)
\end{equation}
\subsection{光锥量子化}
正则量子化的核心是将力学量量子化为算符,泊松括号替换为狄拉克括号。但是量子化还要满足约束\ref{eq:2.6}。光锥量子化思路是取光锥规范定下$\mathrm{diff}\times\mathrm{Weyl}$规范对称性,类似在库伦规范下量子化$U(1)$Yang-Mills理论得到量子电动力学:
\begin{equation}
	\begin{gathered}
		X^\pm=2^{-1/2}(X^0\pm X^1),\quad X^i,i=2,\ldots,D-1\\
		X^+=\tau,\quad\partial_\sigma\gamma_{\sigma\sigma}=0,\quad\det\gamma_{ab}=-1
	\end{gathered}
\end{equation}
在这一规范选取下,$X^{+}$不再拥有动力学演化,而$X^{-}$可以完全由横向$\alpha^i$模展开,所以$\alpha^-$也不用考虑。以Neumann边界条件开弦为例,仅剩下非平凡的$X^i$模展开:
\begin{equation}
	\label{eq:2.12}
	X^i(\tau,\sigma)=x^i+\frac{p^i}{p^+}\tau+i(2\alpha^{\prime})^{1/2}\sum_{n=-\infty}^\infty\frac{1}{n}\alpha_n^i\exp\left(-\frac{\pi in\tau}{\ell}\right)\cos\frac{\pi n\sigma}{\ell}
\end{equation}
这里$x$可以理解为弦的质心动量,$p,\Pi$是相应的共轭动量:
\begin{equation}
	\begin{gathered}
		x^-(\tau)=\frac{1}{\ell}\int_0^\ell d\sigma X^-(\tau,\sigma)\\
		p_-=-p^+=\frac{\partial L}{\partial(\partial_\tau x^-)}=-\frac{\ell}{2\pi\alpha^{\prime}}\gamma_{\sigma\sigma}\\
		\Pi^i=\frac{\partial L}{\partial(\partial_\tau X^i)}=\frac{1}{2\pi\alpha^{\prime}}\gamma_{\sigma\sigma}\partial_\tau X^i=\frac{p^+}{\ell}\partial_\tau X^i
		x^i(\tau)=\frac{1}{\ell}\int_0^\ell d\sigma X^i(\tau,\sigma),\\p^i(\tau)=\int_0^\ell d\sigma\Pi^i(\tau,\sigma)=\frac{p^+}{\ell}\int_0^\ell d\sigma\partial_\tau X^i(\tau,\sigma)
	\end{gathered}
\end{equation}
然后取等时对易子进行标准正则量子化操作:
\begin{equation}
	\begin{aligned}
		[x^-,p^+]&=i\eta^{-+}=-i ,\\
		[X^i(\sigma),\Pi^j(\sigma^{\prime})]&=i\delta^{ij}\delta(\sigma-\sigma^{\prime}) 
	\end{aligned}
	\quad\Rightarrow\quad
	\begin{aligned}
		[x^i,p^j]&=i\delta^{ij} ,\\
		[\alpha_m^i,\alpha_n^j]&=m\delta^{ij}\delta_{m,-n}
	\end{aligned}
\end{equation}
不难看出上面模展开具有谐振子代数,所以弦的粒子谱可以看作不同振动模式激发:
\begin{equation}
	|N;k\rangle=\left[\prod_{i=2}^{D-1}\prod_{n=1}^\infty\frac{(\alpha_{-n}^i)^{N_{in}}}{(n^{N_{in}}N_{in}!)^{1/2}}\right]|0;k\rangle
\end{equation}
其中$\ket{0,k}$是真空态,且由于$p^i$为好量子数而带有背景动量。利用黎曼Zeta正规化$\zeta(-1)=-\frac{1}{12}$可以推出点粒子激发质量谱为:
\begin{equation}
	\label{eq:2.16}
	m^2_\mathrm{op}=\frac{1}{\alpha^{\prime}}\left(N+\frac{2-D}{24}\right),\quad N:=\sum_{i=2}^{D-1}\sum_{n=1}^{\infty}nN_{in}
\end{equation}
但是光锥量子化方法明显地破坏了庞加莱对称性,为了理论自洽,必须要求玻色弦定义在$26$维靶空间:
\begin{equation}
	\boxed{
		D_{\text{boson}}=26
	}
\end{equation}
这其实是第一激发态处于$SO(D-2)$小群表示的必然结果。注意到弦真空是质量平方负定的快子态:
\begin{equation}
	|0;k\rangle\Leftrightarrow m^2=-\frac{1}{\alpha^\prime}
\end{equation}
闭弦也可以同样处理,只是分为左右模,而开弦左右模叠加成为驻波\ref{eq:2.12},闭弦有如下模展开:
\begin{equation}
	\label{eq:2.19}
	\begin{aligned}X^i(\tau,\sigma)&=x^i+\frac{p^i}{p^+}\tau+i\left(\frac{\alpha^{\prime}}{2}\right)^{1/2}\\&\times\sum_{n=-\infty}^\infty\left\{\frac{\alpha_n^i}{n}\exp\left[-\frac{2\pi in(\sigma+\tau)}{\ell}\right]+\frac{\tilde{\alpha}_n^i}{n}\exp\left[\frac{2\pi in(\sigma-\tau)}{\ell}\right]\right\}\end{aligned}
\end{equation}
可以看到相当于两个开弦谱的叠合。同样正则量子化得到弦激发粒子谱:
\begin{equation}
	|N,\tilde{N};k\rangle=\left[\prod_{i=2}^{D-1}\prod_{n=1}^\infty\frac{(\alpha_{-n}^i)^{N_{in}}(\tilde{\alpha}_{-n}^i)^{\tilde{N}_{in}}}{(n^{N_{in}}N_{in}!n^{\tilde{N}_{in}}\tilde{N}_{in}!)^{1/2}}\right]|0,0;k\rangle
\end{equation}
闭弦周期性边界条件要求左右模满足$N=\tilde{N}$\footnote{注意这一条件并非总是满足,对于T紧致化后的靶空间,左右模间可以相差缠绕数。}。同样可以得到闭弦质量谱:
\begin{equation}
	m^2_{\mathrm{cl}}=\frac{2}{\alpha^\prime}(N+\tilde{N}-2)=\frac{4}{\alpha^\prime}(N-1)
\end{equation}
本论文主要考察无质量激发态,也就是开弦闭弦第一激发态,事实上更高激发态在$\alpha^\prime\to0$的场论极限下会被压低。从上面的弦粒子谱不难看出开弦第一激发态对应无质量规范玻色子,而闭弦第一激发态对应引力子,伸缩子以及$B$-场。
\subsection{协变量子化}
光锥量子化相当于先取规范使用约束条件后量子化,协变量子化相当于先量子化后使用约束条件,好处是明显地保留了庞加莱对称性。下面仍以开弦为例,好处是左右模重合。

要求$T_{ab}=0$从物理上看应当是要求其作用于量子化后的希尔伯特空间上是零算子:
\begin{equation}
	\label{eq:2.22}
	T_{ab}=0\xrightarrow{\text{量子化}} T_{ab}\ket{\psi}=0\Rightarrow L_n^m\ket{\psi}\sim0
\end{equation}
其中$L_n$是能动张量模展开后的Virasoro代数生成元:
\begin{equation}
	L_m=\frac{1}{2}\sum_{n=-\infty}^{+\infty}\alpha_{m-n}\cdot\alpha_n
\end{equation}
上述条件相当于要求物理态为Virasoro最高权态\footnote{$A$是待定常数,不少文献称为正规排序常数,计算上来源于NOP序和产生湮灭算符排序之间相差的常数。本质上来源于真空能贡献。}:
\begin{equation}
	\label{eq:2.23}
	(L_n^\mathrm{~m}+A\delta_{n,0})|\psi\rangle=0\quad\mathrm{for~}n\geq0
\end{equation}
由于不用选取光锥规范,\ref{eq:2.12}\ref{eq:2.19}中模展开应当将指标$i$替换为一般的$\mu$,现在态空间由$\alpha^\mu$生成。\ref{eq:2.23}的限制并不够,因为$\mathscr{H}_{\mathrm{phys}}$中还包含一些零模态,与物理态均正交,张成的希尔伯特空间记为$\mathscr{H}_{\mathrm{null}}$,所以和物理过程是解耦的,应当看作是规范变换的生成元,所以应当有下面的等价关系:
\begin{equation}
	\ket\psi \sim \ket\psi +\ket\chi,\quad \forall\psi\in\mathscr{H}_{\mathrm{phys}},\chi\in\mathscr{H}_{\mathrm{null}}
\end{equation}
最终我们得到协变微扰论的态空间:\footnote{这里似乎很像BRST量子化\ref{eq:2.48},但是由于我们并没有直接考虑鬼场,所以无法写下一个分次链复形}
\begin{equation}
	\label{eq:2.26}
	\mathscr{H}_{\mathrm{CQ}}\simeq\frac{\mathscr{H}_{\mathrm{phys}}}{\mathscr{H}_{\mathrm{null}}}
\end{equation}
计算此商空间便得到弦的激发态,而且自洽性自然要求$A=-1,D=26$,开弦快子态和无质量激发态构造如下:
\begin{equation}
	\begin{gathered}
		|0;k\rangle,\quad m^2=-\frac{1}{\alpha^{\prime}};\\e_{\mu}\alpha_{-1}^\mu|0;k\rangle,\quad m^2=0,k^\mu e_{\mu}=0,\\e_{\mu}\cong e_{\mu}+c k_\mu.
	\end{gathered}
\end{equation}
同时叠合关系给出闭弦激发态为:
\begin{equation}
	\begin{gathered}
		|0;k\rangle,\quad m^2=-\frac{4}{\alpha^{\prime}};\\e_{\mu\nu}\alpha_{-1}^\mu\tilde{\alpha}_{-1}^\nu|0;k\rangle,\quad m^2=0,k^\mu e_{\mu\nu}=k^\nu e_{\mu\nu}=0,\\e_{\mu\nu}\cong e_{\mu\nu}+a_\mu k_\nu+k_\mu b_\nu,a\cdot k=b\cdot k=0.
	\end{gathered}
\end{equation}
好处是态的构造明显保留庞加莱对称性,从而为计算Lorentz不变的弦振幅带来了方便。
\section{路径积分量子化}
不同于量子场论,弦论的相互作用并不需要在作用量中引入新的项来实现,而是直接由不同的世界面拓扑来实现。
\subsection{Polyakov路径积分}
为了使路径积分良定,先利用Wick转动将世界面度规从$(-,+)$的二维闵氏度规$\gamma_{ab}$转为$(+,+)$的二维欧式度规$g_{ab}$。单纯从$\mathrm{diff}\times\mathrm{Weyl}$不变性出发可以在\ref{eq:2.2}基础上额外引入一项正比于:
\begin{equation}
	\chi=\frac{1}{4\pi}\int_Md^2\sigma\mathrm{~g}^{1/2}R+\frac{1}{2\pi}\int_{\partial M}dsk
\end{equation}
其中$k$是测地曲率,利用Gauss-Bonnet定理不难看出这一项正是欧拉示性数。由于二维几何均共形平坦,所以可以选取等温坐标使得:
\begin{equation}
	\label{eq:2.30}
	g = \mathrm{e}^{2\omega} dz d\bar z
\end{equation}
再利用Weyl不变性可以将其变为平坦欧氏度规,也即选取共形规范。在这一规范下\footnote{用$\hat g$表示对应的规范固定选取。},利用Faddeev–Popov方法引入费米统计的$b,c$鬼场消去$\mathrm{diff}\times\mathrm{Weyl}$规范冗余,弦论配分函数可以写作:
\begin{equation}
	\label{eq:2.31}
	Z\left[\hat{g}\right]=\sum_{\substack{\text{worldsheet}\\\text{topologies}}}\frac{1}{V_{\mathrm{CKG}}}\int\mathcal{D}X\mathcal{D}[b\tilde{b}]\mathcal{D}[c\tilde{c}]\exp(-S_m-S_\mathrm{g})
\end{equation}
这里物质场和鬼场作用量分别为:
\begin{equation}
	\label{eq:2.32}
	\begin{aligned}
		S_m&=\frac{1}{2\pi\alpha^{\prime}}\int d^2z\mathrm{~}\partial X^\mu\bar{\partial}X_\mu,\quad
		S_g&=\frac{1}{2\pi}\int d^2z\left(b\bar\partial c+\tilde b\partial \tilde c\right)
	\end{aligned}
\end{equation}
注意$\mathrm{diff}\times\mathrm{Weyl}$规范固定后其实还剩下共形规范自由度没有消去\footnote{对高亏格曲面其实还会有额外的零模鬼场插入,这里留待第\ref{chap:4}章讨论},所以这时$S_m$和$S_g$都对应二维共形场论。所以弦论问题就被转化为了二维共形场论问题。而且可以看到弦论中的展开不同于量子场论中按照顶点数目展开,弦论是按照拓扑展开,相互作用系数有如下关系:
\begin{equation}
	g_0^2\thicksim g_c\thicksim e^\lambda
\end{equation}

注意,对于开弦,由于端点的世界线会构成世界面的边界,所以开弦实际上需要使用边界共形场论进行研究。此时左右模均定义在上半复平面上,而且Neumann边界条件\footnote{鬼场也会有类似的限定}体现在实轴上左右模的对应:
\begin{equation}
	T(z)=\tilde{T}(\bar{z}),\quad c(z)=\tilde{c}(\bar{z}),\quad b(z)=\tilde{b}(\bar{z}),\quad\operatorname{Im}z=0
\end{equation}
这时只能在上复平面讨论共形场论,不过可以用加倍技巧将上复平面右模的信息转移到下复平面左模场的定义:
\begin{equation}
	\label{eq:2.35}
	T(z):= \tilde T({z}^{\prime}),b(z):= \tilde b({z}^{\prime}),c(z):= \tilde c({z}^{\prime})\quad z^\prime = \bar z,\mathrm{Im}z<0
\end{equation}
对左模场延拓\footnote{这种延拓技巧在数学上依赖于Schwarz反射原理}之后左模场已然包含右模场的所有信息,所以可以转换为在整个复平面上仅仅讨论左模场的共形场论。下面的简单例子便能说明加倍技巧:
\begin{equation}
	\begin{aligned}
		L_m&=\frac{1}{2\pi i}\int_C\left(dzz^{m+1}T(z)-d\bar{z}\bar{z}^{m+1}\tilde{T}(\bar z)\right)
		\\&=\frac{1}{2\pi i}\oint dzz^{m+1}T(z)
	\end{aligned}
\end{equation}
开弦左右模重叠,围道积分由于边界的限制只能在上复平面进行。利用加倍技巧将右模的信息转换为左模在下复平面上的信息,从而将边界共形场论转换为一般的共形场论,而且只剩下左模场参与计算。所以闭弦和开弦的计算是一样的,只需要注意开弦只保留左模,或者简单认为左右模场此时相等\footnote{但是注意严谨来讲边界条件只给出实轴上相等,是利用加倍技巧之后才在复平面上等同,而且这种分析技巧只适用于复平面,对于本文主要讨论的树级振幅已足够。}。

另外,根据二维闭曲面分类定理,对世界面拓扑求和也可以是不可定向曲面,对不可定向曲面求和意味着我们将世界面宇称$\sigma^{\prime}=\ell-\sigma^2$作为规范对称性引入。这时的弦论成为不可定向弦。考虑世界面宇称变换由$\Omega$算符生成,对于开弦有:
\begin{equation}
	\Omega\alpha_n^\mu\Omega^{-1}=(-1)^n\alpha_n^\mu
\end{equation}
而对于闭弦则会左右模互换:
\begin{equation}
	\Omega\alpha_n^\mu\Omega^{-1}=\tilde\alpha_n^\mu,\quad \Omega\tilde\alpha_n^\mu\Omega^{-1}=\alpha_n^\mu
\end{equation}
且假设真空态$\Omega=+1$,在这一约定下,自洽的弦相互作用要求不可定向弦只剩下$\Omega=+1$的激发态。本论文主要考虑可定向弦的树级振幅计算。
\subsection{弦振幅}
类似量子场论中振幅依赖于渐进态的定义,弦论中我们提到振幅都是指无穷远处的弦激发态演化后的相互作用关联函数。取等温坐标后,利用共形场论的态算符对应,无穷远处的激发态相当于世界面某个点上插入对应的顶角算符$\mathscr{V}$,用图示来看这个过程相当于图\ref{fig:1}。
\begin{figure}[htbp]
	\centering
	\includegraphics{figs/fig1.pdf}
	\caption{态算符对应与弦振幅}
	\label{fig:1}
\end{figure}
受前面配分函数的启发,由此可以写下弦振幅\footnote{也称作弦的S矩阵。}:
\begin{equation}
	\label{eq:2.39}
	\begin{aligned}
		S_{j_1...j_n}(k_1,\ldots,k_n)&=\llangle[\Big]\prod_{i=1}^n\int d^2\sigma_ig(\sigma_i)^{1/2}\mathscr{V}_{j_i}(k_i,\sigma_i)\rrangle[\Big]
		\\&=\sum_{\substack{\text{worldsheet}\\\text{topologies}}}\mathrm{e}^{-\lambda \chi}\int\frac{\mathcal{D}X\mathcal{D}g}{V_{\mathrm{diff}\times\mathrm{Weyl}}}\mathrm{e}^{-S_m}\prod_{i=1}^n\int d^2\sigma_ig(\sigma_i)^{1/2}\mathscr{V}_{j_i}(k_i,\sigma_i)
	\end{aligned}
\end{equation}
上式中我们使用$\llangle\bullet\rrangle$是为了提醒读者规范固定时可能的零模鬼场插入。而且上式我们并没有利用鬼场进行规范固定,这个问题我们留待到第\ref{chap:4}章解决。
\subsection{顶角算符}
利用前面正则量子化得到的激发态,只需要做如下替换便可以得到对应的顶角算符:
\begin{equation}
	\alpha_{-m}^\mu\to i\left(\frac{2}{\alpha^{\prime}}\right)^{1/2}\frac{1}{(m-1)!}\partial^mX^\mu(0),\quad
	|0;k\rangle\rightarrow e^{ik\cdot X(0,0)}
\end{equation}
由于算符乘积展开(OPE)在插入点相同时奇异,所以替换完成后还需要对算符取正规排序乘积(NOP)保证非奇异。这其实相当于一种重整化的选取,本论文均采用共形场论的NOP技术来重整化。

一般我们把上述构造的算符与世界面上积分$\int d\sigma^2$\footnote{注意开弦由于插入点在边界上,所以积分在边界上进行。}共同称作积分顶角算符,记作$U$,积分的存在使得顶角算符整体共形权为$0$,从而最终的振幅具有共形不变性。
\section{BRST量子化}
本节的核心目标是使用BRST量子化方法给出后续振幅计算中规范固定需要引入的无积分顶角算符。

考虑关于物质场$\phi_i$的某个一般性量子理论,$i$以及后续讨论涉及到的指标可以连续或离散,连续部分标记场的坐标依赖,离散部分则标记场的种类。假设所考虑体系的规范对称性生成元$\delta_\alpha$具有类似李代数的结构:
\begin{equation}
	\label{eq:2.41}
	[\delta_\alpha,\delta_\beta]=f_{\alpha\beta}^\gamma\delta_\gamma
\end{equation}
利用FP方法引入鬼场进行规范固定,规范选取为:
\begin{equation}
	F^A(\phi)=0
\end{equation}
则配分函数计算为:
\begin{equation}
	\int\frac{[d\phi_i]}{V_{\mathrm{gauge}}}\exp(-S_1)\to\int[d\phi_idB_Adb_Adc^\alpha]\exp(-S_m-S_g-S_f)
\end{equation}
这里新引入了鬼场$S_g$和规范固定项$S_f$:\footnote{前面\ref{eq:2.31}没有$S_f$是因为规范选取完全消除了$g$的规范冗余,不难看到$B_A$积分后的效果是引入$\delta(F^A(\phi))$,所以对$g$再次进行路径积分便完全移除了这一项。}
\begin{equation}
	S_g=b_Ac^\alpha\delta_\alpha F^A(\phi),\quad S_f=-iB_AF^A(\phi)
\end{equation}
以上无非是FP鬼场的一般做法,重点在于上述体系存在一个全局对称性:
\begin{equation}
	\begin{aligned}
		\delta_{\mathbf{B}}\phi_i 
		&= -i\epsilon c^\alpha\delta_\alpha\phi_i,  
		& \quad \delta_{\mathbf{B}}B_A 
		&= 0, \\
		\delta_{\mathbf{B}}b_{A} 
		&= \epsilon B_A,  
		& \quad \delta_{\mathbf{B}}c^{\alpha} 
		&= \frac{i}{2}\epsilon f^\alpha{}_{\beta\gamma} c^\beta c^\gamma.
	\end{aligned}
\end{equation}
对应的Noether荷记作$Q_B$,其中$\epsilon$是格拉斯曼变量。$c$的鬼数为$+1$,$b$和$\epsilon$鬼数为$-1$。所以$Q_B$具有$+1$的鬼数,而且具有幂零性:
\begin{equation}
	Q_{\mathrm{B}}^2=0
\end{equation}
考虑物质场和鬼场共同生成的希尔伯特空间,以鬼数为分次可以立即写下BRST上链复形$C^{\bullet}_{\mathrm{BRST}}$:
\begin{equation}
	\cdots\longrightarrow\mathscr{H}_{g-1}\xrightarrow{Q_{g-1}}\mathscr{H}_g\xrightarrow{Q_g}\mathscr{H}_{g+1}\longrightarrow\cdots
\end{equation}
BRST量子化则是要求物理态处于鬼数为0的上同调群中\footnote{对于无积分顶角算符则是要求在鬼数为1的上同调群中,而且额外要求是共形不变的,也就是限制在共形权为0的子复形中讨论上同调。},这其实是要求可观测量具有BRST不变性:\footnote{准确来说$\ket{\text{phys}}\in Z$,闭链要求给出BRST不变性,而模掉边缘链$B$(物理上也称$B$中的态为BRST恰当)意味着$B$中的态都是和物理态脱耦的,类似\ref{eq:2.26}}
\begin{equation}
	\label{eq:2.48}
	\mathscr{H}_{\mathrm{BRST}}\cong H^0(C^\bullet_{\mathrm{BRST}}):=Z(C^\bullet_{\mathrm{BRST}})/B(C^\bullet_{\mathrm{BRST}})
\end{equation}

接下来我们将上面一般性的方法应用到弦论中,BRST荷有如下形式:
\begin{equation}
	\label{eq:2.49}
	Q_B=\frac{1}{2\pi i}\oint(\operatorname{d}zj_B-\operatorname{d}\overline{z}\tilde{j}_B),\quad
	j_{\mathrm{B}}:=cT^m+:bc\partial c:+\frac{3}{2}\partial^2c
\end{equation}
$Q_B^2=0$要求$\{Q_B,Q_B\}=0$,这对应$j_B$OPE的一阶奇点:\footnote{这些复杂的OPE计算可以使用笔者编写的程序完成:\url{https://github.com/WHUZBF/MMA/tree/main/OPE}}
\begin{equation}
	j_B(z)j_B(w)\sim-\frac{c^m-18}{2(z-w)^3}c\partial c(w)-\frac{c^m-18}{4(z-w)^2}c\partial^2c(w)-\frac{c^m-26}{12(z-w)}c\partial^3c(w)
\end{equation}
由此可以看出自洽量子化要求$c^m=26$,也即靶空间维数为$26$。另外,物理态在壳要求:
\begin{equation}
	\label{eq:2.51}
	L_0|\psi\rangle=\{Q_B,b_0\}|\psi\rangle=0\Rightarrow b_0\ket{\psi}=0
\end{equation}
为了构造无质量态无积分顶角算符,首先使用$bc$鬼场物质场$\partial^n X$以及平面波$\mathrm{e}^{ip\cdot X}$构造出最一般的共形权为$0$的顶角算符:\footnote{注意我们这里忽视了鬼数为1这个条件,后面会在求BRST恰当项中引入。也可以先引入这个条件使得计算更加简便,但我们这里考虑最一般的构造帮助读者熟悉OPE计算。}
\begin{equation}
	V_{\text{general}}=:\left(\alpha\partial c+\beta c\partial^2c+\gamma c\partial c\partial^2c+\epsilon_\mu c\partial X^\mu+\zeta_\mu c\partial c\partial X^\mu+\lambda\right)e^{ip\cdot X}:
\end{equation}
上述对态的要求可以转化为对算符的要求:\footnote{这里我们利用了$Q$的定义含围道积分,对易子的计算可以转换为被积算符OPE的计算}
\begin{equation}
	Q_B\ket{\psi}=0\Leftrightarrow [Q_B,V] = 0 \Leftrightarrow Q_B V\sim 0
\end{equation}
首先作用在壳条件得到:
\begin{equation}
	\begin{aligned}
		&\begin{aligned}
		b_0V_{\text{general}}&=\oint\frac{\mathrm{d}z}{2\pi i}zb:\left(\alpha\partial c+\beta c\partial^2c+\gamma c\partial c\partial^2c+\epsilon_\mu c\partial X^\mu+\zeta_\mu c\partial c\partial X^\mu+\lambda\right)e^{ip\cdot X}:\\
		&=\oint\frac{\mathrm{d}z}{2\pi i}:\left(\frac{\alpha}{z}-\frac{\gamma c\partial^2c}{z}-\frac{\zeta_\mu(c\partial X^\mu)}{z}\right)e^{ip\cdot X}:=0\\
	\end{aligned}\\
	&\Rightarrow\alpha,\gamma,\zeta^\mu = 0
	\end{aligned}
\end{equation}
剩下的可能的顶角算符形式为:
\begin{equation}
	V_{\text{general}}\to V_1=:\left(\beta c\partial^2c+\epsilon_\mu c\partial X^\mu+\lambda\right)e^{ip\cdot X}:
\end{equation}
继续作用式\ref{eq:2.49},注意这里考虑开弦,只有左模:
\begin{equation}
	\begin{aligned}
		&Q_BV_1=\oint\frac{\mathrm{d}z}{2\pi i}\left[-\frac{i\alpha^{\prime}}{4z}\epsilon\cdotp:\partial^2ce^{ip\cdot X}c:+\frac{\lambda}{z}c\partial e^{ip\cdot X}\right]=0\\
	&\Rightarrow\lambda=0,\quad\epsilon\cdot p=0
	\end{aligned}
\end{equation}
注意到BRST算符并不改变共形权,所以BRST恰当部分应当也由$V_{\text{general}}$中的项生成,另外注意到鬼数为1的恰当项应当由鬼数为0的部分生成,而鬼数为零的算符只有平面波,所以我们立刻写下:
\begin{equation}
	Q_B\lambda :\mathrm{e}^{ip\cdot X}: = i\lambda:cp\cdot\partial Xe^{ip\cdot X}:
\end{equation}
这给出限制$\epsilon\cong\epsilon+p$,另外回到我们所关注的鬼数$1$上同调群,$c\partial^2 c$鬼数为$2$可以扔掉\footnote{凑巧它其实也是BRST恰当的},最终得到顶角算符:
\begin{equation}
	V_{\text{phys}}=:\epsilon_\mu c\partial X^\mu e^{ip\cdot X}:,\quad \epsilon\cdot p = 0,\epsilon^\mu\cong\epsilon^\mu+p^\mu
\end{equation}

上述计算推广到一般的态是显然的,开弦态只使用了左模场,闭弦态只需要额外使用右模场构造$V$即可。另外,虽然看似无积分顶角算符关联函数明显依赖于世界面坐标,后面会发现这一点被$V_{\mathrm{CKG}}$消除。

不难看出无积分顶角算符和积分顶角算符的联系为$V=cU$\footnote{如果是闭弦则是$V=c\tilde{c}U$},这是$bc$鬼场的性质,即便是RNS超弦这一点也成立,更重要的是这一联系也可以用BRST荷表述为:
\begin{equation}
	\label{eq:UV}
	[Q_B, U]\sim Q_B U= \partial V
\end{equation}
这在后面构造纯旋量超弦的积分顶角算符中非常重要。而且上式立刻说明了$\int\mathrm{d}z U(z)$ BRST闭,所以这也能看出$cU\leftrightarrow\int\mathrm{d}z U$。
\section{鬼场的真空}
现在我们来简要讨论上述结果的物理意义,这对后面构造RNS超弦顶角算符具有很大的意义。首先我们需要区分一下共形场论中的$SL(2,\mathbb{C})$不变真空$\ket{1}$。以及物理上关注的真空$\ket{0}$。对于共形权为$h$的主场,其$SL(2,\mathbb{C})$不变真空定义为:
\begin{equation}
	\phi_{n>-h}\ket{1}=0
\end{equation}
而物理上真空态要求能量最低,$H\sim L_0$而且注意到$[L_n,\phi_m]=[(h-1)n-m]\phi_m$,所以一切$\phi_{m>0}$都会降低能量(共形权),所以:
\begin{equation}
	\phi_{n>0}\ket{0}=0
\end{equation}
不难看出两者一般而言是不同的,这一点对于鬼场这种中心荷为负数的奇异系统尤为显著。由于$c$鬼场共形权为$-1$,所以我们能够构造能量低于$SL(2,\mathbb{C})$不变真空的真空态:
\begin{equation}
	\label{eq:2.62}
	\left|c\right\rangle=c_{1}\left|1\right\rangle,\quad\left|(\partial c)c\right\rangle=c_{0}c_{1}\left|1\right\rangle
\end{equation}
而\ref{eq:2.51}中$b_0$的要求相当于选取$\ket{c}$而不是$\ket{(\partial c) c}$作为微扰展开的鬼场真空。这样我们就能将$V$中出现的$c$自然解释为鬼场真空的贡献。而且$c$的出现是必然的,$bc$鬼场$U(1)$对称性给出下面的Noether流:
\begin{equation}
	\label{eq:2.63}
	j_{b,c}(z)=-:b(z)c(z):
\end{equation}
由此可以用下面的OPE定义算符$\mathcal{O}_q$的鬼数$q$:
\begin{equation}
	j_{b,c}(z)\mathcal{O}_q(w)\sim\frac{q\mathcal{O}_q(w)}{z-w}\Leftrightarrow j_0\ket{q} = q\ket{q}
\end{equation}
而$j_g$其实不是主场,其存在共形反常,可以由下面OPE看出:
\begin{equation}
	T_{b,c}(z)j_{b,c}(w)\sim\frac{-3}{(z-w)^3}+\frac{j_{b,c}(w)}{(z-w)^2}+\frac{\partial j_{b,c}(w)}{z-w}
\end{equation}
所以$bc$鬼系统具有背景鬼数$Q_{b,c}=-3$,这给出厄米共轭修正$j_n^\dagger=(-1)^{\delta_{n,0}}j_{-n}-Q_{b,c}\delta_{n,0}$。这要求只有当关联函数的鬼数(考虑真空背景后)为$0$时关联函数才不为$0$。所以关联函数中出现带$c$的无积分顶角算符是必然的。后面会看到对真空背景鬼数的补偿相当于路径积分中插入一些鬼场零模,Riemann-Roch定理指出鬼场零模插入有如下关系:
\begin{equation}
	\label{eq:2.66}
	N_c-N_b=3-3g
\end{equation}
对于$g=0$的球面情况,注意到$b$鬼数为$-1$,$c$鬼数为$+1$,这正好补偿了$Q_{b,c}=-3$。更高亏格的情况会在第\ref{chap:4}章详细说明。

而前面我们提到过积分顶角算符$U$和无积分顶角算符$V$在振幅计算中都非常重要,具体来说$\int d\sigma U$和$cV$对应同一个态的不同版本的顶角算符。为了补偿鬼场真空背景荷,在球面上就必须选取三个积分顶角算符替换为积分顶角算符,从而才能得到非零的振幅。后面第\ref{chap:4}章会从路径积分的角度直接看出这一点。
\section{*BV形式}
BRST量子化方法适用的前提是规范对称性满足\ref{eq:2.41}的结构,但如果这一结构不满足,比如结构常数不再是常数,而与场本身有关,而且不再是李代数闭的。BRST量子化就需要被扩充为更一般的Batalin–Vilkovisky量子化。更多细节可以在\cite{Weinberg:1996kr,Erbin:2021smf,Henneaux:1994lbw}中找到。

现在考虑下面更一般的规范对称性满足的代数:
\begin{equation}
	\label{eq:2.59}
	[\delta_a,\delta_b]=F_{ab}^c(\phi)\delta_c+\lambda_{ab}^i\frac{\delta S_m}{\delta \phi^i}
\end{equation}
后面一项意味着在壳情况下规范对称性还是闭的,这是物理上的要求,要求在壳时对称性构成一个群。BV形式的核心思想是把鬼场不单单看作是人为引入的消去规范的场,而是看作与物质场同等地位。因为对于更复杂的规范对称性的情况,可能$\delta_a$(第零级规范)之间不是独立的,也就是说\ref{eq:2.59}是一个可约代数\footnote{$p$-形式规范对称性就是个很好的例子,因为$d^2=0$,所以规范参数之间亦有规范不变性,而Yang-Mills理论是$1$-形式的理论,所以用BRST方法就能很好地处理。}。也就是说即便是规范对称性的参数之间仍有规范不变性(第一级规范),这意味着为了消去规范对称性引入鬼场,而为了消去规范对称性参数之间的对称性又要引入鬼场的鬼场,如此循环往复,直到消到第$\ell$级规范时\ref{eq:2.59}不可约。上述过程可以用图表\ref{tab:1}描述。
\begin{table}[htbp]
	\centering
	\begin{tabular}{ccc}
		\hline % 顶部粗线(可选)
		level $0$ & $\delta\phi^i=\epsilon_0^{a_0}R_{a_0}^i(\phi^i)$ & $c_0^{a_0}$ \\ 
		\hline % 中等粗细线
		level $1$ & $\delta c_0^{a_0}=\epsilon_1^{a_1}R_{a_1}^{a_0}(\phi^i,c^{a_0})$ & $c_1^{a_1}$ \\ 
		\hline
		$\cdots$ & $\cdots $ & $\cdots$\\
		\hline
		level $n+1$ & $\delta c_n^{a_n}=\epsilon_{n+1}^{a_{n+1}}R_{a_{n+1}}^{a_n}(\phi^i,c_0^{a_0},\ldots,c_n^{a_n})$ & $c_{n+1}^{a_{n+1}}$ \\ 
		\hline % 底部粗线(可选)
	\end{tabular}
	\caption{BV形式的鬼场}
	\label{tab:1}
\end{table}

现在把所有的鬼场和物质场看作同等地位:
\begin{equation}
	\psi^r=\{c_n^{a_n}\}_{n=-1,...,\ell},\quad c_{-1}:=\phi
\end{equation}
第$0$级物质场鬼数为$0$,且格拉斯曼偶宇称,每向上一级鬼数增加$1$,且格拉斯曼反宇称。同时引入对应的反场$\psi^*_r$\footnote{注意这里我们使用$\bullet^*$而不是$\tilde\bullet$,因为前面$\tilde b$虽然我们称为“反鬼场”,但其实只是鬼场的右模部分。}。和对应的“正”场之间格拉斯曼宇称相反,且鬼数相加为$-1$,反场由反括号诱导的对偶空间定义:
\begin{equation}
	(A,B)=\frac{\partial_RA}{\partial\psi^r}\frac{\partial_LB}{\partial\psi_r^*}-\frac{\partial_RA}{\partial\psi_r^*}\frac{\partial_LB}{\partial\psi^r},\quad \partial_R:=\overset{\rightarrow}{\partial},\partial_L:=\overset{\leftarrow}{\partial}
\end{equation}
量子化后的路径积分表示为:
\begin{equation}
	Z=\int\mathrm{d}\psi^r\mathrm{d}\psi_r^*\mathrm{e}^{-W[\psi^r,\psi_r^*]/\hbar}
\end{equation}
推广的BRST对称性由下式生成:
\begin{equation}
	\delta_\epsilon F=\epsilon\mathcal{Q}_B F=(W,F)-\hbar\Delta F,\quad \Delta:=\frac{\partial_R}{\partial\psi_r^*}\frac{\partial_L}{\partial\psi^r}
\end{equation}
为了自洽地量子化,必须要求$\delta_\epsilon W =0$,这相当于:
\begin{equation}
	(W,W)-2\hbar\Delta W=0
\end{equation}
上式常被称为量子主方程,此方程用于将$S_m$扩充为更一般的满足推广的BRST对称性的作用量$W$从而进行量子化。$\mathcal{Q}_B$依旧是幂零算符,计算其上同调便可得到可观测量,这一点与BRST形式是一致的。
	\chapter{Ramond-Neveu-Schwarz超弦}
本章使用超对称共形场论(SCFT)的方法介绍世界面超对称的RNS超弦理论。更多细节详见\cite{Green:2012oqa,Green:2012pqa}。
\section{世界面超场}
考虑世界面超对称,Polyakov作用量\ref{eq:2.2}中为$X$引入世界面自旋$\frac12$超伴场$\Psi$,为$\gamma$引入世界面自旋$\frac32$超伴场$\chi$,他们都是世界面上的二维旋量。在世界面Wick转动后,RNS超弦作用量为\footnote{$\rho$是二维gamma矩阵。}:
\begin{equation}
	\begin{aligned}
		S_{\mathrm{RNS}}[X,\Psi,g,\chi]=&\frac1{4\pi}\int\mathrm{d}^2\sigma\sqrt{-g}\Big[-\frac1{\alpha^{\prime}}g^{\alpha\beta}\partial_\alpha X^\mu\partial_\beta X_\mu+\overline{\Psi}^\mu\rho^\alpha\nabla_\alpha\Psi_\mu\\&+(\overline{\chi}_\alpha\rho^\beta\rho^\alpha\Psi^\mu)\Big(\frac1{\sqrt{2\alpha^{\prime}}}\partial_\beta X_\mu+\frac18\overline{\chi}_\beta\Psi_\mu\Big)\Big]
	\end{aligned}
\end{equation}
对于闭弦,上述作用量的超对称荷有左右模部分,所以实际上是$\mathcal{N}=2$超对称,而超弦只有左模有贡献,所以是$\mathcal{N}=1$超对称。后面将会看到他们低能极限下的谱分别是$\mathcal{N}=2$超引力以及$\mathcal{N}=1$超对称Yang-Mills理论。

现在$\mathrm{diff}\times\mathrm{Weyl}$不变性被提升为$\mathrm{Super\mbox{-}diff}\times\mathrm{Super\mbox{-}Weyl}$不变性,类似\ref{eq:2.30}取等温坐标到共形规范下消去$g$,这里我们可以取超共形规范消去$\chi$,并且将Majorana旋量\footnote{二维情况下总可以选取$\rho$的实表示从而要求$\Psi$为实的,这在二维情况下意味着是Majorana旋量}$\Psi$分解成左右手Weyl旋量$\psi/\bar\psi$。最终得到世界面上的超对称共形场论的物质项:
\begin{equation}
	S=\frac{1}{2\pi}\int\mathrm{d}^2z\left(\frac{2}{\alpha^{\prime}}\partial X^\mu\overline{\partial}X_\mu+\psi^\mu\overline{\partial}\psi_\mu+\overline{\psi}^\mu\partial\overline{\psi}_\mu\right)
\end{equation}
同时可以引入玻色鬼场$bc$以及费米鬼场$\beta\gamma$:
\begin{equation}
	S_{\mathrm{gh}}=\frac{1}{2\pi}\int\mathrm{d}^2z\left(b\overline{\partial}c+\overline{b}\partial\overline{c}+\beta\overline{\partial}\gamma+\overline{\beta}\partial\overline{\gamma}\right)
\end{equation}
共形变换以及相应的超共形变换的能动张量为:
\begin{equation}
	\label{eq:3.4}
	\begin{gathered}
		\frac{\delta S_m}{\delta g}\sim T^\mathrm{m}(z)=-\frac{1}{\alpha^{\prime}}:\partial X\cdot\partial X:-\frac{1}{2}:\psi\cdot\partial\psi:\\
		\frac{\delta S_m}{\delta\chi}\sim G^\mathrm{m}(z)=i\sqrt{\frac{2}{\alpha^{\prime}}}\psi^\mu\partial X_\mu
	\end{gathered}
\end{equation}
由于$g,\chi$都没有动力学,$T^m,G^m$应当为$0$类似\ref{eq:2.22}作为约束出现。另外$bc\beta\gamma$鬼场总共对中心荷贡献$-15$,而玻色场贡献$D$,费米场贡献$\frac{D}{2}$,所以共形反常消去必须要求:
\begin{equation}
	\boxed{D_{\mathrm{super}}=10}
\end{equation}

后面的讨论主要针对左模。类似开弦边界条件\ref{eq:2.9}的边界条件选取消去作用量泛函导数的边界项贡献,最终加倍技巧后体现为左右模相等\ref{eq:2.35}。而对RNS超弦,类似的条件会导致世界面超场左右模之间相差正负号,从而给出两种不同的超场模展开:
\begin{equation}
	\psi_{\mathrm{NS}}^\mu(z)=\sum_{r\in\mathbb{Z}+\frac{1}{2}}\psi_r^\mu z^{-r-\frac{1}{2}},\quad\psi_{\mathrm{R}}^\mu(z)=\sum_{n\in\mathbb{Z}}\psi_n^\mu z^{-n-\frac{1}{2}}
\end{equation}
世界面超场应当看作是在复平面的双覆盖黎曼面上定义。世界面超流$G$以及$\beta\gamma$鬼场同样可以分为NS和R两个部分,只是$z$的指数依赖要根据共形权重写。后面会看到$R$部分负责产生费米子,NS部分负责产生玻色子。对于闭弦,边界条件\ref{eq:2.8}变化为超场可以满足周期性或者反周期性。左右模部分可以分别处于NS,R部分,所以总共有四个部分。

最后来讨论一下费米物质场的真空。玻色场的真空由$\alpha^\mu_{n\geq 1}$湮灭的$\ket{0;p^\mu}$生成,其中$p^\mu\propto\alpha^\mu_0$用来标记真空动量$p^\mu$。类似的,费米物质场真空也由对应的湮灭算符产生:
\begin{equation}
	\psi_{r\geq\frac12}^\mu\left|0,p\right\rangle_{\mathrm{NS}}=0,\quad \psi_{n\geq1}^\mu\left|0,p\right\rangle_{\mathrm{R}}=0
\end{equation}
$\psi_0$类似$\alpha_0$既不是产生也不是湮灭算符,而是用于标记简并的真空态。不过$\psi_0$只存在于R部分真空,所以NS部分的真空依旧直接是$\ket{0}_{\mathrm{NS}}$,而且这也正是$\frac12$共形权的$\psi$场的$SL(2,\mathbb{C})$不变真空。注意到$\psi_0$之间满足:
\begin{equation}
	[\psi_0^\mu,\psi_0^\nu]=\eta^{\mu\nu}
	\xleftrightarrow{\Gamma\sim\sqrt{2}\psi_0} [\Gamma^\mu,\Gamma^\mu]=2\eta^{\mu\nu}
\end{equation}
也就是说R部分真空应当处于$SO(9,1)$的旋量表示也就是十维Clifford代数表示中,是具有$2^5=32$个分量的Dirac旋量$\left|A^{\prime}\right\rangle_{\mathrm{R}}=\left|A\right\rangle\oplus|\dot A\rangle$,现在考虑$G^m=0$的限制要求,类似\ref{eq:2.23},对R部分有:
\begin{equation}
	G^m_{n\geq0}\left|\mathrm{phys}\right\rangle_{\mathrm{R}}=\sum_{m\in\mathbb{Z}}\alpha_m\cdot\psi_{m-n}\left|\mathrm{phys}\right\rangle_{\mathrm{R}}=0
\end{equation}
这个时候由于\ref{eq:3.4}中$G^m$部分$\psi$与$\partial X$之间OPE正则,所以不需要引入类似\ref{eq:2.23}的正规排序常数\footnote{这其实也是因为鬼场和物质场对R真空的总真空能贡献为0。}。考虑$n=0$时类似$L_0=0$的质量在壳条件,物质场超流要求的在壳条件可以改写为下面的Dirac-Ramond方程:
\begin{equation}
	\left(p\cdot\Gamma+\frac{2\sqrt{2}}{\ell}\sum_{n=1}^\infty(\alpha_{-n}\cdot \psi_n+\psi_{-n}\cdot\alpha_n)\right)\left|\mathrm{phys}\right\rangle_{\mathrm{R}}=0
\end{equation}
上述方程对真空态退化为$p\cdot\Gamma\ket{0}=0$即Dirac方程。这一方程将每个Weyl分量从$16$缩减到$8$。后面GSO投影会对这一自由度再次进行修正。

NS真空是$SL(2,\mathbb{C})$不变真空,而R真空实际上可以看作是NS真空的激发态,自旋场将这两个真空联系起来:
\begin{equation}
	\left|A^{\prime}\right\rangle_{R}=\lim_{z\to 0}S_{{A^{\prime}}}(z)\left|0\right\rangle_{NS},\quad 
	{}_R\left\langle A^{\prime}\right|=\lim_{z\to\infty}{}_{NS}\left\langle0\right|S_{{A^{\prime}}}(z)z^{D/8}
\end{equation}
$z^{D/8}$的出现是BPZ共轭的要求\cite{itocft},$S_A$的共形权为$\frac{D}{16}=\frac58$,其来自于R部分物质场真空能贡献,相应的NS部分物质场真空能贡献为0:
\vspace{4em}% 这里需要调
\begin{equation}
	\label{eq:3.12}
	a^{\mathrm{m}}_R=
	\eqnmarkbox[blue]{node1}{\frac{1}{24}c^{\mathrm{m}}}+
	\left(\eqnmark[red]{node2}{-\frac{1}{24}}+\tikzmarknode{node3}{\frac{1}{24}}\right)D=\frac{1}{16}D
\end{equation}
\annotate[yshift=1em]{right}{node1}{能动张量非主场,共形变换贡献}
\annotate[yshift=-0.5em]{below,left}{node2}{玻色场贡献}
\annotate[yshift=-1em]{below,label below}{node3}{费米场贡献}
\vspace{1em}
\section{正则量子化}
本节使用光锥量子化处理RNS超弦,好处是规范完全固定,只用讨论物质场,能方便看出粒子谱的超对称性\footnote{协变量子化可见\href{https://www.uu.se/en/department/physics-and-astronomy/research/theoretical-physics/oliver-schlotterer}{Oliver Schlotter的在线讲义}。}。后面再使用BRST量子化来构造协变的顶角算符,后面的讨论以开弦为例。

玻色场的光锥规范依旧和第\ref{chap:2}章的讨论相同,费米场NS部分的光锥规范有如下的简单形式:
\begin{equation}
	\psi^+ = 0
\end{equation}
R部分同上式一样,唯一不同是保留零模,用于生成简并R真空。而$\psi^-$部分同样也可以用横向振动激发描述。所以$\psi^\pm$不再拥有动力学,我们只需要关注横向振动激发。粒子谱由$\psi^i_\bullet,\alpha^i_\bullet$作用在R和NS真空上得到。

NS部分的质量谱可以从$L_0^m$最高权限制给出的在壳条件推出:
\begin{equation}
	\label{eq:3.14}
	\alpha^{\prime}m^2_{NS}=\sum_{n=1}^\infty\alpha_{-n}^i\alpha_n^i+\sum_{r=1/2}^\infty r\psi_{-r}^i\psi_r^i-\frac{1}{2}
\end{equation}
这里$-\frac12$来源于\ref{eq:3.12}类似的计算,注意还要加上鬼场的贡献,同理R部分有:
\begin{equation}
	\alpha^{\prime}m^2_{R}=\sum_{n=1}^\infty\alpha_{-n}^i\alpha_n^i+\sum_{n=1}^\infty n\psi_{-n}^i \psi_n^i
\end{equation}
不过从\ref{eq:3.14}能看出NS部分依旧存在快子态。
\section{GSO投影}
RNS形式只保证了世界面上的超对称性,而我们更应当要求保留十维靶空间的超对称性。这一点需要在量子化的基础上剔除一些态。定义如下的G宇称算符:
\begin{equation}
\begin{aligned}
		G_{NS}&=(-1)^{F+1}=(-1)^{\sum_{r=1/2}^\infty \psi_{-r}^i\psi_r^i+1}\\
	G_R&=\Gamma_{11}(-1)^{\sum_{n=1}^\infty \psi_{-n}^i\psi_n^i}
\end{aligned}
\end{equation}
其中$\Gamma_{11}=\Gamma_{0}\Gamma_{1}\ldots\Gamma_{9}$。GSO投影要求NS部分的态满足$G_{NS}=+1$,显然快子态不满足这一要求,所以被剔除了,基态变为无质量矢量玻色子激发$\psi^i_{-1/2}\ket{0}_{NS}$。对于R部分,为了要求处于$G_R$本征态,则是要求剔除掉一般的手征。也就是说如果我们选取$\ket{\alpha;+}_R$\footnote{其实记号$\alpha$就已经表明了$\Gamma_{11}=+1$,这里后面加个$+$只是为了符号更加清晰。}作为基态,那么投影到$G_R=+1$,反之投影到$G_R=-1$。也就是说在GSO投影下,R部分基态从$8\oplus 8$破缺成了仅含一个手征$8$。对于开弦来说,取左右手征完全只是人为约定。

但是对于闭弦,左右模的R部分真空完全可以取相同或者相反手征,然后再把两部分GSO投影后的谱拼起来,这就得到了表\ref{tab:2}所示两种不同的自洽的闭弦构造。
\begin{table}[htbp]
	\centering
	\begin{tabular}{c|cc}
		\hline
		 &Type IIA &Type IIB\\
		 \hline
		$m^2=0$ 
		&\(\displaystyle
			\begin{gathered}
				|\dot\alpha;-\rangle_{\mathrm{R}}\otimes|\alpha;+\rangle_{\mathrm{R}}\\\tilde{\psi}_{-1/2}^i|0\rangle_{\mathrm{NS}}\otimes \psi_{-1/2}^j|0\rangle_{\mathrm{NS}}\\\tilde{\psi}_{-1/2}^i|0\rangle_{\mathrm{NS}}\otimes|\alpha;+\rangle_{\mathrm{R}}\\|\dot\alpha;-\rangle_{\mathrm{R}}\otimes \psi_{-1/2}^i|0\rangle_{\mathrm{NS}}
			\end{gathered}
		\)
		&\(\displaystyle
		\begin{gathered}
			|\alpha;+\rangle_{\mathrm{R}}\otimes|\alpha;+\rangle_{\mathrm{R}}\\\tilde{\psi}_{-1/2}^i|0\rangle_{\mathrm{NS}}\otimes \psi_{-1/2}^j|0\rangle_{\mathrm{NS}}\\\tilde{\psi}_{-1/2}^i|0\rangle_{\mathrm{NS}}\otimes|\alpha;+\rangle_{\mathrm{R}}\\|\alpha;+\rangle_{\mathrm{R}}\otimes \psi_{-1/2}^i|0\rangle_{\mathrm{NS}}
		\end{gathered}
		\)\\
		\hline
	\end{tabular}
	\caption{Type IIA/B超弦}
	\label{tab:2}
\end{table}

他们是可定向的闭弦理论,本身构造是不包含超弦的,为了引入开弦可以通过额外引入D膜。现在来观察无质量谱构成的超多重态:
\begin{equation}
	\text{type IIA: }(\mathbf{8_v}+\mathbf{8_c})\otimes(\mathbf{8_v}+\mathbf{8_s}),\quad \text{type IIB: }(\mathbf{8_v}+\mathbf{8_c})\otimes(\mathbf{8_v}+\mathbf{8_c})
\end{equation}
\begin{itemize}
	\item[$\bullet$]NS-NS部分:A/B型弦论都是$\mathbf{8_v}\otimes\mathbf{8_v}=\mathbf{1}+\mathbf{28}+\mathbf{35}=\phi\oplus B_{\mu\nu}\oplus G_{\mu\nu}$,分解为伸缩子,反对称$B$-场以及引力子
	\item[$\bullet$]NS-R和R-NS部分:注意到$\mathbf{8_v}\otimes\mathbf{8_s}=\mathbf{8_c}\oplus\mathbf{56_s}$以及$\mathbf{8_v}\otimes\mathbf{8_c}=\mathbf{8_s}\oplus\mathbf{56_c}$。所以A/B型弦论的两个部分都给出伸缩超伴子和引力超伴子。但是A型超弦NS-R和R-NS的手性不一样,B型则相同
	\item[$\bullet$]R-R部分:对于A型超弦$\mathbf{8_c}\otimes\mathbf{8_s}=\mathbf{8_v}\oplus\mathbf{56_t}$,分解为1-形式(矢量场)规范场和3-形式规范场;对于B型超弦$\mathbf{8_c}\otimes\mathbf{8_c}=\mathbf{1}+\mathbf{2}\mathbf{8}+\mathbf{3}\mathbf{5_+}$,分解为0-形式(标量场)、2-形式和4-形式规范场。这些场统称为R-R形式场,类似Yang-Mills场作为1-形式场$A^\mu$在世界线上的拉回与点粒子相互作用,高形式场可以与更高维带R-R荷的D膜相互作用,这是D膜作为BPS态在超弦中稳定存在的关键。
\end{itemize}
而且不难看出费米子自由度和玻色子自由度至少在$m^2=0$层面上是吻合的。实际上GSO投影后,玻色子(NS部分生成)和费米子(R部分生成)生成函数为:
\begin{equation}
	\begin{gathered}
		f_{\mathrm{NS}}(w)=\frac{1}{2\sqrt{w}}\left[\prod_{m=1}^{\infty}\left(\frac{1+w^{m-1/2}}{1-w^m}\right)^8-\prod_{m=1}^{\infty}\left(\frac{1-w^{m-1/2}}{1-w^m}\right)^8\right]=\frac{\vartheta_3^4(\tau)-\vartheta_4^4(\tau)}{2\eta^{12}(\tau)}\\
	f_{\mathrm{R}}(w)=8\prod_{m=1}^\infty\left(\frac{1+w^m}{1-w^m}\right)^8=\frac{\vartheta_2^4(\tau)}{2\eta^{12}(\tau)}
	\end{gathered}
\end{equation}
其中$\vartheta_k(\tau)|_{w:=\mathrm{e}^{2\pi\mathrm{i}\tau}}$是Jacobi-$\theta$函数,$\eta(\tau)|_{w:=\mathrm{e}^{2\pi\mathrm{i}\tau}}$是Dedekind-$\eta$函数,定义可在\cite{Blumenhagen:2013fgp}中找到。利用$\vartheta^4_3-\vartheta^4_4=\vartheta^4_2$\cite{wzx}可立刻说明上述两生成函数等价,从而说明了在自由度层面靶空间超对称的保留。

本论文主要考虑开弦盘面振幅的计算,构造中即包含开弦谱的理论为I型超弦。其可以看作是由IIB型超弦将世界面宇称提升为规范对称性,从而取世界面$\mathbb{Z}_2$轨形投影得到,轨形不动点带来$O9$平面自然使得靶空间存在$D9$膜,而且由于需要消去引力反常,所以需要开弦带有$SO(32)$或$E_8\times E_8$对称性\footnote{利用Green-Schwarz机制消去反常还允许$E_8\times U(1)^{248}$和$U(1)^{496}$,不过\cite{PhysRevLett.105.071601}指出这两个李群其实无法自洽消去反常。},只有前者对应I型超弦,所以要求有32个$D9$膜存在,后者可以在杂交弦中发挥作用。这样得到的超弦也可以称作IB型超弦,IIA型超弦由于左右模手征不同,不存在世界面宇称对称性,所以无法直接通过轨形投影得到对应得I型超弦理论,但是可以先通过T对偶将IIA型超弦转换为IIB型超弦,再同时取世界面和靶空间$\mathbb{Z}_2$轨形投影得到IA型超弦。由于宇称作为规范对称性存在,所以I型超弦是非定向超弦。

本论文并不详细讨论自洽弦理论的构造问题,仅仅考虑弦论振幅本身的计算问题。

\section{RNS超弦顶角算符}
\subsection{超鬼场真空}
由于BRST量子化中鬼场也会同样贡献产生算符,前面玻色弦中$bc$鬼场真空存在一些问题,$\beta\gamma$鬼场则更加麻烦。

\section{*弦理论之间的对偶关系}

	\chapter{弦微扰论}
\label{chap:4}
本节核心是将\ref{eq:2.39}的规范固定于黎曼曲面模空间相联系,并利用鬼场进行计算。弦微扰论目前前沿研究方向可见\cite{berkovits2022snowmasswhitepaperstring},超弦的规范固定比较复杂,详细可见\cite{Witten:2012bh}。本节还给出了一些玻色弦和超弦振幅的计算例子,以及Kawai-Lewellen-Tye关系\cite{Kawai:1985xq}和单值关系\cite{Bjerrum-Bohr:2009ulz}。

\section{模空间测度}
\ref{eq:2.39}的路径积分是对所有$\text{diff}\times\text{Weyl}$二维闭曲面等价类($\mathcal{D}g$)以及到靶空间的嵌入($\mathcal{D}X$)的求和。在数学上,$\text{diff}$的等价类由黎曼流形描述,多模去$\text{Weyl}$的等价类则由一维复流形,也就是黎曼曲面描述。我们首先从数学上对此进行叙述,这是代数几何中标准的内容\cite{forster2012lectures,schlichenmaier2010introduction,griffiths2014principles},我们选用更容易接受的物理些的讲法\cite{Giacchetto:2024aka,Staessens:2010vi}。
\subsection{黎曼曲面模空间}
黎曼曲面本身的定义是不含度规的,但是考虑在二维曲面$M$上加入不同的度规结构$[g]_{\text{diff}}$使之称为黎曼流形。下标$\text{diff}$提醒度量本身与坐标卡选取无关\footnote{在物理上微分同胚变换总是用不严谨的坐标变换替代,而数学上更偏向使用坐标无关的语言。}。可以证明任意一个度规结构$g$都给定了$M$上的复结构,而且此复结构仅仅依赖于度规的共形结构,也就是等价类$[g]_{\text{diff}\times\text{Weyl}}$,反过来,任意一个黎曼曲面都存在唯一一个与复结构相容的共形结构$[g]$。这样我们就把${\mathcal{D}[g]_{\text{diff}\times\text{Weyl}}}$的计算彻底与黎曼曲面上的复结构关联起来了。

类似微分同胚的定义,可以给出黎曼曲面之间全纯同构(共性等价)的概念,全纯同构对黎曼曲面进行了非常强的划分:

\begin{boxedtext}[单值化定理]
任何黎曼曲面$M$都共形等价于$\hat{M}/\pi_1(M)$,其中$\hat{M}$为$M$的泛覆叠空间,有下面几种情况:
\begin{equation*}
	\hat{M}=\left\{\begin{array}{c}\mathbb{C}\cup\{\infty\}\cong\mathbb{CP}^1\\\mathbb{C}\\\mathbb{H}\end{array}\right.
\end{equation*}
其中$\mathbb{H}$是上半复平面。
\end{boxedtext}

但是我们真正希望考虑的是复结构,复结构对黎曼曲面的刻画比全纯同构细的多。或者说不同的复结构之间可以是全纯同构的。而复结构的刻画就依赖于黎曼曲面的模空间,记为$\mathfrak{M}_g$,下标$g$表示亏格。而亏格为$g$的黎曼面上的所有度量结构记作$\mathcal{M}_g$。

上面的叙述似乎较为抽象,更加具体的方法是利用复结构与$[g]$的一一对应:
\begin{equation}
	\label{eq:4.1}
	\delta g_{ab}=\mathrm{diff}\oplus\mathrm{weyl}\oplus\mathrm{moduli}=-2(P_1\delta\sigma)_{ab}+(2\delta\omega-\nabla\cdot\delta\sigma)g_{ab}+\sum_{k=1}^{\dim\mathfrak{M}_g}\delta t^k\partial_{t^k}\hat{g}_{ab}
\end{equation}
这里利用了$n$阶对称无迹张量$v$到$n+1$阶对称无迹张量$u$的算符$P_n$:\footnote{这里$|b|$表示$b$不参与下标对称化。}
\begin{equation}
	(P_nv)_{a_1\cdots a_{n+1}}\equiv\nabla_{(a_1}v_{a_2\cdots a_{n+1})}-\frac{n}{n+1}g_{(a_1a_2}\nabla_{|b|}v_{a_3\cdots a_{n+1})}^b
\end{equation}
且此算符是唯一的,其共轭算符$P_n^T$定义为:
\begin{equation}
	(P_n^Tu)_{a_1\cdot\cdot\cdot a_n}\equiv-\nabla_bu^b{}_{a_1\cdot\cdot\cdot a_n}
\end{equation}
式\ref{eq:4.1}中$\mathrm{diff}\oplus\mathrm{weyl}$是规范冗余,是等价类$[g]$内部的映射,而$\mathrm{moduli}$则是黎曼面上不同的复结构,是等价类之间的变换,$t^k$用来标记不同复结构。计算$\mathcal{D}[g]$首先要选取一个规范固定$\hat{g}$,然后利用$\mathrm{diff}\oplus\mathrm{weyl}$规范变换积分掉整个等价类$[\hat{g}]$,与$V_{\mathrm{diff}\times\mathrm{weyl}}$抵消,然后在模空间上进行积分跑遍所有的等价类$[g]$。先取规范固定,然后积分掉规范自由度的过程,可以用图\ref{fig:3}表示。思想其实和Yang-Mills理论一样,但是弦论中选取一个规范固定无法用规范变换$\zeta$跑遍所有的$\mathcal{D}g$,所以还需要最后用模空间积分遍历。
\begin{figure}[htbp]
	\centering
	\includegraphics[width=0.5\linewidth]{figs/fig3.pdf}
	\caption{规范固定}
	\label{fig:3}
\end{figure}

虽然前面把$[g]$和复结构对应起来了,这告诉我们路径积分包含模空间的积分,但是为了对$\mathrm{diff}\oplus\mathrm{weyl}$积分,取规范固定$\hat{g}$后完全定下规范了吗?或者说我们建立了和规范变换$\zeta$和$[\hat g]$内元素的一一对应吗?这样$\int [\mathcal{D}\zeta]$才等于$V_{\mathrm{diff}\times\mathrm{weyl}}$从而完全消去规范冗余。但显然不是的,从前面光锥规范就能看出,单纯取等温坐标固定$g$没有完全定下规范,还又共形变换作为$\mathrm{diff}\oplus\mathrm{weyl}$的子群没有固定。同样的,规范变换$\zeta$跑遍了$[\hat g]$中的元素,但是$\mathrm{diff}\oplus\mathrm{weyl}$变换的子群CKG变换$\hat g$后仍得到$\hat g$,也就是说在规范固定点$\hat g$处仍有冗余的规范$V_{\mathrm{CKG}}$没有被消除。在数学上与之相关的群被称为黎曼曲面的自同胚群$\operatorname{Aut}(M)$。如果这个群是离散群,那么只需要除去$n_R=|\operatorname{Aut}(M)|$就可以消去规范,但是如果这个群是连续李群,这对应黎曼曲面$\pi_1(M)$是阿贝尔群的情况\footnote{对应的黎曼曲面称为例外黎曼面。},CKG的消去可以通过固定黎曼面上的某几个点来消去,也就是说规范选取变成了$(\hat g,\hat\sigma_{i\in\mathcal{F}})$,可以解释为固定了其中几个顶角算符的插入点\footnote{本文不考虑顶角算符个数不足以固定插入点的情况。},具体是固定哪几个顶角算符以及固定点的坐标$z_{i\in\mathcal{F}}$并不会影响最终的关联函数结果。自然联想到这种顶角算符的固定是通过积分顶角算符到无积分顶角算符之间的转换实现的,后面会看到的确如此。

上面规范固定的过程可以看作是下面的积分测度变换:
\begin{equation}
	\label{eq:4.4}
	\mathcal{D}gd^{2n}\sigma\to|J|\mathcal{D} \zeta d^{\mu}td^{2n-\kappa}\sigma
\end{equation}
其中$|J|$是变换的雅可比行列式。其中$\mu = \dim\mathfrak{M}_{g,n}$,$\kappa$则是CKG生成元个数。注意,在黎曼曲面上固定一个点需要一个“复”的CKG生成元,也就是一对实的CKG生成元来固定,例外是对于开弦顶角算符,由于其插入点在盘面边界圆周$\operatorname{Re}z=0$,所以只需要一个实的CKG生成元就能固定。后面谈到维数均指复维数。下面来计算几个简单黎曼面的自同胚群。

\begin{boxedtext}[黎曼曲面自同胚群]
	黎曼曲面$M$的自同胚群可以利用其基本群在其泛覆叠空间$\hat M$中的正规化子计算:
	\begin{equation*}
		\operatorname{Aut}(M)\cong N(\pi_1(M))/\pi_1(M),\quad N(G):=\{h\in \operatorname{Aut}(\hat{M})|hGh^{-1}=G\}
	\end{equation*}
\end{boxedtext}
所以首先要对三种不同的泛覆叠空间的自同胚群进行计算,结果如下:
\begin{equation}
	\operatorname{Aut}(\mathbb{CP}^1)\cong PSL(2,\mathbb{C}),\quad
	\operatorname{Aut}(\mathbb{C})\cong \operatorname{Aff}(1,\mathbb{C}),\quad
	\operatorname{Aut}(\mathbb{H})\cong PSL(2,\mathbb{R})
\end{equation}
注意到比如$\mathbb{CP}^1$和$\mathbb{H}$的自同胚群都包含两个连通分支,其中单位元存在的连通分支$\operatorname{Aut}_0(M)$才生成CKG。闭弦树级振幅涉及到球面,对应$\operatorname{Aut}_0(S^2)=SL(2,\mathbb{C})$,其有三个复自由度,所以可以固定球面上三个点。一圈振幅对应环面$\operatorname{Aut}_0(T^2)=\mathcal{T}_\mathbb{C}$,即由复平面上的平移群生成,所以可以在环面上固定一个点。同时离散对称性$\sigma^a\to -\sigma^a$同样不会改变环面上的$\hat{g}$,这个$\mathbb{Z}_2$对称性给出$n_R=2$。开弦树级振幅对应盘面也即上半复平面,对应CKG为$\operatorname{Aut}_0(D_2)\cong SL(2,\mathbb{R})$,有三个实自由度,所以同样可以固定盘面边界上三个顶角算符插入点。

模去共形Killing群后,我们考虑的黎曼曲面$\mathcal{M}_g$变成了带标记点的黎曼曲面$\mathcal{M}_{g,n}$,现在来关注其模空间$\mathfrak{M}_{g,n}$。注意到$\delta_m g_{ab}$与$\mathrm{diff\times weyl}$正交:
\begin{equation}
	\begin{aligned}
		0&=\int d^2\sigma g^{1/2}\delta_mg_{ab}\left[-2(P_1\delta\sigma)^{ab}+(2\delta\omega-\nabla\cdot\delta\sigma)g^{ab}\right]\\
		&=\int d^2\sigma g^{1/2}\left[-2(P_1^T\delta^{\prime}g)_a\delta\sigma^a+\delta_mg_{ab}g^{ab}(2\delta\omega-\nabla\cdot\delta\sigma)\right]\\
		&\Rightarrow  g^{ab}\delta_m g_{ab} =0,\quad (P^T_1\delta_m g)_a=0
	\end{aligned}
\end{equation}
第一个无迹条件自动满足,迹包含在Weyl变换项中,第二个条件说明模空间对应$\ker P^T_1$。另外CKG生成元满足的共形Killing方程可以写为:
\begin{equation}
	(P_1\delta\sigma)_{ab}=0
\end{equation}
所以CKG对应$\ker P_1$,模空间维数与CKG维数之间有如下公式:
\begin{boxedtext}[Riemann-Roch公式]
	\begin{equation}
		\label{eq:4.8}
		\dim\ker P_n-\dim\ker P_n^T=(n+\frac12)\chi=(2n+1)(1-g)
	\end{equation}
\end{boxedtext}
上式\ref{eq:4.8}只是Riemann-Roch定理的一个特例。注意到$\kappa$正好对应黎曼曲面上固定点的个数,所以上述结果可以推广到固定点任意多的黎曼曲面模空间:
\begin{boxedtext}[模空间维数]
	亏格为$g$且带$n$个标记点的黎曼曲面模空间是一个连通光滑的复轨形,维数为:
	\begin{equation}
		\label{eq:4.9}
		\dim\mathfrak{M}_{g,n} = 3 g - 3 + n
	\end{equation}
	且我们考虑$2g-2+n>0$情况,这对应$\operatorname{Aut}(M_{g,n})$是有限群,也即固定点后完全模去了CKG。
\end{boxedtext}
模空间是一个复轨形来源于其有如下的计算方式\footnote{考虑不带标记点的简单情况。}:
\begin{boxedtext}[Teichm\"uller空间]
	$\mathrm{Diff}^+$表示保定向的微分同胚变换,$\mathrm{Diff}_0$表示与单位映射同伦的微分同胚。则定义模群$\Gamma_g$\footnote{也常称为Mapping Class Group。}和Teichm\"uller空间$\mathfrak{T}_g$:
	\begin{equation*}
		\Gamma_\mathrm{g}:=\mathrm{Diff}^+(M)/\mathrm{Diff}_0(M),\quad \mathfrak{T}_\mathrm{g}\equiv\frac{\mathcal{M}_g}{\mathrm{Weyl}(M)\times\mathrm{Diff}_0(M)}
	\end{equation*}
	黎曼曲面模空间有如下轨形形式:
	\begin{equation}
		\label{eq:4.10}
		\mathfrak{M}_\mathrm{g}=\mathfrak{T}_\mathrm{g}/\Gamma_\mathrm{g}
	\end{equation}
\end{boxedtext}
利用\ref{eq:4.9}计算发现球面$g=0,n=3$情况模空间平凡,所以树图振幅不涉及模空间积分的计算。第一个非平凡的例子是环面$g=1,n=1$,模空间维数为$1$,环面上的复结构由下面的格生成:
\begin{equation}
	\Gamma:=\mathbb{Z}\alpha_1+\mathbb{Z}\alpha_2=\{n\alpha_1+m\alpha_2:n,m\in\mathbb{Z}\}
\end{equation}
$T^2 \cong  \mathbb{C}/\Gamma$,$\alpha_{1,2}\in\mathbb{C}$。不同复结构由$SL(2,\mathbb{Z})$意义下不同构的格生成。两个复自由度$\alpha_1,\alpha_2$约化为一个。在\ref{eq:4.10}的观点下,环面的模空间为:
\begin{equation}
	\mathfrak{M}_{1,1}=\mathbb{H}/PSL(2,\mathbb{Z})
\end{equation}
所以模空间参数$\tau$可以通过模群限制在图\ref{fig:5}所示的阴影部分中,即模空间积分范围为:
\begin{equation}
	\mathscr{F}:=\left\{\tau\in\mathscr{H}:-\frac{1}{2}\leq\mathrm{Re}(\tau)\leq\frac{1}{2},|\tau|\geq1\right\}
\end{equation}
这种模空间边界的自然存在性,或者说因为模不变性,让弦论自然拥有一个截断,从而是紫外有限的理论。\cite{Witten:2015mec}
\begin{figure}[htbp]
	\centering
	\includegraphics{figs/fig4.pdf}
	\caption{环面上不同的复结构,在$SL(2,\mathbb{Z})$同构的意义下,取$(\alpha_1,\alpha_2)=(1,\tau)$}
	\label{fig:4}
\end{figure}
\begin{figure}[htbp]
	\centering
	\includegraphics{figs/fig5.pdf}
	\caption{环面模空间的基本域,$\rho:=e^{2\pi i/6}$}
	\label{fig:5}
\end{figure}

在黎曼面上积分的想法近年来也从弦论渗透到了场论振幅计算中,Cachazo-He-Yuan形式给了场论振幅利用带标记点黎曼面上积分的统一形式\cite{Cachazo:2013iea,Cachazo:2013hca}:
\begin{equation}
	\mathcal{A}_{n}^\text{tree}=\int d\mu_{n}\mathcal{I}_{n}^L\mathcal{I}_{n}^R,\quad d\mu_{n}=\frac{d^{n}\sigma}{\mathrm{volSL}(2,\mathbb{C})}{\prod_{a}}^{\prime}\delta{\left(\sum_{b\neq a}\frac{s_{ab}}{\sigma_{ab}}\right)}
\end{equation}
不同场论的区别在于CHY被积函数$\mathcal{I}$,但是树级振幅都有上述的统一形式!
\subsection{FP量子化}
现在来计算\ref{eq:4.4}中的Jacobi行列式,这个积分测度变换对应插入:
\begin{equation}
	1=\Delta_{\mathrm{FP}}(g,\sigma)\int_\mathfrak{M}d^\mu t\int_\mathrm{diff\times Weyl}\mathcal{D}\zeta\delta(g-\hat{g}(t)^\zeta)\prod_{(a,i)\in \mathcal{F}}\delta(\sigma_i^a-\hat{\sigma}_i^{\zeta a})
\end{equation}
利用\ref{eq:4.1}以及标准的FP鬼场方法计算得到:
\begin{equation}
	\label{eq:4.16}
	\Delta_{\text{FP}}=\frac{1}{n_R}\int\mathcal{D}b_{ab}\mathcal{D}c^a\exp(-S_g)\prod_{k=1}^\mu\frac{1}{4\pi}(b,\partial_{t^k}\hat{g})\prod_{(a,i)\in \mathcal{F}}c^a(\hat{\sigma}_i),\quad S_g=\frac{1}{2\pi}\left(b,(\hat{P}_1c)\right)
\end{equation}
当$\hat g$取共形规范时$b_{ab}$和$c^a$退化为全纯和反全纯左右模,也即\ref{eq:2.32}形式。这里涉及到对称无迹张量之间的内积,定义为:
\begin{equation}
	\label{eq:4.17}
	(t,t^\prime)_{\hat g}:=\int d\sigma^a {\hat g}^{\frac{1}{2}} (t\cdot t^\prime)
\end{equation}
这里$\cdot$表示对所有指标缩并。\ref{eq:2.39}规范固定后的形式为:
\begin{equation}
	\label{eq:4.18}
	\begin{aligned}
		S_{j_1...j_n}(k_1,\ldots,&k_n)=\sum_{\substack{\text{worldsheet}\\\text{topologies}}}\int_{\mathfrak{M}}\frac{d^\mu t}{n_R}\mathcal{D}X\mathcal{D}b\mathcal{D}c\exp(-S_\mathrm{m}-S_\mathrm{g}-\lambda\chi)\\&\times\prod_{(a,i)\notin \mathcal{F}}\int d\sigma_i^a\prod_{k=1}^\mu\frac{1}{4\pi}(b,\partial_{t^k}\hat{g})\prod_{(a,i)\in\mathcal{F}}c^a(\hat{\sigma}_i)\prod_{i=1}^n\hat{g}(\sigma_i)^{1/2}U_{j_i}(k_i,\sigma_i)
	\end{aligned}
\end{equation}
由此可见固定顶角算符插入点确实相当于将无积分顶角算符换为积分顶角算符。现在对$bc$进行模展开,将上式与$bc$鬼场零模的插入联系起来。
\begin{equation}
	\begin{gathered}
		c^a(\sigma)=\sum_Jc_J\mathsf{C}_J^a(\sigma),\quad b_{ab}(\sigma)=\sum_Kb_K\mathsf{B}_{Kab}(\sigma)\\
		P_1^TP_1\mathsf{C}_J^a=\mu_J^{2}\mathsf{C}_J^a,\quad P_1P_1^T\mathsf{B}_{Kab}=\nu_K^2\mathsf{B}_{Kab}
	\end{gathered}
\end{equation}
$\mathsf{C}_J$,$\mathsf{B}_{K}$在\ref{eq:4.17}内积的意义下正交归一。而且两者的非零模之间有一一对应:
\begin{equation}
	\mathsf{B}_{Jab}=\frac{1}{\nu_J}(P_1\mathsf{C}_J)_{ab},\quad\nu_J=\mu_J\neq0
\end{equation}
而且零模正好是$\ker P_1^T$和$\ker{P_1}$中的向量。FP行列式\ref{eq:4.16}路径积分可以直接用模展开得到:
\begin{equation}
	\label{eq:4.21}
	\begin{aligned}
		\Delta_{\text{FP}}=&\int\prod_{k=1}^\mu db_{0k}\prod_{j=1}^\kappa dc_{0j}\prod_Jdb_Jdc_J\exp\left(-\frac{\nu_Jb_Jc_J}{2\pi}\right)\prod_{m=1}^\mu\frac{1}{4\pi}(b,\partial_{t^{m}}\hat{g})\prod_{(a,i)\in\mathcal{F}}c^a(\sigma_i)\\
		=&\int\prod_{k=1}^\mu db_{0k}\prod_{m=1}^\mu\left[\sum_{k^{\prime}=1}^\mu\frac{b_{0k^{\prime}}}{4\pi}\left(\mathrm{B}_{0k^{\prime\prime}},\partial_{t^m}\hat{g}\right)\right]
		\int\prod_{j=1}^\kappa dc_{0j}\prod_{(a,i)\in\mathcal{F}}\left[\sum_{j^{\prime}=1}^\kappa c_{0j^{\prime}}\mathsf{C}_{0j^{\prime}}^a(\sigma_i)\right]
		\\
		&\times\int\prod_Jdb_Jdc_J\exp\left(-\frac{\nu_Jb_Jc_J}{2\pi}\right)\\
		=&\det\frac{(\mathsf{B}_{0k},\partial_{t^m}\hat{g})}{4\pi}{\det}\mathsf{C}_{0j}^a(\sigma_i){\det}^{\prime}\left(\frac{P_1^TP_1}{4\pi^2}\right)^{1/2}
	\end{aligned}
\end{equation}
第二个等号利用了格拉斯曼变量积分的性质,只有在被积函数为积分变量的最高形式时才不为零。最后一个等号中${\det}^\prime$表示不考虑零模贡献,否则显然$\det=0$。由上式不难看出规范固定的过程正是插入$bc$鬼场零模,而Riemann-Sroch定理$\mu-\kappa$给出的正是鬼数补偿,其正好补偿背景鬼数\ref{eq:2.66}。

虽然鬼场方法是极具物理思想的方法,但是其推导出来的结果右有非常清晰的物理解释。\ref{eq:4.21}中第三项可以看作是一个归一化系数不用过多考虑,第二项$c$鬼场零模正好对应CKG生成元,第一项在数学上相当于插入一些Beltrami微分,是复结构的体现。而且,单纯从形式上来说\ref{eq:4.18}有下面更简单的形式:

注意我们将所有的顶角算符插入点全部固定,而增加了Beltrami微分,这也是鬼数补偿的要求。相当于考虑亏格相同,但是固定点更多的模空间:
\begin{equation}
	\mathfrak{M}_{g,n+n_c+n_o},\quad \dim\mathfrak{M} = -\frac32 \chi + \frac12 n_o+n_c
\end{equation}
固定点的信息被转移到了模空间中去。但是从计算的角度上看依旧是\ref{eq:4.18}更方便,因为模空间积分计算比较复杂。

前面都是对玻色弦考虑的,超弦的情况要复杂得多。对于本篇论文,只要知道树图是平凡的,我们只需要关注物质场关联函数计算以及$c$鬼场的插入即可。RNS超弦唯一多要求绘景数求和为$2$。
\section{树级关联函数计算}
本节计算树级物质场和鬼场的关联函数,直接从路径积分出发计算,并说明此结果于OPE计算得到的结果相同。这里我们只对玻色部分物质场进行计算,本章最后计算超弦振幅时会直接使用OPE计算费米部分关联函数。



不难发现上面直接从路径积分得到的结果正好是OPE给出的关联函数奇异部分,实际上这并不是巧合,在树图层面上亚纯函数就是只依赖于奇异性。在后面处理纯旋量超弦的关联函数时会较多使用OPE方法。

\ref{eq:4.21}就是在算剩下的鬼场关联函数,涉及到的就是零模积分。OPE对应零模计算,物质场零模没有,剩下的鬼场零模要额外计算。
\section{玻色弦振幅}
稍微详细讲一下盘面振幅固定三个点之后注意顺序
\section{弦振幅之间的关系}
\subsection{KLT关系}
\subsection{单值关系}

\section{RNS超弦振幅}

	\chapter{Berkovits超弦}
本节介绍靶空间超对称的纯旋量超弦,由Berkovits在2000年发现\cite{Berkovits:2000fe}。从历史上看最早试图从靶空间超对称引入超弦的形式是Green-Schwarz超弦,但是只能在非协变的光锥坐标下量子化。后来Siegle改进了这一形式但存在共形反常且与RNS形式不等价等诸多问题。Berkovits在Siegle的研究基础之上进行改进得到了纯旋量超弦。我们不打算沿用历史性的介绍(引用mafra),而改用自上而下的方式。本节首先从更简单的超对称点粒子模型出发,然后推广得到纯旋量超弦的作用量,更多细节详见()。
	\chapter{BCJ分子的构造}
本章是本论文的核心结论,首先介绍BCJ对偶,然后简述了从纯旋量形式出发如何得到SYM树级振幅以及任意点无质量态超弦盘面振幅公式。最后介绍了如何由此构造出树级SYM理论的BCJ分子。本章是充满技术性的章节,不少证明相当复杂,文中只给出了一些梗概,详细推导过程请见本文所引用文献。
\section{色-运动学对偶}

\section{SYM振幅的超旋量空间表述}

\section{超弦无质量态$n$点盘面振幅}

\section{利用纯旋量超弦构造Yang-Mills理论的树图BCJ分子}
	\chapter{结论}

本文首先从RNS超弦出发,对超弦无质量开弦盘面振幅和闭弦球面振幅进行了一些具体计算,并给出了超弦树级振幅的KLT关系以及单值关系。同时本文也从黎曼曲面的角度在数学上讨论了玻色弦振幅一般的数学形式,同时也对超弦振幅有类似的讨论,指出了超弦振幅计算存在的困难。由于超弦理论本身要求靶空间的超对称,弦振幅本身也自然满足的是靶空间超对称,而RNS超弦属于世界面超对称的弦理论,虽然表述上较为简单,但是靶空间超对称来源于GSO投影,可以看作是被“隐藏”了起来。所以在计算具有靶空间超对称的超弦振幅时有很大的困难。为此我们引入了在GS超弦以及Siegel超弦基础上发展而来的Berkovits(纯旋量)超弦。做为可协变量子化的靶空间超对称超弦理论,纯旋量超弦在超弦振幅的计算上有无限潜能。

本论文利用纯旋量超弦回顾了无质量态盘面开超弦振幅的计算,特别是巧妙利用顶角算符处于BRST上同调定义了多粒子超场从而给出了纯旋量超弦无质量态顶角算符OPE之间的一般形式:
\begin{equation}
	V_A(z_a)U_B(z_b) \to \frac{V_{[A,B]}(z_b)}{z_{ab}}, \quad U_A(z_a)U_B(z_b) \to \frac{U_{[A,B]}(z_b)}{z_{ab}}
\end{equation}
利用这一形式我们最终得到了无质量态盘面开弦超振幅的一般计算公式:
\begin{equation}
		\mathcal{A}_n(P)=(2\alpha^{\prime})^{n-3}\int d\mu_P^n\left[\prod_{k=2}^{n-2}\sum_{m=1}^{k-1}\frac{s_{mk}}{z_{mk}}A_n(1,2,\ldots,n)+\mathrm{perm}(2,3,\ldots,n-2)\right]
\end{equation}
由于球面振幅以及开闭弦混合振幅可以由开弦振幅表达,所以我们从纯旋量超弦出发完全计算得到了树级超弦振幅。并且同时得到了低能近似下的十维超对称杨-米尔斯理论(SYM)超振幅的纯旋量超空间上同调表述:
\begin{equation}
	A_n(P,n):=\sum_{XY=P}\langle M_XM_YM_n\rangle
\end{equation}
纯旋量超弦不仅给出了如此紧凑的振幅表达式,让我们一窥其中的自由李代数组合数学结构,而且还能够直接利用DDM基底构造出满足色-运动学对偶对偶的BCJ因子:
\begin{equation}
	N_{1|XnY|n-1}=(-1)^{|Y|-1}\langle V_{1X}V_{(n-1)\tilde{Y}}V_{n}\rangle
\end{equation}
这无疑也为我们研究微扰引力理论和规范理论之间的双复制关系提供了一种思路。而且纯旋量超弦振幅与SYM理论振幅在低能展开下的关系也给出了弦振幅深层次的数论结构。所以利用纯旋量超弦不仅能够实现RNS超弦难以计算的超弦振幅,还能够给出弦振幅中丰富的数学结构。

不过本论文并未讨论近年来发展迅速的纯旋量超弦圈级振幅的计算\cite{gzq},也未对含有质量态时超弦振幅的计算进行讨论。比如对第一激发态在文献\cite{Berkovits:2002qx,Chakrabarti:2018mqd,Chakrabarti:2018bah}中已有相关讨论,实际上纯旋量超弦BRST上同调包含所有超弦激发态(与RNS超弦粒子谱相同)\cite{Berkovits:2000nn,Berkovits:2001mx,Aisaka:2008vw}。但是具体对更高有质量激发态顶角算符的构造以及超弦振幅的计算仍是目前纯旋量形式发展的前沿问题。期待作者基于本文对超弦无质量态树级振幅的理解能在后续对弦微扰论的研究有所启发。
	
	% 参考文献
	\printbibliography
	
	% 致谢
	\begin{acknowledgements}
	感谢女朋友产生算符的不存在性让我能专心完成这篇毕业论文。
\end{acknowledgements}

	
	% 附录
	\appendix
	% 附录

\chapter{本论文主要使用到的算符乘积展开}
\label{appendix:A}
本附录给出一些振幅计算中经常用到的OPE以及能动张量等CFT相关约定,以供查阅。
\section{玻色弦CFT}
\subsection{自由玻色CFT}
作用量和能动张量定义为:
\begin{equation}
	\label{eq:A1}
	S=\frac{1}{2\pi\alpha^{\prime}}\int d^2z\mathrm{~}\partial X^\mu\bar{\partial}X_\mu,\quad
	T(z)=-\frac{1}{\alpha^{\prime}}:\partial X\partial X:
\end{equation}
根据运动方程$\partial\bar\partial X = 0$,$X(z,\bar z) = X(z) + \tilde X(\bar z)$ ,后面的讨论主要以左行模为主。OPE为:
\begin{equation}
	\label{eq:A2}
	X^\mu(z_1,\bar{z}_1)X^\nu(z_2,\bar{z}_2)\sim-\frac{\alpha^\prime}{2}\eta^{\mu\nu}\ln|z_{12}|^2
\end{equation}
平面波$:e^{ik\cdot X}:$共形权为$\left(\frac{\alpha^{\prime}k^2}{4},\frac{\alpha^{\prime}k^2}{4}\right)$,涉及到$e^A$形式算符的缩并有如下比较常用的等式:
\begin{equation}
\wick{\c A(z): \c B^n:(w)}=nA(z)B(w):B^{n-1}(w):
\end{equation}
\begin{equation}
	\wick{\c A(z):\c e^{B(w)}}:=A(z)B(w):e^{B(w)}:
\end{equation}
\begin{equation}
	\begin{aligned}
		\wick{:\c e^{A(z)}::\c e^{B(z)}:}&=\sum_{m,n,k}\frac{k!}{m!n!}\begin{pmatrix}m\\k\end{pmatrix}\begin{pmatrix}n\\k\end{pmatrix}[{A(z)}B(w)]^k:A^{m-k}(w)B^{n-k}(w):\\&=\exp\left\{\wick{\c A(z)\c B(w)}\right\}:e^{A(w)}e^{B(w)}:
	\end{aligned}
\end{equation}
共形权$h$初级场$\phi$满足:
\begin{equation}
	T(z)\phi(w)\sim\frac{h}{(z-w)^{2}}\phi(w)+\frac{1}{z-w}\partial_{w}\phi(w)
\end{equation}
对应模展开:
\begin{equation}
[L_m,\phi_n]=((h-1)m-n)\phi_{m+n}
\end{equation}
$TT$ OPE为:
\begin{equation}
	\label{eq:A8}
	\begin{aligned}
		T(z)T(w)\sim\frac{c/2}{(z-w)^4}+\frac{2T(w)}{(z-w)^2}+\frac{\partial_wT(w)}{z-w}
	\end{aligned}
\end{equation}
对于$D$维自由玻色场,$c= D$,上述$TT$ OPE对应Virasoro代数:
\begin{equation}
	[L_m,L_n] = (m-n)L_{m+n}+\frac{c}{12}\left(m^{3}-m\right)\delta_{m,-n}
\end{equation}
左右模之间OPE是解耦的,不过,根据加倍技巧,对于BCFT,左右模之间的OPE不是$0$:\footnote{形象理解为右行模在边界附近的反射的影响。}
\begin{equation}
	X^\mu(z_1)\tilde X^\nu(\bar z_2)\sim X^\mu(z_1)X^\nu(z'_2)\sim -\frac{\alpha^{\prime}}{2}\ln|z_1-\overline{z}_2|
\end{equation}
不过弦论中开弦BCFT计算只涉及到实轴上插入,所以不必过于在意上式。而在实轴上源点及其镜像重合,所以左右模OPE和纯左模OPE都会贡献:
\begin{equation}
	 X^\mu (y_1) X^\nu(y_2) \sim -2\alpha'\ln |y_1-y_2|
\end{equation}
这从\ref{eq:4.36}也能直接看出来,由于左右模非零,所以相较于\ref{eq:A2}来说$X(z,\bar z)X(w,\bar w)$ OPE会多一项贡献。这同时也印证了$\alpha'_{\text{cl}}\sim4\alpha'_{\text{op}}$。
\subsection{$bc$鬼场}
左模部分作用量和能动张量为:
\begin{equation}
	\label{eq:A12}
	S=\frac{1}{2\pi}\int d^2zb\bar{\partial}c,\quad T(z)=:(\partial b)c:-\lambda\partial(:bc:)
\end{equation}
$bc$是反对易的费米场,共形权和中心荷为
\begin{equation}
	h_b=\lambda,\quad h_c=1-\lambda\quad c=-3(2\lambda-1)^2+1
\end{equation}
对于弦论,$\lambda = 2$。OPE为:
\begin{equation}
	b(z)c(w)\sim\frac{1}{z-w}
\end{equation}
其它$bb$和$cc$ OPE平凡。
\section{$\mathcal{N}=1$ SCFT}
本节对应RNS超弦世界面CFT,玻色场OPE与上一节相同。
\subsection{自由费米CFT}
左行模部分作用量和能动张量为:
\begin{equation}
S=\frac{1}{2\pi}\int\mathrm{d}^2z\psi^\mu\overline{\partial}\psi_\mu	,\quad T(z) = -\frac{1}{2}:\psi\cdot\partial\psi:(z)
\end{equation}
中心荷为$D/2$,$\psi$共形权为$\frac12$。
\begin{equation}
	\psi^\mu(z)\psi^\nu(w) = \frac{\eta^{\mu\nu}}{z-w}
\end{equation}
$\psi X$之间OPE平凡。总的物质场能动张量为:
\begin{equation}
	T^\mathrm{m}(z)=-\frac{1}{\alpha^{\prime}}:\partial X\cdot\partial X:-\frac{1}{2}:\psi\cdot\partial\psi:
\end{equation}
超共性变换对应的总的物质场超流为:
\begin{equation}
	G^\mathrm{m}(z)=i\sqrt{\frac{2}{\alpha^{\prime}}}\psi^\mu\partial X_\mu
\end{equation}
$TT$ OPE仍旧满足共性代数\ref{eq:A8},超共形代数:
\begin{equation}
	\label{eq:A19}
	\begin{aligned}
		G^\mathrm{m}(z_1)G^\mathrm{m}(z_2)&\sim\frac{\frac{2}{3}c}{(z_1-z_2)^3}+\frac{2T^\mathrm{m}(z_2)}{(z_1-z_2)}\\
T^\mathrm{m}(z_1)G^\mathrm{m}(z_2)&\sim\frac{\frac{3}{2}G^\mathrm{m}(z_2)}{(z_1-z_2)^2}+\frac{\partial G^\mathrm{m}(z_2)}{(z_1-z_2)}
	\end{aligned}
\end{equation}
超共形权为$h$的超共形初级场定义为共形权为$h$的初级场$\phi_h$且额外满足:
\begin{equation}
	\label{eq:A20}
	G^\mathrm{m}(z_1)\phi_h(z_2)\sim\frac{\phi_{h+1/2}(z_2)}{(z_1-z_2)}
\end{equation}
超共形初级场对$(\phi_h,\phi_{h+1/2})$定义为$\psi_h$和$\psi_{h+1/2}$分别为超共形权$h$的超共形初级场和共形权为$h+1/2$的共形初级场,且:
\begin{equation}
	\label{eq:A21}
	G^\mathrm{m}(z_1)\phi_{h+1/2}(z_2)\sim\frac{h\phi_h(z_2)}{(z_1-z_2)^2}+\frac{\partial_{z_2}\phi_h(z_2)}{2(z_1-z_2)}
\end{equation}
\subsection{$\beta\gamma$鬼场}
$\beta\gamma$鬼场和$bc$鬼场非常相似,从作用量和能动张量就能看到这一点:
\begin{equation}
	S=\frac{1}{2\pi}\int d^2z\beta\bar{\partial}\gamma,\quad T(z)=:(\partial\beta)\gamma:-\lambda\partial(:\beta\gamma:)
\end{equation}
共形权和中心荷为:
\begin{equation}
	h_\beta=\lambda^2,\quad h_\gamma=1-\lambda,\quad c=3(2\lambda-1)^2-1
\end{equation}
弦论中$\lambda = \frac32$。区别主要在于$\beta\gamma$鬼满足的是玻色统计,OPE为:
\begin{equation}
	\beta(z)\gamma(w)\thicksim-\frac{1}{z-w}
\end{equation}
总的鬼场超流为:
\begin{equation}
	G^{(\mathrm{gh})}(z)=-\frac{1}{2}(\partial\beta)c+\frac{3}{2}\partial(\beta c)-2b\gamma
\end{equation}
超弦费米部分CFT最重要的是模展开包含NS和R两个部分:
\begin{equation}
\begin{gathered}
		\begin{cases}
		\psi_\mathrm{NS}^\mu(z)=\sum_{r\in\mathbb{Z}+\frac{1}{2}}\psi_r^\mu z^{-r-\frac{1}{2}}\\\psi_\mathrm{R}^\mu(z)=\sum_{n\in\mathbb{Z}}\psi_n^\mu z^{-n-\frac{1}{2}}
		\end{cases}
		,\quad 
		\begin{cases}G_{\mathrm{NS}}(z)=\sum_{r\in\mathbb{Z}+\frac{1}{2}}G_rz^{-r-\frac{3}{2}}\\G_{\mathrm{R}}(z)=\sum_{n\in\mathbb{Z}}G_nz^{-n-\frac{3}{2}}\end{cases}
		\\
	\begin{cases}\beta_{\mathrm{NS}}(z)=\sum_{r\in\mathbb{Z}+\frac{1}{2}}\beta_rz^{-r-\frac{3}{2}}\\\beta_{\mathrm{R}}(z)=\sum_{n\in\mathbb{Z}}\beta_nz^{-n-\frac{3}{2}}\end{cases},\quad
				\begin{cases}\gamma_{\mathrm{NS}}(z)=\sum_{r\in\mathbb{Z}+\frac{1}{2}}\gamma_rz^{-r+\frac{1}{2}}\\\gamma_{\mathrm{R}}(z)=\sum_{n\in\mathbb{Z}}\gamma_nz^{-n+\frac{1}{2}}\end{cases}
\end{gathered}
\end{equation}
OPE \ref{eq:A19}给出的超共形代数可以写成下面的统一形式:
\begin{equation}
	\begin{aligned}&[L_n,G_r]=\frac{n-2r}{2}G_{n+r}\\&\{G_r,G_s\}=2L_{r+s}+\frac{c}{12}(4r^2-1)\delta_{r+s,0}\end{aligned}
\end{equation}
NS和R部分只需要对下标取整数或半整数即可。另外上面讨论并未对$G$加上标m,gh,tot区分具体是物质场超流、鬼场超流还是总超流,因为他们满足的超共形代数是一样的。
\subsection{超空间表述}
用(二维)超对称场论的语言可以把前面的讨论写成更加显现出$\mathcal{N}=1$超对称的形式。引入和$\{z,\bar z\}$对应的超空间坐标$\{\theta,\bar\theta\}$,他们是二维Weyl旋量。超空间导数定义为:
\begin{equation}
	D=\partial_\theta+\theta\partial_z,\quad\bar{D}=\partial_{\bar{\theta}}+\bar{\theta}\partial_{\bar{z}}
\end{equation}
考虑坐标变换:
\begin{equation}
	z=(z,\theta)\to z^{\prime}=(z^{\prime}(z,\theta),\theta^{\prime}(z,\theta))
\end{equation}
共性变换由$\bar \partial z' = 0$的全纯映射生成,普通偏导数由链式法则变换为$\partial=\frac{\partial z^{\prime}}{\partial z}\partial^{\prime}$,这一点是平凡的,超共形变换则是类似要求:
\begin{equation}
	D=(D\theta^{\prime})D^{\prime}\Rightarrow Dz^{\prime}-\theta^{\prime}D\theta^{\prime}=0
\end{equation}
诱导出如下变换:
\begin{equation}
	\begin{aligned}&z^{\prime}=f(z)+\frac{1}{2}\theta g(z)\epsilon(z),\\&\theta^{\prime}=\frac{1}{2}\epsilon(z)+\theta g(z),\quad g^{2}=\partial f+\frac{1}{4}\epsilon\partial\epsilon\end{aligned}
\end{equation}
$f$和$g$是普通函数,$\epsilon$则是Grassmann反对易的函数。此式便是世界面上超共形变换的形式,$\theta = 0$时上式退化为共形变换。上式的无穷小变换形式为:
\begin{equation}
	\label{inf}
	\delta z=\xi+\frac{1}{2}\theta\epsilon,\quad\delta\theta=\frac{1}{2}\epsilon+\frac{1}{2}\theta\partial\xi
\end{equation}
这里$\xi$对易,$\epsilon$反对易。超共形初级场对构成一个手征超场:
\begin{equation}
	\Phi_h(z)=\phi_h(z)+\theta\phi_{h+\frac12}(z),\quad \bar D\Phi(z) = 0
\end{equation}
类似共形初级场定义要求共形变换下:
\begin{equation}
	\phi^{\prime}(z^{\prime},\bar{z}^{\prime})=\left(\frac{\partial z^{\prime}}{\partial z}\right)^{-h}\left(\frac{\partial\bar{z}^{\prime}}{\partial\bar{z}}\right)^{-\bar{h}}\phi(z,\bar{z})
\end{equation}
超共形初级场要求超共形变换下:
\begin{equation}
	\Phi(z)=(D\theta^{\prime})^{2h}\Phi^{\prime}(z^{\prime})
\end{equation}
利用\ref{inf}可以导出上面要求的无穷小变换下对$\phi$提出的要求,而$\delta_\xi$由能动张量$T$的OPE生成,$\delta_\epsilon$由超流$G$的OPE生成,由此可以得到正好是要求$\phi$满足OPE \ref{eq:A20}和\ref{eq:A21},且是通常意义下的共形初级场。能动张量和超流同样可以组成一个超场:
\begin{equation}
	\label{eq:A36}
	\mathscr{T}^\text{m}(z)=G^\text{m}(z)+\theta T^\text{m}(z)
\end{equation}
而且是共形权为$\frac32$的共形超场,\ref{eq:A20}和\ref{eq:A21}的要求转化为:
\begin{equation}
	\mathscr{T}^{\text{m}}(z_1)\Phi(z_2)\sim\frac{h\theta_{12}\Phi(z_2)}{z_{12}^2}+\frac{\frac{1}{2}D\Phi(z_2)}{z_{12}}+\frac{\theta_{12}D^2\Phi(z_2)}{z_{12}}
\end{equation}
类似,\ref{eq:A19}可以被紧凑的写为:
\begin{equation}
	\mathscr{T}^\text{m}(z_1)\mathscr{T}^\text{m}(z_2)\sim\frac{\frac{1}{6}{c^\text{m}}}{z_{12}^3}+\frac{\frac{3}{2}\theta_{12}\mathscr{T}(z_2)}{z_{12}^2}+\frac{\frac{1}{2}D\mathscr{T}(z_2)}{z_{12}}+\frac{\theta_{12}D^2\mathscr{T}(z_2)}{z_{12}}
\end{equation}
RNS超弦中物质场可以组合为一个二维超场:
\begin{equation}
	\mathscr{X}^\mu(z,\bar{z})=X^\mu+i\theta\psi^\mu+i\bar{\theta}\bar{\psi}^\mu+\theta\bar{\theta}F^\mu
\end{equation}
$F^\mu$是人为引入的保证超对称性的辅助场。利用这个定义可以将物质场能动张量写成和\ref{eq:A1}一致且明显保持共形超对称性的形式:
\begin{equation}
	\begin{aligned}
		S_\text{m}&=\frac{1}{2\pi\alpha'}\int d^2zd^2\theta\bar{D}\mathscr{X}^\mu D\mathscr{X}_\mu\\&=\frac{1}{2\pi\alpha'}\int d^2z\left(\partial X^\mu\overline{\partial}X_\mu+\psi^\mu\overline{\partial}\psi_\mu+\overline{\psi}^\mu\partial\overline{\psi}_\mu+F^\mu F_\mu\right)
	\end{aligned}
\end{equation}
$\delta S_m/\delta F$给出$F=0$,所以辅助场其实可以略去,这就完全回到了前面的讨论,同样,对于鬼场,可以组合成两个共形超场:
\begin{equation}
	B=\beta+\theta b,\quad C=c+\theta\gamma
\end{equation}
超共形权分别为$\frac32$和$-1$,鬼场能动张量可以写成\ref{eq:A12}的形式:
\begin{equation}
	S_{\text{gh}}=\frac{1}{2\pi}\int d^2zd^2\theta B\overline{D}C=\frac{1}{2\pi}\int d^2z(b\overline{\partial}c+\beta\overline{\partial}\gamma)
\end{equation}
同样可以类似\ref{eq:A36}引入$\mathscr{T}^{\text{gh}}$,超共形代数是一样的。
\section{纯旋量形式}
这里只讨论左模,作用量见\ref{eq:5.23},能动张量以及对应的超流(费米Lorentz流):
\begin{equation}
	T_{\mathrm{PS}} = :-\frac{1}{2} \Pi^\mu \Pi_\mu - d_\alpha \partial \theta^\alpha + w_\alpha \partial \lambda^\alpha:, \quad M^{\mu\nu} = \Sigma^{\mu\nu} + N^{\mu\nu} = :-\frac{1}{2} (p \gamma^{\mu\nu} \theta) + \frac{1}{2} (w \gamma^{\mu\nu} \lambda):
\end{equation}
其中$h(\lambda^\alpha) = 0$,$h(w_\alpha) = +1$。由于纯旋量约束,这并不是一个自由CFT:
\begin{equation}
	\begin{aligned}
		X^\mu(z,\overline{z}) X^\nu(w,\overline{w}) 
		&\sim -\eta^{\mu\nu} \ln|z-w|^2, 
		& d_\alpha(z) \theta^\beta(w) 
		&\sim \frac{\delta_\alpha^\beta}{z-w}, \\
		d_\alpha(z) d_\beta(w) 
		&\sim -\frac{\gamma_{\alpha\beta}^\mu \Pi_\mu(w)}{z-w}, 
		& d_\alpha(z) \Pi^\mu(w) 
		&\sim \frac{(\gamma^\mu \partial\theta(w))_\alpha}{z-w}, \\
		\Pi^\mu(z) \Pi^\nu(w) 
		&\sim -\frac{\eta^{\mu\nu}}{(z-w)^2},&
		\partial\theta^\alpha(z) \{ \partial\theta^\beta(w),\Pi^\mu(w),&N^{\mu\nu}(w) \} \sim \mathrm{regular}
	\end{aligned}
\end{equation}
对于任意的不显含$X,\theta$导数项的超场$K(X,\theta)$:\footnote{平面波就是一个最简单的例子。}
\begin{equation}
	\begin{aligned}
		d_\alpha(z) K\left(X(w,\overline{w}), \theta(w)\right) 
		&\sim \frac{D_\alpha K\left(X(w,\overline{w}), \theta(w)\right)}{z-w}, \\
		\Pi^\mu(z) K\left(X(w,\overline{w}), \theta(w)\right) 
		&\sim -\frac{\partial^\mu K\left(X(w,\overline{w}), \theta(w)\right)}{z-w}.
	\end{aligned}
\end{equation}
其中超空间导数为:
\begin{equation}
	D_\alpha=\frac{\partial}{\partial\theta^\alpha}+\frac{1}{2}(\gamma^\mu\theta)_\alpha\partial_\mu
\end{equation}
还有一些有关超流的OPE:
\begin{equation}
	\begin{aligned}
		M^{\mu\nu}(z) M^{\rho\sigma}(w) 
		&\sim \frac{\eta^{\rho[\mu} M^{\nu]\sigma}(w) - \eta^{\sigma[\mu} M^{\nu]\rho}(w)}{z-w} + \frac{\eta^{\mu[\sigma} \eta^{\rho]\nu}}{(z-w)^2}, \\
		N^{\mu\nu}(z) N^{\rho\sigma}(w) 
		&\sim \frac{\eta^{\rho[\mu} N^{\nu]\sigma}(w) - \eta^{\sigma[\mu} N^{\nu]\rho}(w)}{z-w} - 3 \frac{\eta^{\mu[\sigma} \eta^{\rho]\nu}}{(z-w)^2}, \\
		N^{\mu\nu}(z) \lambda^\alpha(w) 
		&\sim \frac{1}{2} \frac{(\gamma^{\mu\nu})^\alpha{}_\beta \lambda^\beta(w)}{z-w}.
	\end{aligned}
\end{equation}
	\chapter{一些常用的$\gamma$矩阵计算恒等式}
本附录默认$D=10$,且所有指标置换操作定义本身不带有$k!$因子。首先是一些符号约定:
\begin{equation}
\begin{aligned}
		\gamma^{\mu_1\mu_2...\mu_n} = \frac{1}{n!} \gamma^{[\mu_1} \gamma^{\mu_2} \cdots \gamma^{\mu_n]}\\
	\delta_{b_1b_2...b_n}^{a_1a_2...a_n}=\frac{1}{n!}\delta_{b_1}^{[a_1}\delta_{b_2}^{a_2}\cdots\delta_{b_n}^{a_n]}\\
	\epsilon_{01...9}=1,\quad\epsilon^{01...9}=-1
\end{aligned}
\end{equation}
Fierz恒等式为:
\begin{align}
		\psi^\alpha \chi^\beta &= \frac{1}{16} \gamma_{\mu_1}^{\alpha\beta} (\psi \gamma^{\mu_1} \chi) + \frac{1}{96} (\gamma_{\mu_1...\mu_3})^{\alpha\beta} (\psi \gamma^{\mu_1...\mu_3} \chi) + \frac{1}{3840} (\gamma_{\mu_1...\mu_5})^{\alpha\beta} (\psi \gamma^{\mu_1...\mu_5} \chi) \\
		\psi_\alpha \chi^\beta &= \frac{1}{16} \delta_\alpha^\beta (\psi \chi) + \frac{1}{32} (\gamma_{\mu_1\mu_2})_\alpha^\beta (\psi \gamma^{\mu_1\mu_2} \chi) + \frac{1}{384} (\gamma_{\mu_1...\mu_4})_\alpha^\beta (\psi \gamma^{\mu_1...\mu_4} \chi)
\end{align}
利用$\theta$的反对易性和$\lambda$的纯旋量性,上式有特殊情况:
\begin{equation}
	\lambda^\alpha \lambda^\beta = \frac{1}{3840} (\lambda \gamma^{\mu\nu\rho\sigma\tau} \lambda) \gamma_{\mu\nu\rho\sigma\tau}^{\alpha\beta}, \quad \theta^\alpha \theta^\beta = \frac{1}{96} (\theta \gamma^{\mu\nu\rho} \theta) \gamma_{\mu\nu\rho}^{\alpha\beta}
\end{equation}
$\gamma$矩阵的迹:
\begin{equation}
	\mathrm{Tr}\left(\gamma^P\gamma_Q\right)=16\delta^{p,q}\left[p!\delta_{Q^T}^P+\delta^{p,5}\epsilon_Q^P\right],\quad|P|:=p,\quad|Q|:=q
\end{equation}
对偶性:
\begin{equation}
	\begin{aligned}
		(\gamma^{\mu_1...\mu_5})_{\alpha\beta} 
		&= \frac{1}{5!} \epsilon^{\mu_1...\mu_5\nu_1...\nu_5} (\gamma_{\nu_1...\nu_5})_{\alpha\beta}, 
		& (\gamma^{\mu_1...\mu_5})^{\alpha\beta} 
		&= -\frac{1}{5!} \epsilon^{\mu_1...\mu_5\nu_1...\nu_5} (\gamma_{\nu_1...\nu_5})^{\alpha\beta}, \\
		(\gamma^{\mu_1...\mu_6})_{\alpha}^{\beta} 
		&= \frac{1}{4!} \epsilon^{\mu_1...\mu_6\nu_1...\nu_4} (\gamma_{\nu_1...\nu_4})_{\alpha}^{\beta}, 
		& (\gamma^{\mu_1...\mu_6})^{\alpha}_{\beta} 
		&= -\frac{1}{4!} \epsilon^{\mu_1...\mu_6\nu_1...\nu_4} (\gamma_{\nu_1...\nu_4})^{\alpha}_{\beta}, \\
		(\gamma^{\mu_1...\mu_7})_{\alpha\beta} 
		&= -\frac{1}{3!} \epsilon^{\mu_1...\mu_7\nu_1...\nu_3} (\gamma_{\nu_1...\nu_3})_{\alpha\beta}, 
		& (\gamma^{\mu_1...\mu_7})^{\alpha\beta} 
		&= \frac{1}{3!} \epsilon^{\mu_1...\mu_7\nu_1...\nu_3} (\gamma_{\nu_1...\nu_3})^{\alpha\beta}, \\
		(\gamma^{\mu_1...\mu_8})_{\alpha}^{\beta} 
		&= -\frac{1}{2!} \epsilon^{\mu_1...\mu_8\nu_1\nu_2} (\gamma_{\nu_1\nu_2})_{\alpha}^{\beta}, 
		& (\gamma^{\mu_1...\mu_8})^{\alpha}_{\beta} 
		&= \frac{1}{2!} \epsilon^{\mu_1...\mu_8\nu_1\nu_2} (\gamma_{\nu_1\nu_2})^{\alpha}_{\beta}, \\
		(\gamma^{\mu_1...\mu_9})_{\alpha\beta} 
		&= \epsilon^{\mu_1...\mu_9\nu_1} (\gamma_{\nu_1})_{\alpha\beta}, 
		& (\gamma^{\mu_1...\mu_9})^{\alpha\beta} 
		&= -\epsilon^{\mu_1...\mu_9\nu_1} (\gamma_{\nu_1})^{\alpha\beta}, \\
		(\gamma^{\mu_1...\mu_{10}})_{\alpha}^{\beta} 
		&= \epsilon^{\mu_1...\mu_{10}} \delta_{\alpha}^{\beta}, 
		& (\gamma^{\mu_1...\mu_{10}})^{\alpha}_{\beta} 
		&= -\epsilon^{\mu_1...\mu_{10}} \delta_{\beta}^{\alpha}.
	\end{aligned}
\end{equation}
$\gamma$矩阵的乘积:
\begin{equation}
	\gamma_{\mu_1\mu_2\ldots\mu_p} \gamma^{\nu_1\nu_2\ldots\nu_q} = \sum_{k=0}^{\min(p,q)} k! \binom{p}{k} \binom{q}{k} \delta_{[\mu_p}^{[\nu_1} \delta_{\mu_{p-1}}^{\nu_2} \cdots \delta_{\mu_{p-k+1}}^{\nu_k} {\gamma_{\mu_1\ldots\mu_{p-k}]}}^{\nu_{k+1}\ldots\nu_q]}
\end{equation}
其它常用等式:
\begin{align}
		\gamma_{\alpha(\beta}^\mu \gamma_{\gamma\delta)}^\mu &= 0, \\
		\gamma_{\alpha[\beta}^{\mu\nu\rho} \gamma_{\gamma\delta]}^{\mu\nu\rho} &= 0, \\
		\gamma_{\mu\nu\rho}^{\alpha\beta} \gamma_{\gamma\delta}^{\mu\nu\rho} &= 48 \left( \delta_\gamma^\alpha \delta_\delta^\beta - \delta_\gamma^\beta \delta_\delta^\alpha \right), \\
		\gamma_{\alpha\beta}^{\mu\nu\rho} \gamma_{\gamma\delta}^{\mu\nu\rho} &= 12 \left( \gamma_{\alpha\delta}^\mu \gamma_{\beta\gamma}^\mu - \gamma_{\alpha\gamma}^\mu \gamma_{\beta\delta}^\mu \right), \\
		\gamma_{\alpha\beta}^\mu \gamma_{\delta\sigma}^\mu &= -\frac{1}{2} \gamma_{\alpha\delta}^\mu \gamma_{\beta\sigma}^\mu - \frac{1}{24} \gamma_{\alpha\delta}^{\mu\nu\rho} \gamma_{\beta\sigma}^{\mu\nu\rho}, \\
		\gamma_{\alpha\beta}^{\mu\nu\rho} \gamma_{\delta\sigma}^{\mu\nu\rho} &= -18 \gamma_{\alpha\delta}^\mu \gamma_{\beta\sigma}^\mu + \frac{1}{2} \gamma_{\alpha\delta}^{\mu\nu\rho} \gamma_{\beta\sigma}^{\mu\nu\rho}, \\
		\gamma_{\alpha\beta}^{\mu\nu\rho} \gamma_{\delta\sigma}^{\mu\nu\rho} &= -12 \gamma_{\alpha\beta}^\mu \gamma_{\delta\sigma}^\mu - 24 \gamma_{\alpha\delta}^\mu \gamma_{\beta\sigma}^\mu, \\
		(\gamma^{\mu\nu})_\alpha^\delta (\gamma_{\mu\nu})_\beta^\sigma &= -8 \delta_\alpha^\sigma \delta_\beta^\delta - 2 \delta_\alpha^\delta \delta_\beta^\sigma + 4 \gamma_{\alpha\beta}^\mu \gamma_\mu^{\delta\sigma}, \\
		(\gamma^{\mu\nu\rho\sigma})_\alpha^\beta (\gamma_{\mu\nu\rho\sigma})_\sigma^\delta &= 315 \delta_{\alpha}^{\delta} \delta_{\sigma}^{\beta} + \frac{21}{2} (\gamma^{\mu\nu})_{\alpha}{}^{\delta} (\gamma_{\mu\nu})_{\sigma}{}^{\beta} + \frac{1}{8} (\gamma^{\mu\nu\rho\sigma})_{\alpha}{}^{\delta} (\gamma_{\mu\nu\rho\sigma})_{\sigma}{}^{\beta}, \\
		(\gamma^{\mu\nu\rho\sigma})_{\alpha}^{\beta} (\gamma_{\mu\nu\rho\sigma})_{\sigma}^{\delta} &= -48 \delta_\alpha^\beta \delta_\sigma^\delta + 288 \delta_\alpha^\delta \delta_\sigma^\beta + 48 \gamma_{\alpha\sigma}^\mu \gamma_\mu^{\beta\delta},\\
		\gamma_{\alpha\beta}^{\mu\nu\rho\sigma\tau} \gamma_{\delta\sigma}^{\mu\nu\rho\sigma\tau} &= 0, \\
		(\lambda \gamma^\mu)_\alpha (\lambda \gamma_\mu)_\beta &= 0, \\
		(\lambda \gamma_\mu)_\alpha (\lambda \gamma^{\mu\nu\rho\sigma\tau} \lambda) &= 0.
\end{align}
\end{document}