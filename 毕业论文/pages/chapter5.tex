\chapter{Berkovits超弦}
\label{chap:5}
本节介绍靶空间超对称的纯旋量超弦,由Berkovits在2000年发现\cite{Berkovits:2000fe}。从历史上看最早试图从靶空间超对称引入超弦的形式是Green-Schwarz超弦\cite{Green:1983wt,Green:1983sg},但是只能在非协变的光锥坐标下量子化。后来Siegle改进了这一形式但存在共形反常且与RNS形式不等价等诸多问题\cite{Siegel:1985xj}。Berkovits在Siegle的研究基础之上进行改进得到了纯旋量超弦。我们不打算沿用历史性的介绍\cite{Berkovits:2002zk,Mafra:2008gkx},而改用自上而下的方式。本章首先从更简单的超对称点粒子模型出发,然后推广得到纯旋量超弦的作用量,更多细节详见\cite{Berkovits:2017ldz,Mafra:2022wml}。

\section{Brink-Schwarz 超粒子}
本节目的是说明在引入靶空间旋量进行超对称化,会导致体系无法协变量子化,以更简单的点粒子模型来说明,弦论类似。靶空间超对称粒子作用量:\cite{Brink:1981nb,Ferber:1977qx}
\begin{equation}
	\label{eq:5.1}
	S_{\text{BS}}=\int d\tau\left(\Pi^\mu P_\mu+eP^\mu P_\mu\right),\quad\Pi^\mu:=\dot{X}^\mu-\frac{1}{2}\dot{\theta}^\alpha\gamma_{\alpha\beta}^\mu\theta^\beta
\end{equation}
这里旋量都是十维靶空间旋量,选取了Weyl基底\ref{eq:4.65},$P$是$X$共轭动量,且看作独立变量。$\mathcal{N}=1$超对称以及对应的超荷为:
\begin{equation}
	\label{eq:5.2}
	\begin{gathered}
		\delta\theta^\alpha=\epsilon^\alpha,\quad\delta X^\mu=\frac{1}{2}\theta^\alpha\gamma_{\alpha\beta}^\mu\epsilon^\beta,\quad\delta P^\mu=\delta e=0\\
		Q_\alpha:=p_\alpha-\frac{1}{2}\gamma_{\alpha\beta}^\mu\theta^\beta P_\mu,\quad p_\alpha:=\frac{\partial L}{\partial\dot{\theta}^\alpha}=-\frac{1}{2}\gamma_{\alpha\beta}^\mu\theta^\beta P_\mu
	\end{gathered}
\end{equation}
这一作用量其实是GS形式弦理论的无质量点粒子极限,GS形式下有额外的格拉斯曼奇的规范对称性,称为$\kappa$对称性:
\begin{equation}
	\delta\theta^\alpha=P^\mu\gamma_\mu^{\alpha\beta}\kappa_\beta,\quad\delta X^\mu=-\frac{1}{2}\theta^\alpha\gamma_{\alpha\beta}^\mu\delta\theta^\beta,\quad\delta P^\mu=0,\quad\delta e=\dot{\theta}^\alpha\kappa_\alpha
\end{equation}
再来看下该体系的约束,首先是$\delta S/\delta e$给出$P^2=0$的无质量约束,这类似前面能动张量给出的约束。另外引入了两对共轭变量$\{X,P\}$和$\{\theta,p\}$,前面一对可以看作是独立的,但是从$\ref{eq:5.2}$不难看出$p$的定义本身包含$\theta$,所以并不能看作完全独立,而是要求有下面的约束条件:
\begin{equation}
	d_\alpha:=p_\alpha+\frac{1}{2}\gamma_{\alpha\beta}^\mu\theta^\beta P_\mu=0
\end{equation}
利用共轭变量之间的泊松括号得到约束条件满足的代数结构:
\begin{equation}
	\label{eq:5.4}
	\{d_\alpha,d_\beta\}_{\mathrm{PB}}=-\gamma_{\alpha\beta}^\mu P_\mu
\end{equation}
对于无质量粒子取小群表示$P^\mu=(E,0,\ldots,E)$,并且取光锥坐标得到:\footnote{与第\ref{chap:2}章中不同,本章我们取光锥坐标为$\gamma^{\pm}:=\frac{1}{\sqrt{2}}(\gamma^0\pm\gamma^9)$。}
\begin{equation}
	\label{eq:5.5}
	\{d_\alpha,d_\beta\}_{\mathrm{PB}}=-\gamma_{\alpha\beta}^-P^+\propto\begin{pmatrix}{1}_{8\times8}&{0}_{8\times8}\\{0}_{8\times8}&{0}_{8\times8}\end{pmatrix}
\end{equation}
由此发现\ref{eq:5.4}给出的约束条件包含八个第一类约束和八个第二类约束,而且只有在不协变的光锥规范下,两类约束才不会混合在一起。所以GS形式只能在光锥坐标下进行量子化。

第一类约束比如前面玻色弦和RNS超弦量子化中能动张量给出的约束可以在协变量子化框架中将约束看作是\ref{eq:2.23}来解决,但是第二类约束需要用到$\S$\ref{sec:5.2}中的式\ref{2nd}先替换掉泊松括号,然后再进行正则量子化。下面我们直接选取规范固定将第一类约束解除,然后处理第二类约束。

在光锥规范下可以固定$\kappa$规范对称性为$(\gamma^+\theta)_\alpha=0$,只用选取$\kappa_\beta=\frac{1}{2P^+}(\gamma^+\theta)_\beta$对$\theta$变换即可做到这一点。在这一规范固定下,作用量\ref{eq:5.1}改写为:\footnote{计算需要利用靶空间Majorana-Weyl旋量性质$\dot{\theta}\gamma^+\theta=\dot{\theta}\gamma^i\theta=0$。}
\begin{equation}
S_{\text{BS}}=\int d\tau\left(\dot{X}^\mu P_\mu-\frac{1}{2}\dot{S}_aS_a+eP^\mu P_\mu\right),\quad S^a:=2^{1/4}\sqrt{P^+}\theta^a
\end{equation}
这里利用了十维Weyl旋量可以进一步分解为Weyl和反Weyl旋量,$\theta$规范固定后还剩下一半的分量:\footnote{这是将十维Weyl旋量分解为了两个八维Weyl旋量,而且八维Weyl旋量升降指标的度量矩阵是$\delta^{ab}$单位阵,一般情况下利用荷共轭矩阵的分量来升降指标\cite{Freedman:2012zz},比如四维情况下一般取基底使得升降矩阵为Levi-Civita张量$\varepsilon^{ab}$。}
\begin{equation}
	\theta^\alpha=\begin{pmatrix}\theta^a\\\theta^{\dot{a}}\end{pmatrix},\quad a,\dot{a}=1,2,\ldots,8
\end{equation}
在这一规范固定下\ref{eq:5.5}剩下八个第二类约束:
\begin{equation}
	p_a:=\frac{\partial L}{\partial\dot{S}^a}=-\frac{1}{2}S_a,\quad \{d_a,d_b\}_{\mathrm{PB}}=-\delta_{ab}
\end{equation}
利用\ref{eq:2nd}得到:
\begin{equation}
	\label{eq:5.10}
	\begin{aligned}
		\{S_a,S_b\}_*&=\{S_a,S_b\}_{\mathrm{PB}}-\{S_a,d_c\}_{\mathrm{PB}}\{d^c,d^e\}_{\mathrm{PB}}^{-1}\{d_e,S_b\}_{\mathrm{PB}}\\&=0-(-\delta_{ac})(-\delta_{ce})(-\delta_{eb})\\&=\delta_{ab}
	\end{aligned}
\end{equation}
这正是$SO(8)$ Clifford代数,其存在八维Weyl旋量表示和八维矢量表示,正好对于十维SYM的八个胶子自由度和八个胶微子自由度。
\section{*约束哈密顿体系量子化}
\label{sec:5.2}
本节简要介绍约束体系正则量子化方法,更多细节详见\cite{lcb,dirac}。
\subsection{经典约束系统}
我们这里所说的约束并非$f(q,t)=0$的情况,这称为完整约束,这种约束总可以通过引入独立的广义坐标消去。从体系拉格朗日量出发:
\begin{equation}
	L=L(q,\dot q),\quad p:=\frac{\partial L}{\partial \dot q}
\end{equation}
为了正则量子化需要勒让德变换到哈密顿量,如果利用$p:={\partial L}/{\partial \dot q}$可以完全反解出$\dot q = \dot q(q,p)$,那么就称为正规哈密顿体系,哈密顿量和运动方程为:
\begin{equation}
	H(q,p) := p\dot{q}-L(q,\dot{q}),\quad \{f,H\}_{\text{PB}} = \frac{\mathrm{d} f}{\mathrm{d} t}
\end{equation}
这种体系可以直接进行正则量子化:
\begin{equation}
	\label{canonical}
	f\mapsto \hat f,\quad \{\bullet,\bullet\}_{\text{PB}}\mapsto \frac{1}{i\hbar} [\bullet,\bullet]^{\pm}_{\text{DB}}
\end{equation}
但是如果不能完全反解出$\dot q$,那么这时就称为约束哈密顿体系,这对应下面的雅可比矩阵$\rank J := R<N$,$N$是体系自由度:
\begin{equation}
	J_{ij}:=\frac{\partial P_i}{\partial \dot{q}^i}:=\frac{\partial^2 L}{\partial \dot{q}^i\partial \dot{q}^j}
\end{equation}
这时$N$个$\dot{q}^i$中只有$R$个可以被反解为$\dot{q}^i=\dot{q}^i(q,p;\dot{q}^j)$,$dot{q}^j$表示剩下的$N-R$个广义坐标,剩下的信息以约束的形式体现为:
\begin{equation}
	\phi_m(q,p)=0,\quad m = 1,\ldots,N-R 
\end{equation}
但是在变分原理导出哈密顿正则方程时需要$q,p$独立,现在不独立了,所以正则方程失效,需要用拉格朗日乘子法表达为:
\begin{equation}
	\frac{\mathrm{d}f}{\mathrm{d}t} = \{f,H\}_{\text{PB}}+\{f,\phi_m\}_{\text{PB}}\lambda^m,\quad 	\phi_m(q,p)\approx0
\end{equation}
这里$\lambda^m$是待定的拉格朗日乘子,注意上面我们对约束使用了$\approx$,这其实意味着它们是弱方程,这是为了强调只在约束面上为$0$,但计算与$f$的泊松括号时涉及到求导,也就是切空间上运算,所以只有在计算完全体泊松括号后才能用约束条件。

取上式中$f=\phi_m$得到:
\begin{equation}
	\label{consist}
	0\approx\dot{\phi}_m=\{\phi_m,H\}_{\text{PB}}+\{\phi_m,\phi_n\}_{\text{PB}}\lambda^n
\end{equation}
第一个弱等号是因为在壳演化时,必须时时刻刻有$\phi_m = 0$,所以演化自洽性要求$\dot{\phi}_m\approx 0$。而上式是个自洽方程,给了$\lambda$约束,也暗含对$H$选取时的约束,约束哈密顿体系的$H$实则不唯一,我们假设已经做到了这一点,否则无论$\lambda^m(t)$如何选取都无法满足自洽方程:
\begin{equation}
	\Phi_{mn}:=\{\phi_m,\phi_n\}_{\text{PB}},\quad h_m:=-\{\phi_m,H\}_{\text{PB}}\Rightarrow \Phi_{mn}\lambda^n\approx h_m
\end{equation} 
如果$\Phi$可逆,则$\lambda^m$完全被约束确定,并非自由。更常见的情况是不可逆,我们考虑下面的极端情况:
\begin{equation}
	\label{consist2}
	\Phi_{mn}\overset{\Gamma_1}{\approx} 0,\quad h_{mn}\overset{\Gamma_1}{\approx} 0
\end{equation}
这里$\Gamma_1\Gamma$,表示体系相空间$\Gamma$被约束到一个子流形上。这时无论$\lambda^m(t)$是怎样的函数,自洽方程\ref{consist}始终被满足,任意选取一个$\lambda^m(t)$后代入\ref{eq:5.16}得到唯一解$f(t)$,不同的$\lambda^m$给出不同的相空间演化曲线,他们应当解释为描述同一个体系同一个演化,只是不同规范下的相空间轨迹,也就是说$\lambda^m$这时可以被解释为规范自由度。

但是\ref{consist2}完全可以要求$h_{mn}\overset{\Gamma_2\subset\Gamma_1}{\approx} 0$。这个时候自洽性意味着相空间被约束到更小的流形$\Gamma_2$,这相当于引入了新的约束,而且注意到由于自洽方程本身是从在壳运动方程得来的,所以这种约束是在壳后才来的约束,表明$\Gamma_1$上并非所有点都有演化曲线经过。这种约束我们称为次级约束,而原先的$\Phi_{mn}\overset{\Gamma_1}{\approx} 0$称为初级约束。次级约束又会产生类似\ref{consist}的新的约束条件,而约束条件又会带来新的次级约束,如此反复,最终相空间被约束在$\Gamma'\subset \Gamma$上。

但是在量子化时,我们并不需要深究初级约束和次级约束的区别,把他们统称为约束。但是如果某个约束与其它所有约束的泊松括号都(弱)为$0$,那么就称为第一类约束,否则称为第二类约束。
\subsection{量子化}
从\ref{consist2}的讨论大致清楚第一类约束对应规范自由度,比如$P^2=\frac12\{P,P\}\approx 0$在壳约束就是第一类约束,这类约束量子化并不复杂,正如协变量子化那里做的一样,他们无非是在正则量子化\ref{canonical}后,额外对态空间施加算符约束:
\begin{equation}
	\label{eq:1st}
	\hat{\phi}_m\ket{\text{phys}} = 0
\end{equation}
第二类约束其实意味着某些自由度在体系的描述中是无关紧要的,但是在经典理论中,定义泊松括号是需要对所有的自由度求导来定义的,所以我们需要修改经典理论的泊松括号,使得其自然丢掉那些无关紧要的自由度,Dirac发现下面的修正:
\begin{equation}
	\label{eq:2nd}
	\{f,g\}_{*}:=\{f,g\}_{\mathrm{PB}}-\{f,\chi_m\}_{\mathrm{PB}}C_{mn}^{-1}\{\chi_n,g\}_{\mathrm{PB}},\quad C_{mn}:=\{\chi_m,\chi_n\}_\text{PB}
\end{equation}
这里$\chi_m$表示那些第二类约束。而且可以证明上式依然满足泊松括号所满足的李代数结构。考虑上式下面的特殊情况:
\begin{equation}
\begin{aligned}
		\{f,\chi_k\}_{*} &= \{f,\chi_k\}_{\mathrm{PB}} -\{f,\chi_m\}_{\mathrm{PB}}C_{mn}^{-1}\{\chi_n,\chi_k\}_{\mathrm{PB}}\\
	&=\{f,\chi_k\}_{\mathrm{PB}} -\{f,\chi_k\}_{\mathrm{PB}}=0
\end{aligned}
\end{equation}
这意味着在新的泊松括号定义下,约束$\chi_m$不再是弱方程,而是在可以求出泊松括号前就认为是$0$的强方程:
\begin{equation}
	\label{2nd}
	\chi_m = 0
\end{equation}
至此,我们可以给出约束哈密顿体系的量子化步骤:
\begin{itemize}
	\item[1.] 首先对约束$\phi_m$进行线性组合,使得尽可能多的约束落入第一类;
	\item[2.] 利用剩下的第二类约束$\chi_m$,修改经典泊松括号为\ref{eq:2nd};
	\item[3.] 采用标准的正则量子化步骤\ref{canonical},但是用修改后的泊松括号$\{\bullet,\bullet\}_*$代替$\{\bullet,\bullet\}_{\mathrm{PB}}$;
	\item[4.] 第一类约束的弱方程看作是加在态空间上的辅助方程\ref{eq:1st},而第二类约束的强方程\ref{2nd}看作是算符需要满足的方程$\hat{\chi}_m =0$。
\end{itemize}
\section{纯旋量超弦}
\subsection{超对称点粒子作用量}
Berkovits发现,可以通过引入额外的一组费米的共轭变量$(\theta^\alpha,p_\alpha)$\footnote{这里的记号有点容易混淆,这里$\theta$和$p$与前文提到的毫无关系,完全是新的独立变量,用这个记号主要是为了后续与文献中常用的纯旋量超弦的记号接轨。}使约束变为第一类:
\begin{equation}
\begin{gathered}
		S=\int d\tau\left(\dot{X}^\mu P_\mu-\frac{1}{2}\dot{S}_aS_a+eP^\mu P_\mu+\dot{\theta}^\alpha p_\alpha+f^\alpha\hat{d}_\alpha\right)\\
	\hat{d}_\alpha:=d_\alpha+\frac{1}{\sqrt{P^+}}P_\mu(\gamma^\mu\gamma^+S)_\alpha,\quad d_\alpha:=p_\alpha+\frac{1}{2}P_\mu(\gamma^\mu\theta)_\alpha
\end{gathered}
\end{equation}
利用\ref{eq:5.10}不难发现$\hat{d}_\alpha$满足代数结构:
\begin{equation}
	\{\hat{d}_\alpha,\hat{d}_\beta\}=-\frac{1}{2P^+}P^2(\gamma^+)_{\alpha\beta}\approx 0
\end{equation}
所以确实是第一类约束,第一类约束对应体系有规范对称性,可以利用引入鬼场消去。上式中有$e$和$f$两个拉格朗日乘子,$e$对应diff对称性可以直接通过引入$bc$鬼场消去,剩下的$f$对应的规范对称性可以通过引入玻色的鬼场$\{\hat\lambda,\hat w\}$消去,他们是十维Weyl旋量:
\begin{equation}
	\label{eq:5.14}
	S=\int d\tau\left(\dot{X}^\mu P_\mu-\frac{1}{2}\dot{S}_aS_a-\frac{1}{2}P^\mu P_\mu+\dot{\theta}^\alpha p_\alpha+\dot{c}b-\dot{\hat{\lambda}}^\alpha\hat{w}_\alpha\right)
\end{equation}
纯旋量形式量子化最常用的方法是BRST量子化,上面作用量的BRST荷为:\cite{Berkovits:2002zk,Berkovits:2001rb}
\begin{equation}
	\hat{Q}_B=\hat{\lambda}^\alpha\hat{d}_\alpha+cP^2+\frac{i}{4P^+}b(\hat{\lambda}\gamma^+\hat{\lambda})
\end{equation}
可以证明这个复杂的BRST上同调等价于下面的BRST算符的上同调:\cite{Berkovits:2002zk}
\begin{equation}
	\label{eq:5.16}
	Q_B=\lambda^\alpha d_\alpha, \quad \lambda\gamma^\mu\lambda = 0
\end{equation}
$\lambda\gamma^\mu\lambda = 0$就是纯旋量条件。而且注意到,$Q_B$和$S_a,c$无关,也就是说最后量子化得到的结果与$bc$鬼场和$S_a$是否存在无关,所以\ref{eq:5.14}可以写作下面更加简单的形式:
\begin{equation}
	\label{eq:5.17}
	S_{\mathrm{PS}}=\int d\tau\left(\dot{X}^\mu P_\mu-\frac{1}{2}P^\mu P_\mu+\dot{\theta}^\alpha p_\alpha-\dot{\lambda}^\alpha w_\alpha\right)
\end{equation}
不同于RNS形式中的$bc$和$\beta\gamma$两套鬼场,纯旋量形式现在只剩下了一套$\lambda w$鬼场。剩下就是标准BRST量子化的方法,类似\ref{eq:4.75}将超多重态编码至一个超波函数中描述:
\begin{equation}
	\Omega(X,\theta,\lambda)=C(X,\theta)+\lambda^\alpha A_\alpha(X,\theta)+(\lambda\gamma^{\mu_1,\ldots,\mu_5}\lambda)A_{\mu_1,\ldots,\mu_5}^*(X,\theta)+\lambda^\alpha\lambda^\beta\lambda^\gamma C_{\alpha\beta\gamma}^*(X,\theta)+\cdots
\end{equation}
BRST算符作用在波函数上可以用正则量子化表示为算符:
\begin{equation}
	\label{eq:5.19}
	d_\alpha=p_\alpha+\frac{1}{2}P_\mu(\gamma^\mu\theta)_\alpha\to D_\alpha=\frac1i\frac{\partial}{\partial\theta^\alpha}+\frac{1}{2}(\gamma^\mu\theta)_\alpha\frac1i\frac{\partial}{\partial X^\mu}
\end{equation}
然后BRST闭给出超多重态中的胶子和胶微子波函数满足十维SYM运动方程,BRST恰当给出超规范对称性。
\subsection{超弦作用量}
利用$P^\mu$运动方程将\ref{eq:5.17}中的$P^\mu$提前积分掉得到:
\begin{equation}
	S_{\mathrm{PS}}=\int d\tau\left(\frac{1}{2}\dot{X}^\mu\dot{X}_\mu+\dot{\theta}^\alpha p_\alpha-\dot{\lambda}^\alpha w_\alpha\right)
\end{equation}
推广到弦只需要将世界线坐标扩充到世界面坐标,场算符因此扩充为左右模各一个:
\begin{equation}
	\begin{aligned}(\tau)&\to(z,\bar{z}),\\\{X(\tau),\theta(\tau),p(\tau),\lambda(\tau),w(\tau)\}&\to\{X(z,\overline{z}),\theta(z,\overline{z}),p(z,\overline{z}),\lambda(z,\overline{z}),w(z,\overline{z})\}\end{aligned}
\end{equation}
\begin{equation}
	\boxed{
	S_{\mathrm{PS}}=\frac{1}{2\pi\alpha^{\prime}}\int d^2z\left(\frac{1}{2}\partial X^\mu\overline{\partial}X_\mu+p_\alpha\overline{\partial}\theta^\alpha-w_\alpha\overline{\partial}\lambda^\alpha+\tilde{p}_\alpha\partial\tilde{\theta}^\alpha-\tilde{w}_\alpha\partial\tilde{\lambda}^\alpha\right)
}
\end{equation}
这里$\alpha$都是十维Weyl旋量指标,共$2^{\frac{10}{2}-1}=16$个分量,而左右模旋量的手征是否相同反映了II A/B型闭弦理论。上述体系有如下的靶空间超对称性:
\begin{equation}
	\delta\theta^\alpha=\epsilon^\alpha,\quad\delta x^\mu=\frac{1}{2}(\epsilon\gamma^\mu\theta)
\end{equation}
对于开弦,边界条件同样同RNS超弦一样,在实轴上左右模相等,所以利用加倍技巧之后对于开弦我们只需要关注上式中不含右模的部分:
\begin{equation}
	\label{eq:5.23}
	S_{\mathrm{PS}}=\frac{1}{\pi}\int d^2z\left(\frac{1}{2}\partial X^\mu\overline{\partial}X_\mu+p_\alpha\overline{\partial}\theta^\alpha-w_\alpha\overline{\partial}\lambda^\alpha\right)
\end{equation}
这里以及之后的计算中,我们都假设闭弦$\alpha' = 2$,开弦$\alpha'=\frac12$,利用下面的质量量纲可以很快补写出任何等式中的$\alpha'$:
\begin{equation}
	[\alpha^{\prime}]=2,\quad[X^\mu]=1,\quad[\theta^\alpha]=[\lambda^\alpha]=\frac{1}{2},\quad[p_\alpha]=[w_\alpha]=-\frac{1}{2}
\end{equation}
同样有\ref{eq:5.19}的定义:
\begin{equation}
	\Pi^\mu=\partial X^\mu+\frac{1}{2}(\theta\gamma^\mu\partial\theta),\quad d_\alpha=p_\alpha-\frac{1}{2}{\left(\partial X^\mu+\frac{1}{4}(\theta\gamma^\mu\partial\theta)\right)}(\gamma_\mu\theta)_\alpha
\end{equation}
附录\ref{appendix:A}中给了纯旋量形式计算中的OPE。与RNS超弦对比,鬼场对应$\lambda w$,世界面上旋量$\psi$改为了与靶空间坐标$X$对应的超靶空间坐标$\theta$。利用BRST量子化,BRST算符依旧有\ref{eq:5.16}的形式:
\begin{equation}
	Q_B:=\oint dz\left(\lambda^\alpha d_\alpha\right),\quad \lambda\gamma^\mu\lambda=0
\end{equation}
利用BRST量子化以及$Q_B U = \partial V$可以给出无质量多重态的积分顶角算符$U$和无积分顶角算符$V$:
\begin{equation}
	\label{eq:5.39}
	U(z)=\partial\theta^\alpha A_\alpha(X,\theta)+A_\mu(X,\theta)\Pi^\mu+d_\alpha W^\alpha(X,\theta)+\frac{1}{2}N_{\mu\nu}F^{\mu\nu}(X,\theta)
\end{equation}
\begin{equation}
	\label{eq:5.40}
	V=\lambda^\alpha A_\alpha(X,\theta)
\end{equation}
注意这里就不像RNS超弦一样要区分玻色和费米部分了,所以纯旋量超弦直接计算的就是超振幅。另外由于保证了靶空间超对称,所以也不需要GSO投影。这里$A,W,F$都是十维SYM的线性化超场。对于闭弦则有:\footnote{这里使用$\otimes$而不是$\times$是为了提醒平面波不需要双复制,比如$V(z,\overline{z})=\mathrm{e}^{ik\cdot X}\lambda^\alpha\overline{\lambda}^\beta A_\alpha(\theta)\overline{A}_\beta(\theta)$}
\begin{equation}
	V(z,\bar z) = V(z)\otimes \tilde V(\bar z),\quad U(z,\bar z) = U(z)\otimes \tilde U(\bar z)
\end{equation}


\subsection{十维超对称Yang-Mills理论}
本节对上一节最后提到的十维SYM超场特别是其展开式进行一些总结。十维SYM粒子谱中只有胶子$\mathbb{A}_\alpha=\mathbb{A}_\alpha(X,\theta)$和胶微子$\mathbb{A}_\mu=\mathbb{A}_\mu(X,\theta)$,定义超协变导数和超空间导数:
\begin{equation}
	\nabla_\alpha = D_\alpha - \mathbb{A}_\alpha, \quad
	\nabla_\mu = \partial_\mu - \mathbb{A}_\mu, \quad
	D_\alpha = \frac{\partial}{\partial\theta^\alpha} + \frac{1}{2} (\gamma^\mu \theta)_\alpha \partial_\mu
\end{equation}
十维SYM场的运动方程可以表达为:
\begin{equation}
	\label{eq:5.43}
	\begin{aligned}
		\left\{\nabla_\alpha, \nabla_\beta\right\} 
		&= \gamma_{\alpha\beta}^\mu \nabla_\mu, \quad 
		& \left\{\nabla_\alpha, \nabla_\mu\right\} 
		&= -(\gamma_\mu \mathbb{W})_\alpha, \\
		\left\{\nabla_\alpha, \mathbb{W}^\beta\right\} 
		&= \frac{1}{4} (\gamma^{\mu\nu})_\alpha{}^\beta \mathbb{F}_{\mu\nu}, 
		& \left[\nabla_\alpha, \mathbb{F}^{\mu\nu}\right] 
		&= (\mathbb{W}^{[\mu} \gamma^{\nu]})_\alpha
	\end{aligned}
\end{equation}
其中:
\begin{equation}
		\mathbb{F}_{\mu\nu} := -\left[\nabla_\mu, \nabla_\nu\right], \quad \mathbb{W}_\mu^\alpha := \left[\nabla_\mu, \mathbb{W}^\alpha\right]
\end{equation}
丢去上面公式的所有非线性部分得到线性化超场的运动方程:
\begin{equation}
	\label{eq:5.45}
\begin{aligned}
	D_\alpha A_\beta^i + D_\beta A_\alpha^i 
	&= \gamma_{\alpha\beta}^\mu A_\mu^i, \quad 
	& D_\alpha A_\mu^i 
	&= (\gamma_\mu W_i)_\alpha + \partial_\mu A_\alpha^i, \\
	D_\alpha W_i^\beta 
	&= \frac{1}{4} (\gamma^{\mu\nu})_\alpha{}^\beta F_{\mu\nu}^i, \quad 
	& D_\alpha F_{\mu\nu}^i 
	&= \partial_{[\mu} (\gamma_{\nu]} W_i)_\alpha
\end{aligned}
\end{equation}
这里指标$i$是用来标记外腿粒子,对SYM超场本身来说这无关紧要,但后面会推广到多个粒子对应的超场。选取Harnad–Shnider规范$\theta^\alpha A_{\alpha} = 0$,求解上面的线性化超场运动方程得到\footnote{这里我们使用$\chi$而不是前面所用的$u$表示胶微子波函数}:\cite{Policastro:2006vt}
\begin{equation}
	\begin{aligned}
		&A_\alpha^{(n)} = \frac{1}{n+1} (\gamma^\mu \theta)_\alpha A_\mu^{(n-1)}, \\
		&A_\mu^{(n)} = \frac{1}{n} (\theta \gamma_\mu W^{(n-1)}), \\
		&W^{\alpha(n)} = -\frac{1}{2n} (\gamma^{\mu\nu} \theta)^\alpha \partial_\mu A_\nu^{(n-1)}
	\end{aligned}
\end{equation}
这里$K=\sum_n K^{(n)}$,比如:
\begin{equation}
	A_i^\mu(X,\theta)=\left\{{(\cosh\sqrt{O})^\mu}_\nu e_i^\nu+{\left(\frac{\sinh\sqrt{O}}{\sqrt{O}}\right)^m}_\nu(\theta\gamma^\nu\chi_i)\right\}e^{i k_i\cdot X},\quad {\mathcal{O}^\mu}_\nu=\frac{i}{2}(\theta{\gamma^\mu}_{\nu\rho}\theta)k_i^\rho
\end{equation}
对于纯旋量超弦的计算,只需要知道$\mathcal{O}(\theta^{n\leq 4})$的项,显式写出为:
\begin{equation}
	\label{eq:5.48}
\begin{aligned}
		A_\alpha^i&(X,\theta)=\left\{\frac{1}{2}(\theta\gamma_\mu)_\alpha e_i^\mu+\frac{1}{3}(\theta\gamma_\mu)_\alpha(\theta\gamma^\mu\chi_i)-\frac{i}{32}(\theta\gamma^\mu)^\alpha(\theta\gamma_{\mu\nu\rho}\theta)f_i^{\nu\rho}\right.\\
	&\left.+\frac{i}{60}(\theta\gamma^\mu)_\alpha(\theta\gamma_{\mu\nu\rho}\theta)k_i^\nu(\chi_i\gamma^\rho\theta)-\frac{1}{1152}(\theta\gamma^\mu)_\alpha(\theta\gamma_{\mu\nu\rho}\theta)(\theta{\gamma^\rho}_{\sigma\tau}\theta)k_i^\nu f_i^{\sigma\tau}+\cdots\right\}e^{ik_i\cdot X}
\end{aligned}
\end{equation}
\begin{equation}
	\label{eq:5.49}
	\begin{aligned}
		A_i^\mu&(X,\theta)=\left\{e_i^\mu+(\theta\gamma^\mu\chi_i)-\frac{i}{8}(\theta{\gamma^\mu}_{\nu\rho}\theta)f_i^{\nu\rho}+\frac{i}{12}(\theta{\gamma^\mu}_{\nu\rho}\theta)k_i^\nu(\chi_i\gamma^\rho\theta)\right.\\
		&\left.-\frac{1}{192}(\theta{\gamma^\mu}_{\nu\sigma}\theta)(\theta{\gamma^\sigma}_{\rho\tau}\theta)k_i^\nu f_i^{\rho\tau}+\frac{1}{480}(\theta{\gamma^\mu}_{\nu\sigma}\theta)(\theta{\gamma^\sigma}_{\rho\tau}\theta)k_i^\nu k_i^\rho(\chi_i\gamma^\tau\theta)+\cdots\right\}e^{ik_i\cdot X}
	\end{aligned}
\end{equation}
\begin{equation}
	\label{eq:5.50}
	\begin{aligned}
		&W_i^\alpha (X, \theta) =
		\left\{
		\chi_i^\alpha
		+ \frac{i}{4} (\theta \gamma_{\mu\nu})^\alpha f_i^{\mu\nu}
		- \frac{i}{4} (\theta \gamma_{\mu\nu})^\alpha k_i^\mu (\chi_i \gamma^\nu \theta)+\frac{1}{48}(\theta{\gamma_\mu}^\sigma)^\alpha(\theta\gamma_{\sigma\nu\rho}\theta)k^\mu_i f^{\nu\rho}_i\right.\\
		& \left.- \frac{1}{96} (\theta \gamma_\mu^{\phantom{\mu} \sigma})^\alpha (\theta \gamma_{\sigma\nu\rho} \theta) k_i^\mu k_i^\nu (\chi_i \gamma^\rho \theta)
		+ \frac{i}{1920} (\theta \gamma_\mu^{\phantom{\mu} \tau})^\alpha (\theta \gamma_{\nu\tau}^{\phantom{\nu\tau} \sigma} \theta) (\theta \gamma_{\sigma\rho\kappa} \theta) k_i^\mu k_i^\nu f_i^{\rho\kappa}
		+ \cdots
		\right\} e^{ik_i \cdot X}
	\end{aligned}
\end{equation}
\begin{equation}
	\label{eq:5.51}
	\begin{aligned}
		F_i^{\mu\nu} &(X, \theta) = i
		\left\{
		f_i^{\mu\nu}
		- k_i^{[\mu} (\chi_i \gamma^{\nu]} \theta)
		- \frac{1}{8} (\theta \gamma_{\rho\sigma}^{\phantom{\rho\sigma} [\mu} \theta) k_i^{\nu]} k_i^\rho f_i^{\rho\sigma}
		+ \frac{1}{12} (\theta \gamma_{\rho\sigma}^{\phantom{\rho\sigma} [\mu} \theta) k_i^{\nu]} k_i^\rho k_i^\sigma (\chi_i \gamma^\sigma \theta)\right.\\
		&\left.+\frac{1}{192}(\theta{\gamma_{\rho\phi}}^{[\mu}\theta)k_i^{\nu]}k_i^\rho f_i^{\sigma\tau}(\theta{\gamma^\phi}_{\sigma\tau}\theta)
		- \frac{1}{480} (\theta \gamma_{\rho\phi}^{\phantom{\rho\phi} [\mu} \theta) k_i^{\nu]} k_i^\rho k_i^\sigma(\chi_i\gamma^\tau\theta) (\theta \gamma_{\sigma\tau}^{\phantom{\sigma\tau} \phi} \theta)
		+ \cdots
		\right\} e^{ik_i \cdot X}
	\end{aligned}
\end{equation}
其中定义:
\begin{equation}
	f_i^{\mu\nu}:=k_i^\mu e_i^\nu-k_i^\nu e_i^\mu
\end{equation}
这些线性超场有如下的质量量纲:
\begin{equation}
	[A_\alpha]=\frac{1}{2},\quad[A_\mu]=0,\quad[W^\alpha]=-\frac{1}{2},\quad[F_{\mu\nu}]=-1
\end{equation}

这些线性超场的作用就相当于Yang-Mills理论中极化矢量的作用,费米子计算中费米波函数$u$的作用,从他们展开的最低阶也确实能看出这一点。
\subsection{*$U(5)$分解}
\label{sec:u5}
纯旋量约束导致纯旋量表述下的OPE不是自由的,比如:\footnote{关于式子中每项的解释请见\cite{Berkovits:2000fe}。}
\begin{equation}
	\label{eq:5.55}
	w_\alpha(y)\lambda^\beta(z)\sim(y-z)^{-1}\delta_\alpha^\beta-\frac{1}{2}(y-z)^{-1}\gamma_m^{\beta+}\xi e^{-\phi}(\gamma^m\lambda)_\alpha
\end{equation}
前面纯旋量的约束条件之间其实不是独立的,$U(5)$分解可以完全提取$\lambda^\alpha$的相互之间独立的分量,从而解掉纯旋量约束得到自由的CFT。虽然这么做也破坏了协变性,但最后得到的OPE总可以整合成协变形式。这里我们对$U(5)$分解本身做一些介绍,附录\ref{appendix:A}中我们会利用$U(5)$分解构造OPE。

OPE中涉及到的那些场都是$SO(1,9)$的矢量表示或旋量表示,假设我们做了Wick转动到$SO(10)$\footnote{这也是不少文献喜欢将$\eta^{\mu\nu}$写作$\delta^{mn}$的原因。},$U(5)\cong SU(5)\otimes U(1)$作为$SO(10)$的子群,我们希望计算$SO(10)$表示在其子群$U(5)$上的分导表示。在粒子物理唯象学上的大统一理论(Grand Unified Theory)其实就涉及到这种分解:\cite{Georgi:2000vve,Zee:2016fuk}
\begin{equation}
	E_8\to SO(16)\to SO(10)\to SU(5)\to SU(3)\otimes SU(2)\otimes U(1)
\end{equation}

分导表示的计算可以通过\texttt{Mathematica}程序包\texttt{LieART}\cite{FEGER2020107490}进行,比如一阶二阶张量分解:
\begin{equation}
	\begin{aligned}
		V^{m}&\to\upsilon^{a}\oplus\upsilon_{a}\quad&M^{mn}&\to m^{ab}\oplus m_{ab}\oplus m_{b}^{a}\oplus m\mathrm{~,}\\
		\mathbf{10}&\to\mathbf{5}_{-1}\oplus\mathbf{\overline{5}}_{1}\quad&\mathbf{45}&\to\mathbf{10}_{-2}\oplus\mathbf{\overline{10}}_{2}\oplus\mathbf{24}_{0}\oplus\mathbf{1}_{0}
		\end{aligned}
\end{equation}
这可以通过选取Cartan-Weyl基底具体实现:\cite{Nekrasov:2005wg}
\begin{equation}
	v^a=\frac{1}{\sqrt{2}}\left(V^a+iV^{a+5}\right),\quad v_a=\frac{1}{\sqrt{2}}\left(V^a-iV^{a+5}\right),\quad a=1,\ldots,5
\end{equation}
\begin{equation}
	\begin{aligned}
		m^{ab}&=\frac{1}{2}\left(M^{ab}+iM^{a(b+5)}+iM^{(a+5)b}-M^{(a+5)(b+5)}\right)\\
		m_{ab}&=\frac{1}{2}\left(M^{ab}-iM^{a(b+5)}-iM^{(a+5)b}-M^{(a+5)(b+5)}\right)\\
		m_b^a&=\frac{1}{2}\left(M^{ab}-iM^{a(b+5)}+iM^{(a+5)b}+M^{(a+5)(b+5)}\right)\\
		m&=\sum_{a=1}^5m_a^a=i\sum_{a=1}^5M^{(a+5)a}
	\end{aligned}
\end{equation}
对于旋量表示,这里考虑十维Weyl旋量,取:
\begin{equation}
	b^a=\frac{1}{2}\left(\Gamma^a+i\Gamma^{a+5}\right),\quad b_a=\frac{1}{2}\left(\Gamma^a-i\Gamma^{a+5}\right)
\end{equation}
在Cliford代数下$b$有下面的海森堡代数:
\begin{equation}
	\{b_a,b^b\}=\delta_a^b,\quad\{b_a,b_b\}=\{b^a,b^b\}=0
\end{equation}
那么左手($\Gamma_{11}=-1$)Weyl旋量$\ket{\lambda}$和右手($\Gamma_{11}=+1$)Weyl旋量$\ket{\omega}$可以用下面的式子生成:
\begin{equation}
	\begin{aligned}
		\left|\lambda\right\rangle&=\lambda^+|0\rangle+\frac{1}{2}\lambda_{ab}b^bb^a|0\rangle+\frac{1}{4!}\lambda^a\epsilon_{abcde}b^eb^db^cb^b|0\rangle,\\\left|{\omega}\right\rangle&=\frac{1}{5!}\omega_+\varepsilon_{abcde}b^ab^bb^cb^db^e\ket{0}+\frac{1}{2!3!}\omega^{ab}\varepsilon_{abcde}b^cb^db^e\ket{0}+\omega_ab^a\ket{0}
	\end{aligned}
\end{equation}
\begin{equation}
	\begin{aligned}
		\lambda^\alpha&\to(\lambda^+,\lambda_{ab},\lambda^a),\quad&\omega_{\dot\alpha}\to(\omega_+,\omega^{ab},\omega_a)\\\mathbf{16}&\to(\mathbf{1},\mathbf{\overline{10}},\mathbf{5}),\quad&\mathbf{16}^{\prime}\to(\mathbf{1},\mathbf{10},\mathbf{\overline{5}})
	\end{aligned}
\end{equation}
$U(5)$分解下荷共轭矩阵以及$\Gamma_{11}$可以表达为:
\begin{equation}
	\mathcal{C}=\prod_{i=1}^{5}\Gamma_i=\prod_{a=1}^5(b_a+b^a),\quad\Gamma_{11}=\prod_{a=1}^5(b^ab_a-b_ab^a)
\end{equation}
左手Weyl旋量的纯旋量约束$\lambda \gamma^\mu \lambda = 0$是在$16\times 16$的Weyl基底下写的,补写成$32\times 32$的Dirac旋量有形式:
\begin{equation}
	\label{eq:diracps}
	\Lambda^T\mathcal{C}\Gamma^m\Lambda=0,\quad \Gamma_{11}\Lambda=-\Lambda, \quad \Lambda = (\lambda , 0)^{\mathrm{T}}
\end{equation}
上式在$U(5)$分解的记号下可以表达为:
\begin{equation}
	\label{eq:5.64}
	\llangle\lambda|\mathcal{C}b^a|\lambda\rangle=0,\quad\llangle\lambda|\mathcal{C}b_a|\lambda\rangle=0,\quad a=1,\ldots 5
\end{equation}
注意,这里$\bra{\lambda}:=\ket{\lambda}^\dagger$,$\llangle{\lambda}|:=\ket{\lambda}^T$,利用$\Gamma$矩阵的下列性质:
\begin{equation}
	\Gamma_m^T=\begin{cases}-\Gamma_m,&m=1,\ldots,5\\+\Gamma_m,&m=6,\ldots,10\end{cases},\quad \Gamma_m^\dagger =\Gamma_m,\quad \mathcal{C}\Gamma_m=-\Gamma_m^T\mathcal{C}
\end{equation}
对应得到$b^a$的性质:
\begin{equation}
\begin{gathered}
		b_a^\dagger=b^a,\quad(b^a)^\dagger=b_a,\quad b_a^T=-b^a,\quad(b^a)^T=-b_a,\quad \mathcal{C}b_a=b^a\mathcal{C},\quad \mathcal{C}b^a=b_a\mathcal{C}\\
	\langle0|\mathcal{C}b^ab^bb^cb^db^e|0\rangle=\epsilon^{abcde}\Rightarrow\langle0|\mathcal{C}b^a|\lambda\rangle=\lambda^a,\cdots
\end{gathered}
\end{equation}
所以:\footnote{$\llangle{0}|=\bra{0}$}
\begin{equation}
	\llangle\lambda|=\langle0|\lambda^++\frac{1}{2}\langle0|b_ab_b\lambda_{ab}+\frac{1}{24}\langle0|b_bb_cb_db_e\lambda^a\epsilon_{abcde}
\end{equation}
带入到\ref{eq:5.64}纯旋量条件后计算得到:
\begin{equation}
	\label{eq:5.69}
	2\lambda^+\lambda^a-\frac{1}{4}\epsilon^{abcde}\lambda_{bc}\lambda_{de}=0,\quad \langle0|\mathcal{C}b_b|0\rangle=2\lambda^a\lambda_{ab}
\end{equation}
注意第二个约束由第一个约束隐含,所以看似纯旋量约束有$10$个分量,但在$U(5)$分解下看到其独立分量只有五个,所以纯旋量有$11$个自由度。
\section{纯旋量超弦振幅}
\label{sec:5.4}
选取规范固定$(z_1,z_{n-1},z_n)\to(0,1,\infty)\text{ or }(1,0,\infty)$,类似在$\S$\ref{sec:4.3}做的,上面两种选取分别对应外腿奇偶两种排序,使得盘面顺序自洽。纯旋量形式下的盘面色序超振幅形式为:
\begin{equation}
	\label{eq:5.70}
	A_n(P)=\int_{\partial D}dz_2\mathrm{~}dz_3\ldots dz_{n-2}\llangle V_1(z_1)U_2(z_2)U_3(z_3)\ldots U_{n-2}(z_{n-2})V_{n-1}(z_{n-1})V_n(z_n)\rrangle
\end{equation}
这里$\llangle\bullet\rrangle$表示需要零模和非零模两重计算。Type II闭弦球面超弦振幅的形式完全类似。下面的OPE说明$\{ \partial\theta^\alpha, \Pi^\mu,d_\alpha, N^{\mu\nu} \}$都是共形权为$1$的初级场:
\begin{equation}
	T_{\mathrm{PS}}(z) \left\{ \partial\theta^{\alpha}, \Pi^{\mu}, d_{\alpha}, N^{\mu\nu} \right\}(w) \sim \frac{\left\{ \partial\theta^{\alpha}, \Pi^{\mu}, d_{\alpha}, N^{\mu\nu} \right\}(w)}{(z-w)^{2}} + \frac{\partial \left\{ \partial\theta^{\alpha}, \Pi^{\mu}, d_{\alpha}, N^{\mu\nu} \right\}(w)}{z-w}
\end{equation}
它们在球面上没有零模,所以非零模的计算可以完全由OPE进行,关联函数完全由OPE缩并得来的奇异性决定\cite{Berkovits:2004px},不过这里涉及到OPE之后不是常数(非自由)的情况,所以缩并时要小心一些,这里给一个四点球面振幅的计算例子,\ref{eq:5.39}中有四项,第一项由于$V$中没有$d_\alpha$项和$\partial\theta$缩并给出非平凡OPE所以为$0$,第二项涉及到下面OPE:\footnote{为了后面表达式书写的方便,这里暂时取$z_{1,2,3}$的规范固定。}
\begin{equation}
	\begin{aligned}
		&\left\langle :A_\mu^4(\theta)\Pi^\mu(z_4)\mathrm{e}^{ik_4\cdot X(z_4,\overline{z}_4)}:\prod_{j=1}^3:(\lambda A^j(\theta))\mathrm{e}^{ik_4\cdot X(z_j,\overline{z}_j)}:\right\rangle\\
		=&\sum_{j=1}^3\left\langle(\lambda A^1(\theta))(\lambda A^2(\theta))(\lambda A^3(\theta))A_\mu^4(\theta)\wick{\c\Pi^\mu(z_4)\c{:\mathrm{e}^{ik_j\cdot X_j}:}}\times \text{other plan waves}\right\rangle\\
		=& \sum_{j=1}^3\frac{ik_j^\mu}{z_j-z_4}\langle(\lambda A^1)(\lambda A^2)(\lambda A^3)A_\mu^4\rangle
	\end{aligned}
\end{equation}
这里由于$\Pi$和超场缩并涉及到的是普通导数,所以我们将$A(X,\theta)$的平面波部分单独提取出来和$\Pi$缩并,剩下的部分记作$A(\theta)$。上式无非是在用OPE计算$z_4\to z_1,z_2,z_3$的奇异行为,然后把他们全部加起来。事实上这就是对非零模积分的过程。奇异性就足以确定整个关联函数的非零模部分。第三项由于OPE涉及超导数,所以不能提取平面波因子:
\begin{equation}
	\begin{aligned}
		&\left\langle(\lambda A^1)(\lambda A^2)(\lambda A^3):d_\alpha W_4^\alpha:(z_4)\right\rangle\\
		=&\wick{\left\langle(\lambda \c A^1)(\lambda A^2)(\lambda A^3): \c d_\alpha W_4^\alpha:\right\rangle}+\wick{\left\langle(\lambda  A^1)(\lambda\c A^2)(\lambda A^3): \c d_\alpha W_4^\alpha:\right\rangle}\\
		&+\wick{\left\langle(\lambda  A^1)(\lambda A^2)(\lambda\c A^3):\c d_\alpha  W_4^\alpha:\right\rangle}\\
		=&\frac{1}{z_1-z_4}\langle D_\alpha(\lambda A^1)(\lambda A^2)(\lambda A^3)W_4^\alpha\rangle-(1\leftrightarrow2)+(1\leftrightarrow3)
	\end{aligned}
\end{equation}
注意负号来源于$d_\alpha$和$A_\alpha$的费米性。同理,第四项为:
\begin{equation}
	\label{eq:5.74}
	\begin{aligned}
		\frac{1}{2}\big\langle(\lambda A^1)(\lambda A^2)(\lambda A^3)(N^{\mu\nu}F_{\mu\nu}^4)\big\rangle
		= \frac{1}{4(z_1-z_4)}\big\langle(\lambda\gamma^{\mu\nu}A^1)(\lambda A^2)(\lambda A^3)F_{\mu\nu}^4\big\rangle + \text{perms}
	\end{aligned}
\end{equation}
注意到线性超场运动方程给出$D_\alpha(\lambda A)=-(\lambda D)A_\alpha+(\lambda\gamma^\mu)_\alpha A_\mu$:
\begin{equation}
	\begin{aligned}
		\left\langle D_\alpha(\lambda A^1)(\lambda A^2)(\lambda A^3)W_4^\alpha\right\rangle=-\langle(\lambda DA_\alpha^1)(\lambda A^2)(\lambda A^3)W_4^\alpha\rangle+\langle A_\mu^1(\lambda A^2)(\lambda A^3)(\lambda\gamma^\mu W^4)\rangle
	\end{aligned}
\end{equation}
接下来就是非常有技巧性的化简,可以证明上式中的第一项与$\ref{eq:5.74}$之间只相差包含BRST恰当项的关联函数,所以自然差值就是$0$。这种化简技巧对一般盘面振幅的构造是极其重要的,下一章我们会系统地处理这些OPE计算。合并起来,并考虑到左模的贡献,最终得到:
\begin{equation}
	\begin{aligned}
		M_4=\int_\mathbb{C} d^2z_4\left(\frac{\mathcal{F}_{12}}{z_4}+\frac{\mathcal{F}_{21}}{1-z_4}\right)\left(\frac{\overline{\mathcal{F}}_{12}}{\overline{z}_4}+\frac{\overline{\mathcal{F}}_{21}}{1-\overline{z}_4}\right)|z_4|^{-\frac{1}{2}\alpha^{\prime}t}|1-z_4|^{-\frac{1}{2}\alpha^{\prime}u}
	\end{aligned}
\end{equation}
其中:
\begin{equation}
	\mathcal{F}_{12} := i k_1^\mu \big\langle (\lambda A^1(\theta))(\lambda A^2(\theta))(\lambda A^3(\theta)) A_\mu^4(\theta) \big\rangle + \big\langle A_\mu^1(\theta) (\lambda A^2(\theta))(\lambda A^3(\theta))(\lambda \gamma^\mu W^4(\theta)) \big\rangle
\end{equation}
$\mathcal{F}_{21}$由上式$1\leftrightarrow 2$得到。在对非零模积分之后,剩下的关联函数是零模积分,这些关联函数是和世界面坐标无关的,所有世界面坐标依赖都已经利用OPE进行非零模积分给出了。所以$\mathcal{F}$都不包含世界面坐标依赖,积分后得到:
\begin{equation}
		M_4=-2\pi K_0\overline{K}_0\frac{\Gamma(-\frac{\alpha^{\prime}t}{4})\Gamma(-\frac{\alpha^{\prime}u}{4})\Gamma(-\frac{\alpha^{\prime}s}{4})}{\Gamma(1+\frac{\alpha^{\prime}t}{4})\Gamma(1+\frac{\alpha^{\prime}u}{4})\Gamma(1+\frac{\alpha^{\prime}s}{4})},\quad 
		K_0:=\frac{1}{2}(uF_{12}+tF_{21})
\end{equation}
剩下的是鬼场零模积分。首先利用超场$K(\theta)$的展开式\ref{eq:5.48},\ref{eq:5.49},\ref{eq:5.50}和\ref{eq:5.51},最终的形式为$\sum_{n}\langle{\lambda^3\theta^m}\rangle$\footnote{只有三个固定的无积分顶角算符能贡献$\lambda$。}。$\lambda w$鬼场同样有$U(1)$对称性生成鬼场流:
\begin{equation}
	J_{\text{PS}}(z):=:\lambda^\alpha w_{\alpha}:(z)
\end{equation}
$\lambda$鬼数为$+1$,$w$鬼数为$-1$。下面的OPE表明其只是个准初级场,存在共形反常:
\begin{equation}
	\begin{aligned}
		T_{\text{PS}}(z)J_{\text{PS}}(y)\sim-\frac{8}{(z-y)^3}+\frac{1}{(z-y)^2}J_{\text{PS}}(y)+\frac{1}{(z-y)}\partial J_{\text{PS}}(y)
	\end{aligned}
\end{equation}
与前面接触过的$bc$鬼场和$\beta\gamma$鬼场完全类似,路径积分应当插入鬼数$+8$来平衡背景鬼数。$h(w_\alpha) = +1$故$w_\alpha$路径积分中不包含零模:
\begin{equation}
	[\mathcal{D}\lambda][\mathcal{D}\omega]\to[d\lambda_0^\alpha]\prod_{i=1}[d\lambda_i^\alpha][d\omega_\beta^i]
\end{equation}
后面的非零模鬼数贡献抵消,也就是说$[d\lambda_0^\alpha]$贡献鬼数$+8$。由于纯旋量空间是$16$维复流形的$11$维子流形,在纯旋量空间中积分应当构造与下面的顶微分形式成正比的协变$11$形式$[\mathcal{D}\lambda]$:
\begin{equation}
	\epsilon_{\alpha_1...\alpha_5\beta_1...\beta_{11}}d\lambda^{\beta_1}\wedge\cdots\wedge d\lambda^{\beta_{11}}:=[d\lambda^{11}]
\end{equation}
其鬼数为$11$,但三个无积分顶角算符贡献的$\lambda^3$和$[d\lambda_0^\alpha]$合在一起正好贡献$11$的鬼数,再加上旋量指标全反对称的要求,以及纯旋量约束$d(\lambda\gamma^{m}\lambda)=2\lambda\gamma^{m}d\lambda=0$给出的要求$\lambda^\beta\gamma_{\beta\alpha_i}^{\boldsymbol{m}}(d^{11}\lambda)^{[\alpha_1...\alpha_{11}]}=0$,导出下面的测度定义:
\begin{equation}
	[d\lambda^\alpha](\lambda\gamma^{\mu_1})_{\alpha_1}(\lambda\gamma^{\mu_2})_{\alpha_2}(\lambda\gamma^{\mu_3})_{\alpha_3}(\gamma_{\mu_1\mu_2\mu_3})_{\alpha_4\alpha_5}=\epsilon_{\alpha_1...\alpha_5\beta_1...\beta_{11}}d\lambda^{\beta_1}\wedge\cdots\wedge d\lambda^{\beta_{11}}
\end{equation}
将剩下的旋量指标与$\theta$缩并,给出零模关联函数计算测度:
\begin{equation}
	\label{eq:5.84}
	\boxed{
	\left\langle(\lambda^3\theta^5)\right\rangle:=\left\langle((\lambda\gamma^\mu\theta)(\lambda\gamma^\nu\theta)(\lambda\gamma^\sigma\theta)(\theta\gamma_{\mu\nu\sigma}\theta)\right\rangle = 2880
	}
\end{equation}
这里的$2880$完全只是一个人为约定,如此奇怪是因为代入后刚好和RNS超弦计算结果一致。这个式子的作用其实就相当于$bc$鬼场\ref{eq:bc_norm}的作用。可以证明\ref{eq:5.84}BRST闭且不恰当,而且是所有$\lambda^3\theta^5$形式能构造出来的唯一$SO(10)$标量,也就是说$\lambda^3\theta^5$最终都能约化为和\ref{eq:5.84}成正比,比如:
\begin{equation}
	\label{eq:5.85}
	\begin{aligned}
		&\langle(\lambda\gamma^m\theta)(\lambda\gamma^s\theta)(\lambda\gamma^u\theta)(\theta\gamma_{fgh}\theta)\rangle=24\delta_{fgh}^{msu},\\&\langle(\lambda\gamma_m\theta)(\lambda\gamma_s\theta)(\lambda\gamma^{ptu}\theta)(\theta\gamma_{fgh}\theta)\rangle=\frac{288}{7}\delta_{[m}^{[p}\eta_{s][f}\delta_g^t\delta_{h]}^{u]}
	\end{aligned}
\end{equation}
附录\ref{appendix:B}中给出了更多例子。利用\texttt{FORM}计算软件可实现自动化计算。\cite{Mafra:2010pn}

这里对纯旋量空间协变积分测度的定义采用非常简略的启发式的导出方法,对于后面树图计算而言已足够,更加详细的协变积分测度构造方法,特别是在圈图计算上的应用可参考文献\cite{gzq,Berkovits:2004px,GomezZuniga:2011soo,Gomez:2009qd,Berkovits:2004bw}。

最后,再来看一个三点超振幅计算的例子。只有无积分顶角算符插入,不需要计算OPE,只需要展开超场算零模就好了。由于纯旋量超弦直接计算的就是超振幅,但是从超场展开式可以看出$A_\alpha$的$\theta^{2k}$分量代表胶微子,$\theta^{2k-1}$分量代表胶子,通过这种方式便可以提取分量振幅\ref{eq:4.71}和\ref{eq:4.74}。比如三胶子振幅有三项:\footnote{由于三点特殊的运动学性质,不难发现Koba-Nielsen因子为$1$。}
\begin{equation}
		A_{3}(\epsilon_1,\epsilon_2,\epsilon_3) \sim -\frac{1}{64} \left( k_{\mu}^{3} \epsilon_{\sigma}^{1} \epsilon_{\tau}^{2} \epsilon_{\nu}^{3} - k_{\mu}^{2} \epsilon_{\sigma}^{1} \epsilon_{\nu}^{2} \epsilon_{\tau}^{3} + k_{\mu}^{1} \epsilon_{\nu}^{1} \epsilon_{\sigma}^{2} \epsilon_{\tau}^{3} \right) \big\langle (\lambda \gamma^{\sigma} \theta)(\lambda \gamma^{\tau} \theta)(\lambda \gamma_{\rho} \theta)(\theta \gamma^{\rho\mu\nu} \theta) \big\rangle
\end{equation}
比如第一项就来源于$A^1_\alpha$和$A^2_\alpha$贡献$\theta^1$,$A^3_\alpha$贡献$\theta^3$。再利用\ref{eq:5.85}即可得到\ref{eq:4.71}。同理\ref{eq:4.74}一个胶子两个胶微子振幅也可以用类似方法计算,其来源$A^1_\alpha$贡献$\theta^1$,$A^2_\alpha$和$A^3_\alpha$贡献$\theta^2$:\footnote{似乎少了电荷共轭矩阵,但这只是基底选取的不同。}
\begin{equation}
	\begin{aligned}
		A_{3}(\epsilon_1,u_2,u_3) 
		\sim -10 \epsilon_{\nu_1}^1 (u^2 \gamma^\sigma u^3) \big\langle (\lambda \gamma^{\nu} \theta)(\lambda \gamma^{\mu} \theta)(\lambda \gamma^{\rho} \theta)(\theta \gamma_{\mu\sigma\rho} \theta) \big\rangle 
		= \epsilon_\mu^1 (u^2 \gamma^\mu u^3)
	\end{aligned}
\end{equation}
从树图就能看出纯旋量形式计算振幅要麻烦不少,但这只是暂时的,因为RNS形式计算圈级振幅格外复杂,比如D'Hoker和Phong的一系列工作\cite{DHoker:2001kkt,DHoker:2001qqx,DHoker:2001foj,DHoker:2001jaf,DHoker:2005dys,DHoker:2005vch,DHoker:2002hof},相较而言纯旋量形式就轻松不少\cite{Berkovits:2005df,Berkovits:2005ng}。

在下一章我们会系统处理盘面振幅的计算,而闭弦球面振幅可以用KLT关系确定,所以原则上我们已经完全知晓树级Type I/II弦论(无质量态)的振幅计算\footnote{前文$\S$\ref{sec:4.3}最后也提到了开闭弦混合振幅可以用纯开弦振幅表达。}。弦论中还有两类自洽的闭弦理论,$SO(32)$和$E_8\times E_8$杂交弦理论\cite{Gross:1984dd,Gross:1985fr,Gross:1985rr}。这种弦理论是左行模的RNS超弦与右行模的玻色弦混合而成的弦理论\footnote{由于玻色弦定义在$26$维时空,所以构造杂交弦时还需要额外引入一些右行世界面旋量使得在$10$维时空便能抵消掉右行玻色弦的中心荷。}。由于杂交弦振幅计算并不是本文重点,所以这里并不对杂交弦进行完整介绍,只是说明纯旋量超弦依旧可以用于计算杂交弦振幅,只需要对左行模使用纯旋量描述,对右行模不加改变仍采用玻色弦的描述即可,无质量态顶角算符为:
\begin{equation}
	V_i^{\mathrm{het}}=\lambda^\alpha A_\alpha^i(\theta)e^{k_i\cdot X}\times\tilde{c}\cdot
	\begin{cases}
		\overline{\mathcal{J}}^{q_i},&\text{gauge multiplets}\\
		\sqrt{\frac2{\alpha'}}\tilde\epsilon_i^m\bar{\partial}X_m,&\text{garvity multiplys}
	\end{cases}
\end{equation}
\begin{equation}
	U_i^{\mathrm{het}}=\left(\partial\theta^\alpha A_\alpha^i(\theta)+\Pi_pA_i^p(\theta)+d_\alpha W_i^\alpha(\theta)+\frac{1}{2}N_{pq}F_i^{pq}(\theta)\right)e^{k_i\cdot X}\times
	\tilde c\cdot
	\begin{cases}
	\overline{\mathcal{J}}^{q_i}\\
	\sqrt{\frac2{\alpha'}}\tilde\epsilon_i^m\bar{\partial}X_m
	\end{cases}
\end{equation}
这里极化矢量有$\epsilon\dot k =0$,$\mathcal{J}$是Kac-Moody流代数生成元,有OPE:
\begin{equation}
	\overline{\mathcal{J}}^a(z)\overline{\mathcal{J}}^b(w)\sim\frac{\delta^{ab}}{(\overline{z}-\overline{w})^2}+\frac{f^{abc}\overline{\mathcal{J}}^c(w)}{\overline{z}-\overline{w}}
\end{equation}
类似\ref{eq:5.70}可以写下杂化弦振幅:
\begin{equation}
	\mathcal{M}_{n}^{\mathrm{het}}\sim\int_{\mathbb{CP}^{1}}d^{2}z_{2}d^{2}z_{3}\ldots d^{2}z_{n-2}\llangle(V_{1}^{\mathrm{het}}(z_{1})U_{2}^{\mathrm{het}}(z_{2})...U_{n-2}^{\mathrm{het}}(z_{n-2})V_{n-1}^{\mathrm{het}}(z_{n-1})V_{n}^{\mathrm{het}}(z_{n})\rrangle
\end{equation}
至于其具体计算,本文不再深入讨论,只是强调其低能极限为Einstein-Yang-Mills理论。\cite{Du:2017gnh,Fu:2017uzt}
\section{*非最小纯旋量超弦}
纯旋量超弦可以通过多引入一组费米鬼场$(r_\alpha,s^\beta)$和玻色鬼场$(\hat\lambda^\alpha,\hat w_\beta)$等价描述,只考虑左模的开弦作用量为:\cite{Berkovits:2005bt}
\begin{equation}
	S_{\mathrm{NMPS}}=\frac{1}{\pi}\int d^2z(\frac{1}{2}\partial x^m\overline{\partial}x_m+p_\alpha\overline{\partial}\theta^\alpha-w_\alpha\overline{\partial}\lambda^\alpha-\hat{w}^\alpha\overline{\partial}\hat{\lambda}_\alpha+s^\alpha\overline{\partial}r_\alpha)
\end{equation}
另外还要额外附加约束:
\begin{equation}
	(\overline{\lambda}\gamma^mr)=0
\end{equation}
新引入变量有如下OPE:
\begin{equation}
	\overline{\lambda}_\alpha(z)\overline{w}^\beta(y)\sim\frac{\delta_\alpha^\beta}{z-y},\quad s^\alpha(z)r_\beta(w)\sim\frac{\delta_\beta^\alpha}{z-w}
\end{equation}
类似有能动张量鬼数流和Lorentz流的定义:
\begin{equation}
		\hat{N}_{mn} = \frac{1}{2} \left( \hat{w} \gamma_{mn} \hat{\lambda} - s \gamma_{mn} r \right), \quad
		\hat{J}_{\hat{\lambda}} = \hat{w}^\alpha \hat{\lambda}_\alpha - s^\alpha r_\alpha, \quad 
		T_{\hat{\lambda}} = \hat{w}^\alpha \partial \hat{\lambda}_\alpha - s^\alpha \partial r_\alpha.
\end{equation}
BRST荷为:
\begin{equation}
	\hat Q_B=\int dz(\lambda^\alpha d_\alpha+\hat{w}^\alpha r_\alpha)\cong Q_B
\end{equation}
$\hat Q_B$和$Q_B$的上同调相同意味着顶角算符总能找到不含额外变量的表示,而这个表示正是前面求出的\ref{eq:5.39}和\ref{eq:5.40}。非最小纯旋量形式对树级振幅的求解与纯旋量形式是一致的。圈级振幅计算重点是找到Beltrami微分插入的类似项,而非最小纯旋量形式正好能找到$b$鬼场的类似物:
\begin{equation}
	\label{eq:5.94}
	\begin{aligned}
		b =&  s^\alpha \partial \overline{\lambda}_\alpha + \frac{1}{4(\overline{\lambda}\lambda)} \left[ 2\Pi^\mu (\overline{\lambda}\gamma_\mu d) - N_{\mu\nu} (\overline{\lambda}\gamma^{\mu\nu} \partial\theta) - J_\lambda (\overline{\lambda}\partial\theta) - (\overline{\lambda}\partial^2\theta) \right] \\
		&+ \frac{(\overline{\lambda}\gamma^{\mu\nu\rho} r)(d\gamma_{\mu\nu\rho}d + 24N_{\mu\nu}\Pi_\rho)}{192(\overline{\lambda}\lambda)^2} - \frac{(r\gamma_{\mu\nu\rho}r)(\overline{\lambda}\gamma^\mu d)N^{\nu\rho}}{16(\overline{\lambda}\lambda)^3} 
		+ \frac{(r\gamma_{\mu\nu\rho}r)(\overline{\lambda}\gamma^{\rho\sigma\tau}r)N^{\mu\nu}N_{\sigma\tau}}{128(\overline{\lambda}\lambda)^4}
	\end{aligned}
\end{equation}
从而可以把振幅写成类似\ref{eq:4.18}和\ref{eq:4.22}的形式:
\begin{equation}
	\label{eq:5.95}
	\mathcal{A}=\int d^{3g-3}\tau\llangle[\Bigg]\mathcal{N}(y)\prod_{i=1}^{3g-3}(\int dw_i\mu_i(w_j)b(w_j))\prod_{j=1}^N\int dz_jU(z_j)\rrangle[\Bigg]
\end{equation}
上式非零模积分可以用OPE进行,但是共形权为$1$的场在亏格$g$黎曼面上有$g$个零模,共形权为$0$的则有一个,而$\mathcal{N}$是正规化因子。本文并不去详细探讨这些细节,因为弦振幅任何圈级计算都会大大超出本文讨论范围。文献\cite{Berkovits:2006bk}中给出了更多利用非最小纯旋量形式计算圈级弦振幅的例子。

虽然\ref{eq:5.95}看起来以及极其复杂,但若是使用纯旋量形式计算会碰见更加复杂的式子:
\begin{equation}
\begin{aligned}
		A=&\int d^2\tau_1\ldots d^2\tau_{3\boldsymbol{g}-3}\llangle[\Bigg]\left|\prod_{P=1}^{3\boldsymbol{g}-3}\int d^2u_P\mu_P(u_P)\tilde{b}_{B_P}(u_P,z_P)\prod_{P=3g-2}^{10g}Z_{B_P}(z_P)\right.\\
	&\times\left.\prod_{R=1}^gZ_J(v_R)\prod_{I=1}^{11}Y_{C_I}(y_I)\right|^2\prod_{T=1}^N\int d^2t_TU_T(t_T)\rrangle[\Bigg]
\end{aligned}
\end{equation}
其中:
\begin{equation}
	Y_C = C_\alpha \theta^\alpha \delta(C_\beta \lambda^\beta), \quad 
	Z_B = \frac{1}{2} B_{\mu\nu} (\lambda \gamma^{\mu\nu} d) \delta(B^{\rho\sigma} N_{\rho\sigma}), \quad 
	Z_J = (\lambda^\alpha d_\alpha) \delta(J_{\text{PS}})
\end{equation}
其中最为关键的$\tilde{b}$的构造更是比\ref{eq:5.94}复杂得多,其详细表达式和相关细节请见\cite{Berkovits:2004px,Oda:2005sd,Oda:2004bg}。