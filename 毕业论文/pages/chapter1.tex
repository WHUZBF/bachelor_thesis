\chapter{引言}
弦理论是量子引力理论的重要候选者之一,有望统一四大基本相互作用。对弦论的研究最早起源于对强相互作用的研究,1968年,Veneziano给出了强子散射的经验公式,这被认为是弦理论的第一个公式,现代语境下其描述开弦四快子振幅。从某种角度上来说对弦论的研究起源于对弦振幅的研究。而且目前弦论只有微扰意义上的定义,所以弦振幅的研究尤其重要\cite{berkovits2022snowmasswhitepaperstring}。

最早被发展的弦理论是不含费米子谱的玻色弦理论,为了能够描述费米子并消除不稳定真空,需要引入超对称研究超弦理论。由于弦理论本身可以看作是世界面上的共形场论,所以引入超对称最简单的方式是直接引入世界面场论,利用二维$\mathcal{N}=1$ 超对称共形场论来构造超弦,这便是Ramond-Neveu-Schwarz(RNS)超弦。但最终弦振幅的结果和世界面没有关系,真正需要的超对称构造是靶空间超对称,在RNS超弦中这通过GSO投影实现。由于弦振幅本身是具有靶空间超对称而不是世界面超对称的物理量,所以直接使用RNS超弦计算振幅会十分复杂。比如两圈四点RNS超弦振幅计算,D'Hoker和Phong用了六篇论文才完全解决\cite{DHoker:2001kkt,DHoker:2001qqx,DHoker:2001foj,DHoker:2001jaf,DHoker:2005dys,DHoker:2005vch,DHoker:2002hof}。

虽然超弦已经经历过两次革命,但关于保持靶空间超对称的超弦理论的寻找还是个难题。最早Green-Schwarz超弦\cite{Green:1983wt,Green:1983sg}实现了靶空间超对称,但是只能在非协变的光锥坐标下量子化。后续Siegle改进了这一形式\cite{Siegel:1985xj}但最终被发现与RNS形式不等价,所以无法得到正确的弦振幅。本世纪初,Nathan Berkovits在Siegle超弦形式下成功发展出了能够协变量子化的保持靶空间超对称的超弦\cite{Berkovits:2000fe}。由于Berkovis的构造依赖于一种特殊的靶空间旋量——纯旋量,所以这一形式也被称为纯旋量超弦。

利用这一形式,原本在RNS超弦中难以计算的弦振幅被大大简化,比如利用纯旋量超弦计算两圈四点振幅要容易得多\cite{Berkovits:2005df},后来这也被证明与RNS超弦的计算结果等价\cite{Berkovits:2005ng}。利用纯旋量超弦,Berkovits的学生Mafra,和Schlotterer、Stieberger合作给出了任意点开弦无质量态盘面振幅的一般公式\cite{Mafra:2011nv,Mafra:2011nw}:
\begin{equation}
	\mathcal{A}_{n}(P)=(2\alpha^{\prime})^{n-3}\int d\mu_{P}^{n}\left[\prod_{k=2}^{n-2}\sum_{m=1}^{k-1}\frac{s_{mk}}{z_{mk}}A^{\text{SYM}}_{n}(1,2,\ldots,n)+\mathrm{perm}(2,3,\ldots,n-2)\right]
\end{equation}
这是本论文的核心,本论文将聚焦于弦论盘面振幅的计算,尤其是如何从纯旋量超弦导出此公式。下面我将给出本论文的行文结构,方便读者阅读。

第\ref{chap:2}章首先介绍了玻色弦理论,特别是正则量子化、路径积分量子化和BRST量子化三种后面会经常用到的量子化方法;第\ref{chap:3}章介绍RNS超弦和GSO投影,详细构造了RNS超弦无质量顶角算符。

由于弦振幅涉及到黎曼曲面模空间的计算,所以第\ref{chap:4}章首先简短的从数学上介绍黎曼曲面及其模空间。不过本论文主要考虑球面和盘面振幅,所以模空间是计算是平凡的。为了完整性,在第四章依旧介绍了玻色弦任意圈弦振幅的数学形式,虽然本论文主要考虑超弦振幅,但了解玻色弦振幅的数学形式对理解超弦振幅是必不可少的。可惜由于超弦圈级振幅计算极其复杂,所以本论文并未讨论其一般形式。文献\cite{Witten:2012bh,DHoker:2002hof}中有较为细致的考虑。本章还利用RNS超弦进行了一些具体计算,给出了球面和盘面弦振幅之间的Kawai-Lewellen-Tye关系以及盘面振幅满足的单值关系。

第\ref{chap:5}章是本文主要工具纯旋量超弦的介绍,从Brink-Schwarz超对称粒子出发指出GW形式不可协变量子化的问题,然后启发性(自上而下)地构造出了纯旋量超弦。从历史的角度(自下而上)出发构造纯旋量超弦可见文献\cite{Berkovits:2002zk,Mafra:2008gkx}。

第\ref{chap:6}章则是本论文的主要结论,利用纯旋量超弦导出了任意点无质量态开弦盘面振幅的一般形式,并且讨论了其场论极限,十维超对称Yang-Mills理论振幅。由于纯旋量超弦的计算中会自然涌现出自由李代数结构,这一结构非常容易帮助讨论规范理论的色-运动学对偶,所以在本章最后讨论了如何构造满足色-运动学对偶的Bern-Carrasco-Johanson分子\cite{Mafra:2011kj}。

最后,本论文还给出了两个方便于实际计算的附录,附录\ref{appendix:A}给出了本论文主要使用到的算符乘积展开,在正文中将不再重复提及;纯旋量超弦计算涉及到大量有关$\gamma$矩阵的恒等式,这被囊括在附录\ref{appendix:B}中。另外,本论文还有一些标有$*$号的章节,这表示其脱离于本论文主线但是在本论文正文中有所提及,为了完整性而包括在内,略过并不影响对本文结论的理解。

本论文也是作者对过去半年多时间学习弦理论的总结,若存在疏漏之处,恳请学界同仁不吝指正!