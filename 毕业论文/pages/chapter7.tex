\chapter{结论}

本文首先从RNS超弦出发,对超弦无质量开弦盘面振幅和闭弦球面振幅进行了一些具体计算,并给出了超弦树级振幅的KLT关系以及单值关系。同时本文也从黎曼曲面的角度在数学上讨论了玻色弦振幅一般的数学形式,同时也对超弦振幅有类似的讨论,指出了超弦振幅计算存在的困难。由于超弦理论本身要求靶空间的超对称,弦振幅本身也自然满足的是靶空间超对称,而RNS超弦属于世界面超对称的弦理论,虽然表述上较为简单,但是靶空间超对称来源于GSO投影,可以看作是被“隐藏”了起来。所以在计算具有靶空间超对称的超弦振幅时有很大的困难。为此我们引入了在GS超弦以及Siegel超弦基础上发展而来的Berkovits(纯旋量)超弦。做为可协变量子化的靶空间超对称超弦理论,纯旋量超弦在超弦振幅的计算上有无限潜能。

本论文利用纯旋量超弦回顾了无质量态盘面开超弦振幅的计算,特别是巧妙利用顶角算符处于BRST上同调定义了多粒子超场从而给出了纯旋量超弦无质量态顶角算符OPE之间的一般形式:
\begin{equation}
	V_A(z_a)U_B(z_b) \to \frac{V_{[A,B]}(z_b)}{z_{ab}}, \quad U_A(z_a)U_B(z_b) \to \frac{U_{[A,B]}(z_b)}{z_{ab}}
\end{equation}
利用这一形式我们最终得到了无质量态盘面开弦超振幅的一般计算公式:
\begin{equation}
		\mathcal{A}_n(P)=(2\alpha^{\prime})^{n-3}\int d\mu_P^n\left[\prod_{k=2}^{n-2}\sum_{m=1}^{k-1}\frac{s_{mk}}{z_{mk}}A_n(1,2,\ldots,n)+\mathrm{perm}(2,3,\ldots,n-2)\right]
\end{equation}
由于球面振幅以及开闭弦混合振幅可以由开弦振幅表达,所以我们从纯旋量超弦出发完全计算得到了树级超弦振幅。并且同时得到了低能近似下的十维超对称杨-米尔斯理论(SYM)超振幅的纯旋量超空间上同调表述:
\begin{equation}
	A_n(P,n):=\sum_{XY=P}\langle M_XM_YM_n\rangle
\end{equation}
纯旋量超弦不仅给出了如此紧凑的振幅表达式,让我们一窥其中的自由李代数组合数学结构,而且还能够直接利用DDM基底构造出满足色-运动学对偶对偶的BCJ因子:
\begin{equation}
	N_{1|XnY|n-1}=(-1)^{|Y|-1}\langle V_{1X}V_{(n-1)\tilde{Y}}V_{n}\rangle
\end{equation}
这无疑也为我们研究微扰引力理论和规范理论之间的双复制关系提供了一种思路。而且纯旋量超弦振幅与SYM理论振幅在低能展开下的关系也给出了弦振幅深层次的数论结构。所以利用纯旋量超弦不仅能够实现RNS超弦难以计算的超弦振幅,还能够给出弦振幅中丰富的数学结构。

不过本论文并未讨论近年来发展迅速的纯旋量超弦圈级振幅的计算\cite{gzq},也未对含有质量态时超弦振幅的计算进行讨论。比如对第一激发态在文献\cite{Berkovits:2002qx,Chakrabarti:2018mqd,Chakrabarti:2018bah}中已有相关讨论,实际上纯旋量超弦BRST上同调包含所有超弦激发态(与RNS超弦粒子谱相同)\cite{Berkovits:2000nn,Berkovits:2001mx,Aisaka:2008vw}。但是具体对更高有质量激发态顶角算符的构造以及超弦振幅的计算仍是目前纯旋量形式发展的前沿问题。期待作者基于本文对超弦无质量态树级振幅的理解能在后续对弦微扰论的研究有所启发。