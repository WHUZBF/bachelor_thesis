% 附录

\chapter{本论文主要使用到的算符乘积展开}
\label{appendix:A}
本附录给出一些振幅计算中经常用到的OPE以及能动张量等CFT相关约定,以供查阅。
\section{自由CFT}
本节对应玻色弦世界面CFT。
\subsection{自由玻色CFT}
$XX$包含四项

另外,根据加倍技巧,对于BCFT,左右模之间的OPE不是$0$:
\begin{equation}
	X^\mu(z_1)\tilde X^\nu(\bar z_2)\sim X^\mu(z_1)X^\nu(z'_2)\sim -\frac{\alpha^{\prime}}{2}\ln|z_1-\overline{z}_2|
\end{equation}
不过弦论中开弦BCFT计算只涉及到实轴上插入,所以不必过于在意上式。而在实轴上源点及其镜像重合,所以左右模OPE和纯左模OPE都会贡献:
\begin{equation}
	 X^\mu (y_1) X^\nu(y_2) \sim -2\alpha'\ln |y_1-y_2|
\end{equation}
这从\ref{eq:4.36}也能直接看出来,这也印证了$\alpha'_{\text{cl}}\sim4\alpha'_{\text{op}}$。
\subsection{$bc$鬼场}
\section{$\mathcal{N}=1$ SCFT}
本节对应RNS超弦世界面CFT,玻色场OPE与上一节相同。
\subsection{自由费米CFT}

\subsection{$\beta\gamma$鬼场}

\section{纯旋量形式}
这里只讨论左模,作用量见\ref{eq:5.23},能动张量以及对应的超流(费米Lorentz流):
\begin{equation}
	T_{\mathrm{PS}} = -\frac{1}{2} \Pi^\mu \Pi_\mu - d_\alpha \partial \theta^\alpha + w_\alpha \partial \lambda^\alpha, \quad M^{\mu\nu} = \Sigma^{\mu\nu} + N^{\mu\nu} = -\frac{1}{2} (p \gamma^{\mu\nu} \theta) + \frac{1}{2} (w \gamma^{\mu\nu} \lambda)
\end{equation}
由于纯旋量约束,这并不是一个自由CFT:
\begin{equation}
	\begin{aligned}
		X^\mu(z,\overline{z}) X^\nu(w,\overline{w}) 
		&\sim -\eta^{\mu\nu} \ln|z-w|^2, 
		& d_\alpha(z) \theta^\beta(w) 
		&\sim \frac{\delta_\alpha^\beta}{z-w}, \\
		d_\alpha(z) d_\beta(w) 
		&\sim -\frac{\gamma_{\alpha\beta}^\mu \Pi_\mu(w)}{z-w}, 
		& d_\alpha(z) \Pi^\mu(w) 
		&\sim \frac{(\gamma^\mu \partial\theta(w))_\alpha}{z-w}, \\
		\Pi^\mu(z) \Pi^\nu(w) 
		&\sim -\frac{\eta^{\mu\nu}}{(z-w)^2},&
		\partial\theta^\alpha(z) \{ \partial\theta^\beta(w),\Pi^\mu(w),&N^{\mu\nu}(w) \} \sim \mathrm{regular}
	\end{aligned}
\end{equation}
对于任意的不显含$X,\theta$导数项的超场$K(X,\theta)$:\footnote{平面波就是一个最简单的例子。}
\begin{equation}
	\begin{aligned}
		d_\alpha(z) K\left(X(w,\overline{w}), \theta(w)\right) 
		&\sim \frac{D_\alpha K\left(X(w,\overline{w}), \theta(w)\right)}{z-w}, \\
		\Pi^\mu(z) K\left(X(w,\overline{w}), \theta(w)\right) 
		&\sim -\frac{\partial^\mu K\left(X(w,\overline{w}), \theta(w)\right)}{z-w}.
	\end{aligned}
\end{equation}
其中超对称协变导数为:
\begin{equation}
	D_\alpha=\frac{\partial}{\partial\theta^\alpha}+\frac{1}{2}(\gamma^\mu\theta)_\alpha\partial_\mu
\end{equation}
还有一些有关超流的OPE:
\begin{equation}
	\begin{aligned}
		M^{\mu\nu}(z) M^{\rho\sigma}(w) 
		&\sim \frac{\eta^{\rho[\mu} M^{\nu]\sigma}(w) - \eta^{\sigma[\mu} M^{\nu]\rho}(w)}{z-w} + \frac{\eta^{\mu[\sigma} \eta^{\rho]\nu}}{(z-w)^2}, \\
		N^{\mu\nu}(z) N^{\rho\sigma}(w) 
		&\sim \frac{\eta^{\rho[\mu} N^{\nu]\sigma}(w) - \eta^{\sigma[\mu} N^{\nu]\rho}(w)}{z-w} - 3 \frac{\eta^{\mu[\sigma} \eta^{\rho]\nu}}{(z-w)^2}, \\
		N^{\mu\nu}(z) \lambda^\alpha(w) 
		&\sim \frac{1}{2} \frac{(\gamma^{\mu\nu})^\alpha{}_\beta \lambda^\beta(w)}{z-w}.
	\end{aligned}
\end{equation}