% 附录

\chapter{本论文主要使用到的算符乘积展开}
\label{appendix:A}
本附录给出一些振幅计算中经常用到的OPE以及能动张量等CFT相关约定,以供查阅。
\section{玻色弦CFT}
\subsection{自由玻色CFT}
作用量和能动张量定义为:
\begin{equation}
	\label{eq:A1}
	S=\frac{1}{2\pi\alpha^{\prime}}\int d^2z\mathrm{~}\partial X^\mu\bar{\partial}X_\mu,\quad
	T(z)=-\frac{1}{\alpha^{\prime}}:\partial X\partial X:
\end{equation}
根据运动方程$\partial\bar\partial X = 0$,$X(z,\bar z) = X(z) + \tilde X(\bar z)$ ,后面的讨论主要以左行模为主。OPE为:
\begin{equation}
	\label{eq:A2}
	X^\mu(z_1,\bar{z}_1)X^\nu(z_2,\bar{z}_2)\sim-\frac{\alpha^\prime}{2}\eta^{\mu\nu}\ln|z_{12}|^2
\end{equation}
平面波$:e^{ik\cdot X}:$共形权为$\left(\frac{\alpha^{\prime}k^2}{4},\frac{\alpha^{\prime}k^2}{4}\right)$,涉及到$e^A$形式算符的缩并有如下比较常用的等式:
\begin{equation}
\wick{\c A(z): \c B^n:(w)}=nA(z)B(w):B^{n-1}(w):
\end{equation}
\begin{equation}
	\wick{\c A(z):\c e^{B(w)}}:=A(z)B(w):e^{B(w)}:
\end{equation}
\begin{equation}
	\begin{aligned}
		\wick{:\c e^{A(z)}::\c e^{B(z)}:}&=\sum_{m,n,k}\frac{k!}{m!n!}\begin{pmatrix}m\\k\end{pmatrix}\begin{pmatrix}n\\k\end{pmatrix}[{A(z)}B(w)]^k:A^{m-k}(w)B^{n-k}(w):\\&=\exp\left\{\wick{\c A(z)\c B(w)}\right\}:e^{A(w)}e^{B(w)}:
	\end{aligned}
\end{equation}
共形权$h$初级场$\phi$满足:
\begin{equation}
	T(z)\phi(w)\sim\frac{h}{(z-w)^{2}}\phi(w)+\frac{1}{z-w}\partial_{w}\phi(w)
\end{equation}
对应模展开:
\begin{equation}
[L_m,\phi_n]=((h-1)m-n)\phi_{m+n}
\end{equation}
$TT$ OPE为:
\begin{equation}
	\label{eq:A8}
	\begin{aligned}
		T(z)T(w)\sim\frac{c/2}{(z-w)^4}+\frac{2T(w)}{(z-w)^2}+\frac{\partial_wT(w)}{z-w}
	\end{aligned}
\end{equation}
对于$D$维自由玻色场,$c= D$,上述$TT$ OPE对应Virasoro代数:
\begin{equation}
	[L_m,L_n] = (m-n)L_{m+n}+\frac{c}{12}\left(m^{3}-m\right)\delta_{m,-n}
\end{equation}
左右模之间OPE是解耦的,不过,根据加倍技巧,对于BCFT,左右模之间的OPE不是$0$:\footnote{形象理解为右行模在边界附近的反射的影响。}
\begin{equation}
	X^\mu(z_1)\tilde X^\nu(\bar z_2)\sim X^\mu(z_1)X^\nu(z'_2)\sim -\frac{\alpha^{\prime}}{2}\ln|z_1-\overline{z}_2|
\end{equation}
不过弦论中开弦BCFT计算只涉及到实轴上插入,所以不必过于在意上式。而在实轴上源点及其镜像重合,所以左右模OPE和纯左模OPE都会贡献:
\begin{equation}
	 X^\mu (y_1) X^\nu(y_2) \sim -2\alpha'\ln |y_1-y_2|
\end{equation}
这从\ref{eq:4.36}也能直接看出来,由于左右模非零,所以相较于\ref{eq:A2}来说$X(z,\bar z)X(w,\bar w)$ OPE会多一项贡献。这同时也印证了$\alpha'_{\text{cl}}\sim4\alpha'_{\text{op}}$。
\subsection{$bc$鬼场}
左模部分作用量和能动张量为:
\begin{equation}
	\label{eq:A12}
	S=\frac{1}{2\pi}\int d^2zb\bar{\partial}c,\quad T(z)=:(\partial b)c:-\lambda\partial(:bc:)
\end{equation}
$bc$是反对易的费米场,共形权和中心荷为
\begin{equation}
	h_b=\lambda,\quad h_c=1-\lambda\quad c=-3(2\lambda-1)^2+1
\end{equation}
对于弦论,$\lambda = 2$。OPE为:
\begin{equation}
	b(z)c(w)\sim\frac{1}{z-w}
\end{equation}
其它$bb$和$cc$ OPE平凡。
\section{$\mathcal{N}=1$ SCFT}
本节对应RNS超弦世界面CFT,玻色场OPE与上一节相同。
\subsection{自由费米CFT}
左行模部分作用量和能动张量为:
\begin{equation}
S=\frac{1}{2\pi}\int\mathrm{d}^2z\psi^\mu\overline{\partial}\psi_\mu	,\quad T(z) = -\frac{1}{2}:\psi\cdot\partial\psi:(z)
\end{equation}
中心荷为$D/2$,$\psi$共形权为$\frac12$。
\begin{equation}
	\psi^\mu(z)\psi^\nu(w) = \frac{\eta^{\mu\nu}}{z-w}
\end{equation}
$\psi X$之间OPE平凡。总的物质场能动张量为:
\begin{equation}
	T^\mathrm{m}(z)=-\frac{1}{\alpha^{\prime}}:\partial X\cdot\partial X:-\frac{1}{2}:\psi\cdot\partial\psi:
\end{equation}
超共性变换对应的总的物质场超流为:
\begin{equation}
	G^\mathrm{m}(z)=i\sqrt{\frac{2}{\alpha^{\prime}}}\psi^\mu\partial X_\mu
\end{equation}
$TT$ OPE仍旧满足共性代数\ref{eq:A8},超共形代数:
\begin{equation}
	\label{eq:A19}
	\begin{aligned}
		G^\mathrm{m}(z_1)G^\mathrm{m}(z_2)&\sim\frac{\frac{2}{3}c}{(z_1-z_2)^3}+\frac{2T^\mathrm{m}(z_2)}{(z_1-z_2)}\\
T^\mathrm{m}(z_1)G^\mathrm{m}(z_2)&\sim\frac{\frac{3}{2}G^\mathrm{m}(z_2)}{(z_1-z_2)^2}+\frac{\partial G^\mathrm{m}(z_2)}{(z_1-z_2)}
	\end{aligned}
\end{equation}
超共形权为$h$的超共形初级场定义为共形权为$h$的初级场$\phi_h$且额外满足:
\begin{equation}
	\label{eq:A20}
	G^\mathrm{m}(z_1)\phi_h(z_2)\sim\frac{\phi_{h+1/2}(z_2)}{(z_1-z_2)}
\end{equation}
超共形初级场对$(\phi_h,\phi_{h+1/2})$定义为$\psi_h$和$\psi_{h+1/2}$分别为超共形权$h$的超共形初级场和共形权为$h+1/2$的共形初级场,且:
\begin{equation}
	\label{eq:A21}
	G^\mathrm{m}(z_1)\phi_{h+1/2}(z_2)\sim\frac{h\phi_h(z_2)}{(z_1-z_2)^2}+\frac{\partial_{z_2}\phi_h(z_2)}{2(z_1-z_2)}
\end{equation}
\subsection{$\beta\gamma$鬼场}
$\beta\gamma$鬼场和$bc$鬼场非常相似,从作用量和能动张量就能看到这一点:
\begin{equation}
	S=\frac{1}{2\pi}\int d^2z\beta\bar{\partial}\gamma,\quad T(z)=:(\partial\beta)\gamma:-\lambda\partial(:\beta\gamma:)
\end{equation}
共形权和中心荷为:
\begin{equation}
	h_\beta=\lambda^2,\quad h_\gamma=1-\lambda,\quad c=3(2\lambda-1)^2-1
\end{equation}
弦论中$\lambda = \frac32$。区别主要在于$\beta\gamma$鬼满足的是玻色统计,OPE为:
\begin{equation}
	\beta(z)\gamma(w)\thicksim-\frac{1}{z-w}
\end{equation}
总的鬼场超流为:
\begin{equation}
	G^{(\mathrm{gh})}(z)=-\frac{1}{2}(\partial\beta)c+\frac{3}{2}\partial(\beta c)-2b\gamma
\end{equation}
超弦费米部分CFT最重要的是模展开包含NS和R两个部分:
\begin{equation}
\begin{gathered}
		\begin{cases}
		\psi_\mathrm{NS}^\mu(z)=\sum_{r\in\mathbb{Z}+\frac{1}{2}}\psi_r^\mu z^{-r-\frac{1}{2}}\\\psi_\mathrm{R}^\mu(z)=\sum_{n\in\mathbb{Z}}\psi_n^\mu z^{-n-\frac{1}{2}}
		\end{cases}
		,\quad 
		\begin{cases}G_{\mathrm{NS}}(z)=\sum_{r\in\mathbb{Z}+\frac{1}{2}}G_rz^{-r-\frac{3}{2}}\\G_{\mathrm{R}}(z)=\sum_{n\in\mathbb{Z}}G_nz^{-n-\frac{3}{2}}\end{cases}
		\\
	\begin{cases}\beta_{\mathrm{NS}}(z)=\sum_{r\in\mathbb{Z}+\frac{1}{2}}\beta_rz^{-r-\frac{3}{2}}\\\beta_{\mathrm{R}}(z)=\sum_{n\in\mathbb{Z}}\beta_nz^{-n-\frac{3}{2}}\end{cases},\quad
				\begin{cases}\gamma_{\mathrm{NS}}(z)=\sum_{r\in\mathbb{Z}+\frac{1}{2}}\gamma_rz^{-r+\frac{1}{2}}\\\gamma_{\mathrm{R}}(z)=\sum_{n\in\mathbb{Z}}\gamma_nz^{-n+\frac{1}{2}}\end{cases}
\end{gathered}
\end{equation}
OPE \ref{eq:A19}给出的超共形代数可以写成下面的统一形式:
\begin{equation}
	\begin{aligned}&[L_n,G_r]=\frac{n-2r}{2}G_{n+r}\\&\{G_r,G_s\}=2L_{r+s}+\frac{c}{12}(4r^2-1)\delta_{r+s,0}\end{aligned}
\end{equation}
NS和R部分只需要对下标取整数或半整数即可。另外上面讨论并未对$G$加上标m,gh,tot区分具体是物质场超流、鬼场超流还是总超流,因为他们满足的超共形代数是一样的。
\subsection{*超空间表述}
用(二维)超对称场论的语言可以把前面的讨论写成更加显现出$\mathcal{N}=1$超对称的形式。引入和$\{z,\bar z\}$对应的超空间坐标$\{\theta,\bar\theta\}$,他们是二维Weyl旋量。超空间导数定义为:
\begin{equation}
	D=\partial_\theta+\theta\partial_z,\quad\bar{D}=\partial_{\bar{\theta}}+\bar{\theta}\partial_{\bar{z}}
\end{equation}
考虑坐标变换:
\begin{equation}
	z=(z,\theta)\to z^{\prime}=(z^{\prime}(z,\theta),\theta^{\prime}(z,\theta))
\end{equation}
共性变换由$\bar \partial z' = 0$的全纯映射生成,普通偏导数由链式法则变换为$\partial=\frac{\partial z^{\prime}}{\partial z}\partial^{\prime}$,这一点是平凡的,超共形变换则是类似要求:
\begin{equation}
	D=(D\theta^{\prime})D^{\prime}\Rightarrow Dz^{\prime}-\theta^{\prime}D\theta^{\prime}=0
\end{equation}
诱导出如下变换:
\begin{equation}
	\begin{aligned}&z^{\prime}=f(z)+\frac{1}{2}\theta g(z)\epsilon(z),\\&\theta^{\prime}=\frac{1}{2}\epsilon(z)+\theta g(z),\quad g^{2}=\partial f+\frac{1}{4}\epsilon\partial\epsilon\end{aligned}
\end{equation}
$f$和$g$是普通函数,$\epsilon$则是Grassmann反对易的函数。此式便是世界面上超共形变换的形式,$\theta = 0$时上式退化为共形变换。上式的无穷小变换形式为:
\begin{equation}
	\label{inf}
	\delta z=\xi+\frac{1}{2}\theta\epsilon,\quad\delta\theta=\frac{1}{2}\epsilon+\frac{1}{2}\theta\partial\xi
\end{equation}
这里$\xi$对易,$\epsilon$反对易。超共形初级场对构成一个手征超场:
\begin{equation}
	\Phi_h(z)=\phi_h(z)+\theta\phi_{h+\frac12}(z),\quad \bar D\Phi(z) = 0
\end{equation}
类似共形初级场定义要求共形变换下:
\begin{equation}
	\phi^{\prime}(z^{\prime},\bar{z}^{\prime})=\left(\frac{\partial z^{\prime}}{\partial z}\right)^{-h}\left(\frac{\partial\bar{z}^{\prime}}{\partial\bar{z}}\right)^{-\bar{h}}\phi(z,\bar{z})
\end{equation}
超共形初级场要求超共形变换下:
\begin{equation}
	\Phi(z)=(D\theta^{\prime})^{2h}\Phi^{\prime}(z^{\prime})
\end{equation}
利用\ref{inf}可以导出上面要求的无穷小变换下对$\phi$提出的要求,而$\delta_\xi$由能动张量$T$的OPE生成,$\delta_\epsilon$由超流$G$的OPE生成,由此可以得到正好是要求$\phi$满足OPE \ref{eq:A20}和\ref{eq:A21},且是通常意义下的共形初级场。能动张量和超流同样可以组成一个超场:
\begin{equation}
	\label{eq:A36}
	\mathscr{T}^\text{m}(z)=G^\text{m}(z)+\theta T^\text{m}(z)
\end{equation}
而且是共形权为$\frac32$的共形超场,\ref{eq:A20}和\ref{eq:A21}的要求转化为:
\begin{equation}
	\mathscr{T}^{\text{m}}(z_1)\Phi(z_2)\sim\frac{h\theta_{12}\Phi(z_2)}{z_{12}^2}+\frac{\frac{1}{2}D\Phi(z_2)}{z_{12}}+\frac{\theta_{12}D^2\Phi(z_2)}{z_{12}}
\end{equation}
类似,\ref{eq:A19}可以被紧凑的写为:
\begin{equation}
	\mathscr{T}^\text{m}(z_1)\mathscr{T}^\text{m}(z_2)\sim\frac{\frac{1}{6}{c^\text{m}}}{z_{12}^3}+\frac{\frac{3}{2}\theta_{12}\mathscr{T}(z_2)}{z_{12}^2}+\frac{\frac{1}{2}D\mathscr{T}(z_2)}{z_{12}}+\frac{\theta_{12}D^2\mathscr{T}(z_2)}{z_{12}}
\end{equation}
RNS超弦中物质场可以组合为一个二维超场:
\begin{equation}
	\mathscr{X}^\mu(z,\bar{z})=X^\mu+i\theta\psi^\mu+i\bar{\theta}\bar{\psi}^\mu+\theta\bar{\theta}F^\mu
\end{equation}
$F^\mu$是人为引入的保证超对称性的辅助场。利用这个定义可以将物质场能动张量写成和\ref{eq:A1}一致且明显保持共形超对称性的形式:
\begin{equation}
	\begin{aligned}
		S_\text{m}&=\frac{1}{2\pi\alpha'}\int d^2zd^2\theta\bar{D}\mathscr{X}^\mu D\mathscr{X}_\mu\\&=\frac{1}{2\pi\alpha'}\int d^2z\left(\partial X^\mu\overline{\partial}X_\mu+\psi^\mu\overline{\partial}\psi_\mu+\overline{\psi}^\mu\partial\overline{\psi}_\mu+F^\mu F_\mu\right)
	\end{aligned}
\end{equation}
$\delta S_m/\delta F$给出$F=0$,所以辅助场其实可以略去,这就完全回到了前面的讨论,同样,对于鬼场,可以组合成两个共形超场:
\begin{equation}
	B=\beta+\theta b,\quad C=c+\theta\gamma
\end{equation}
超共形权分别为$\frac32$和$-1$,鬼场能动张量可以写成\ref{eq:A12}的形式:
\begin{equation}
	S_{\text{gh}}=\frac{1}{2\pi}\int d^2zd^2\theta B\overline{D}C=\frac{1}{2\pi}\int d^2z(b\overline{\partial}c+\beta\overline{\partial}\gamma)
\end{equation}
同样可以类似\ref{eq:A36}引入$\mathscr{T}^{\text{gh}}$,超共形代数是一样的。
\section{纯旋量形式}
这里只讨论左模,作用量见\ref{eq:5.23},能动张量以及对应的超流(费米Lorentz流):
\begin{equation}
	T_{\mathrm{PS}} = :-\frac{1}{2} \Pi^\mu \Pi_\mu - d_\alpha \partial \theta^\alpha + w_\alpha \partial \lambda^\alpha:, \quad M^{\mu\nu} = \Sigma^{\mu\nu} + N^{\mu\nu} = :-\frac{1}{2} (p \gamma^{\mu\nu} \theta) + \frac{1}{2} (w \gamma^{\mu\nu} \lambda):
\end{equation}
其中$h(\lambda^\alpha) = 0$,$h(w_\alpha) = +1$。由于纯旋量约束,这并不是一个自由CFT,只有$(p_\alpha,\theta^\beta)$这对共轭变量可以正常量子化为自由CFT形式:
\begin{equation}
	p_\alpha(z)\theta^\beta(w)\sim\frac{\delta_\alpha^\beta}{z-w}
\end{equation}
对于$\lambda w$鬼系统,则复杂一些,前面\ref{eq:5.55}中也看到了这一点。由于纯旋量约束,$w$鬼场其实存在一个规范变换:
\begin{equation}
	\delta w_\alpha=\Lambda^\mu(\gamma_\mu\lambda)_\beta
\end{equation}
$\Lambda^\mu$是规范变换参数,这个规范变换可以消去$w_\alpha$的五个自由度,所以$w_\alpha$和$\lambda^\alpha$一样都是11个自由度的体系。由于鬼场存在规范变换,所以或许引入“鬼场的鬼场”,利用BV形式可以进行协变量子化,但更简单的做法是利用$\S$\ref{sec:u5}引入的$U(5)$分解,进行如下计算:\footnote{前文我们使用旋量指标的上下关系来表示手征,而不是用Van der Waerden记号\cite{Schwichtenberg:2015jcd},所以$w_\alpha \overline{\partial}\lambda^\alpha$应该理解为$w_{\dot{\alpha}} \overline{\partial}\mathcal{C_{\dot{\alpha}\beta}} \lambda^\beta$,下面的式子类似\ref{eq:diracps}用Dirac旋量的形式表示了出来。}
\begin{equation}
	\begin{aligned}
		w_\alpha \bar{\partial}\lambda^\alpha =& \Omega^T \mathcal{C}\partial\Lambda = \llangle \omega \mid \partial\lambda\rangle \\
		=&\frac{1}{5!}w_+\bar{\partial}\lambda^+\varepsilon_{abcde} \bra{0} b_{a}b_{b}b_{c}b_{d}b_{e} \mathcal{C}\ket{0}\cdot (-1)^{5+5}
		\\&+\frac12\frac{1}{2!3!}w^{fg}bar{\partial}\lambda_{ab}\bra{0}b_cb_db_e \mathcal{C} b^b b^a\ket{0}\varepsilon_{fgcde}\cdot (-1)^{3+3}+\mathcal{O}(w_a)\\
		=&w_+\bar{\partial}\lambda^+-\frac{1}{2}w^{ab}\bar{\partial}\lambda_{ab}+\mathcal{O}(w_a)
	\end{aligned}
\end{equation}
计算中利用了$b_a\mathcal{C}=\mathcal{C}b^a$以及:\cite{ZSDZ201905004}\footnote{注意这里反对易括号$[\bullet]$的定义不带归一化因子。}
\begin{equation}
	\varepsilon^{\mu_{1}\cdots\mu_{n-m}\alpha_{1}\cdots\alpha_{m}}\varepsilon_{\mu_{1}\cdots\mu_{n-m}\beta_{1}\cdots\beta_{m}}=(n-m)!\delta_{\beta_{1}}^{[\alpha_{1}}\cdots\delta_{\beta_{m}}^{\alpha_{m}]}
\end{equation}
所以可以将纯旋量超弦作用量中的$\lambda w$鬼部分写成如下形式:
\begin{equation}
	S_{\lambda w}=\frac{1}{2\pi\alpha'}\int d^2z(-w_+\bar{\partial}\lambda^++\frac{1}{2}w^{ab}\bar{\partial}\lambda_{ab})
\end{equation}
这里首先利用\ref{eq:5.69}消去$\lambda^a$,然后利用$w$的规范变换取$w_\alpha = 0$规范固定。不少文献也定义:
\begin{equation}
	\label{eq:A49}
\begin{aligned}
		&\lambda^+=e^s,\quad\lambda_{ab}=u_{ab},\quad\lambda^a=\frac{1}{8}e^{-s}\epsilon^{abcde}u_{bc}u_{de}\\
	&w_+= e^{-s}t,\quad w^{ab} = v^{ab}
\end{aligned}
\end{equation}
则作用量变为如下形式:\footnote{其实历史上的顺序是反过来的,是先有\ref{eq:A50},然后利用\ref{eq:A49}写成Lorenz协变的形式。}
\begin{equation}
	\label{eq:A50}
	S_{\lambda w}=\frac{1}{2\pi \alpha'}\int d^2z\left(-\partial t\overline{\partial}s+\frac{1}{2}v^{ab}\overline{\partial}u_{ab}\right)
\end{equation}
这里$a,b$取值都是$1,\ldots,5$。现在场都是独立的分量,所以可以直接量子化得到OPE:
\begin{equation}
	t(z)s(w)\sim\ln\left(z-w\right),v^{ab}(z)u_{cd}(w)\sim\frac{\delta_c^a\delta_d^b-\delta_d^a\delta_c^b}{z-w}
\end{equation}
剩下要做的就是将$d_\alpha,\Pi^\mu,N^{\mu\nu}$进行$U(5)$分解,比如$N^{\mu\nu}$可以分解为:
\begin{equation}
	N^{\mu\nu}\to (n,n_b^a,n^{ab},n_{ab})=-\left(\frac{m}{\sqrt{5}},m_b^a-\frac{1}{5}\delta_b^am,m^{ab},m_{ab}\right)
\end{equation}
这里$m$的意思是$\S$\ref{sec:u5}中分解Lorenz流的定义,实际使用中对$(\frac12,\frac12)$的分解常用上面的定义\cite{Berkovits:2001us,Hoogeveen:2009hk}。用$u,v,t,s$表示为:
\begin{equation}
	\begin{aligned}
		n&=-\frac{1}{\sqrt{5}}\left(\frac{1}{4}u_{ab}v^{ab}+\frac{5}{2}\partial t-\frac{5}{2}\partial s\right),\\n_b^a&=-u_{bc}v^{ac}+\frac{1}{5}\delta_b^au_{cd}v^{cd},\\n^{ab}&=-e^sv^{ab},\\n_{ab}&=-e^{-s}\left(2\partial u_{ab}-u_{ab}\partial t-2u_{ab}\partial s+u_{ac}u_{bd}v^{cd}-\frac{1}{2}u_{ab}u_{cd}v^{cd}\right).
	\end{aligned}
\end{equation}
然后就可以计算$U(5)$分解下的OPE,比如:
\begin{equation}
	\begin{aligned}
		n^{ab}(z)\lambda^c(w)&=-\frac{1}{8}e^{s(z)}\epsilon^{cdefg}\left(\wick{\c v^{ab}(z)\c u_{de}(w)}u_{fg}(w)+u_{de}(w)\wick{\c v^{ab}(z)\c u_{fg}(w)}\right)e^{-s(w)}\\&\sim-\frac{1}{8}e^{s(z)}\frac{\left(2\epsilon^{cabfg}u_{fg}(w)+2\epsilon^{cdeab}u_{de}(w)\right)}{z-w}e^{-s(w)}\\&\sim-\frac{1}{2}\epsilon^{abcde}\frac{\lambda_{de}(w)}{z-w}
	\end{aligned}
\end{equation}

注意$U(5)$分解破坏了协变性,并不意味着我们的量子化是非协变的,前面说过其实可以用协变的方法量子化只是很麻烦。这里虽然过程上看是非协变的,但是最终OPE都可以还原成协变的形式表达,最终我们得到纯旋量超弦CFT的如下物质场OPE:
\begin{equation}
	\begin{aligned}
		X^\mu(z,\overline{z}) X^\nu(w,\overline{w}) 
		&\sim -\eta^{\mu\nu} \ln|z-w|^2, 
		& d_\alpha(z) \theta^\beta(w) 
		&\sim \frac{\delta_\alpha^\beta}{z-w}, \\
		d_\alpha(z) d_\beta(w) 
		&\sim -\frac{\gamma_{\alpha\beta}^\mu \Pi_\mu(w)}{z-w}, 
		& d_\alpha(z) \Pi^\mu(w) 
		&\sim \frac{(\gamma^\mu \partial\theta(w))_\alpha}{z-w}, \\
		\Pi^\mu(z) \Pi^\nu(w) 
		&\sim -\frac{\eta^{\mu\nu}}{(z-w)^2},&
		\partial\theta^\alpha(z) \{ \partial\theta^\beta(w),\Pi^\mu(w),&N^{\mu\nu}(w) \} \sim \mathrm{regular}
	\end{aligned}
\end{equation}
对于任意的不显含$X,\theta$导数项的超场$K(X,\theta)$:\footnote{平面波就是一个最简单的例子。}
\begin{equation}
	\begin{aligned}
		d_\alpha(z) K\left(X(w,\overline{w}), \theta(w)\right) 
		&\sim \frac{D_\alpha K\left(X(w,\overline{w}), \theta(w)\right)}{z-w}, \\
		\Pi^\mu(z) K\left(X(w,\overline{w}), \theta(w)\right) 
		&\sim -\frac{\partial^\mu K\left(X(w,\overline{w}), \theta(w)\right)}{z-w}.
	\end{aligned}
\end{equation}
其中超空间导数为:
\begin{equation}
	D_\alpha=\frac{\partial}{\partial\theta^\alpha}+\frac{1}{2}(\gamma^\mu\theta)_\alpha\partial_\mu
\end{equation}
还有一些有关超流的OPE:
\begin{equation}
	\begin{aligned}
		M^{\mu\nu}(z) M^{\rho\sigma}(w) 
		&\sim \frac{\eta^{\rho[\mu} M^{\nu]\sigma}(w) - \eta^{\sigma[\mu} M^{\nu]\rho}(w)}{z-w} + \frac{\eta^{\mu[\sigma} \eta^{\rho]\nu}}{(z-w)^2}, \\
		N^{\mu\nu}(z) N^{\rho\sigma}(w) 
		&\sim \frac{\eta^{\rho[\mu} N^{\nu]\sigma}(w) - \eta^{\sigma[\mu} N^{\nu]\rho}(w)}{z-w} - 3 \frac{\eta^{\mu[\sigma} \eta^{\rho]\nu}}{(z-w)^2}, \\
		N^{\mu\nu}(z) \lambda^\alpha(w) 
		&\sim \frac{1}{2} \frac{(\gamma^{\mu\nu})^\alpha{}_\beta \lambda^\beta(w)}{z-w}.
	\end{aligned}
\end{equation}