\begin{acknowledgements}
	回望大学四年只不过是我人生大尺度结构的微小量子涨落。但与我世界体碰撞过的每一个人的热心帮助才得到了如今美妙的散射振幅。在此我想衷心感谢他们!
	
	首先需要感谢本科期间的学术导师杜一剑老师,感谢他带领我进入理论物理的奇妙世界。杜老师从不吝啬对我的夸赞和鼓励,极大增强了我的自信心,同时他对学术的热爱也时刻影响着我。本科低年级时肖孟和吴昊老师对我的指导也让我明确了本科的学习方向。周国全老师教授的电磁学、电动力学和数学物理方法是我本科期间听过的最好的物理课,周老师风趣幽默文理兼修,难以忘怀他在黑板上精彩的演绎。吴冯成老师的量子力 学课程也让我获益良多。数学学院李绎楠老师教会了我不少当下热门的量子计算与量子信息知识,程哲驰老师以及陈煜辉等同学让我体会到了拓扑学的魅力。他们的帮助无疑为我今后的学习打下了坚实的基础。
	
	特别感谢理论所的何颂老师允许我在北京访问一个月,同时也感谢曹趣和朱凡师兄的热心接纳。我更不会忘记那个暑假在吉林大学与师兄刘杰希和师姐谢崇思一起度过的难忘岁月。当然,学术之路总是充满遗憾,虽然最后没能参与黄宇廷教授的课题组开展暑期研究,没能同弦论大师Nathan Berkovits教授攻读博士学位,但他们对我的肯定给了我走出低谷的勇气,给了我再度出发的力量!
	
	在武大的最后一年半我组织了三届理论物理讨论班,感谢同学们的热情参与,特别感谢王澳洲、张加楠、张子锐、沈正、陈俊烨、柳淇瀚和俞千野同学,没有他们,讨论班也不会成功举办。也特别感谢黄晨、陶一笑、聂俊雄、肖瑞灏和童心海学长对讨论班的大力支持,他们在学业、留学申请以及本论文写作上都对我有很大帮助。尤其是黄晨学长的弦论笔记教会了我许多。也愿讨论班这份自发形成的学术火炬能在珞珈山下继续传递。
	
	物理之外,更需感谢那些为我的生活填满色彩的朋友。感谢管知鱼关键时刻给我放电影看。感谢董可和冯俊杰容忍我在他们宿舍背诵雅思单词,并向我介绍《炉石传说》这款超赞的游戏,也无比怀念那段一起做实验参加CUPT的时光。感谢盛宇辰和陈牧天与我一起吐槽,排忧解闷。感谢张文俊、黄龙杰、辛知雨、雷雨声、汤志豪、李周博、刘洋等哥们一声“上号”的随叫随到。感谢“青青草原”群聊全体兄弟姐妹以及胡与何自高中以来的陪伴。与他们寒暑假的相聚永远是我最期盼的日子。最后也感谢超对称自发破缺机制让我无法遇见我的伴侣粒子,得以让本篇毕业论文在单粒子近似下高效完成!
	
	感谢爸妈的经济支持与精神鼓励,感谢爷爷奶奶寒暑假的悉心照顾让我在寒暑假依然能安心学习,祝爷爷奶奶身体健康!
\end{acknowledgements}
